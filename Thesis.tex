%%%%%%%%%%%%%%%%%%%%%%%%%%%%%%%%%%%%%%%%%
% OIST Doctoral Thesis
% LaTeX Template
% Version 0.3 (2018/03)
%
% Original author:
% Jeremie Gillet
%
%%%%%%%%%%%%%%%%%%%%%%%%%%%%%%%%%%%%%%%%%

%-------------------------------------------------------------------------------
%	REQUIRED PACKAGES AND  CONFIGURATIONS
%-------------------------------------------------------------------------------
\documentclass[final]{oist_thesis} % Final version for thesis submission

% The documentclass oist_thesis includes the following packages: geometry, caption, xkeyval

\usepackage[english]{babel} % The document is in English
\usepackage[utf8]{inputenc} % UTF8 encoding
\usepackage[T1]{fontenc} % Font encoding

\usepackage{graphicx} % For including images
\graphicspath{{./Images/}} % Specifies the directory where pictures are stored

\usepackage{eso-pic} % For the background picture on the title page

\usepackage{setspace} % For using single or double spacing
\usepackage{longtable} % tables that can span several pages
\usepackage{pdfpages} % To include a pdf files of your published papers as an appendix
\usepackage{fancyhdr} % For the headers
\usepackage[hidelinks]{hyperref} % Adds clickable links at references
\usepackage{subcaption}

%----------------------------------------------------------------------------------------
%	ADD YOUR PACKAGES (be careful of package interaction)
%----------------------------------------------------------------------------------------

\usepackage{amsthm, amsmath, amssymb, amsfonts, bbm, bm} % Math symbols
\usepackage{fontspec} % More fonts
% \usepackage[parfill]{parskip} % Gaps instead of indents
\usepackage{gensymb} % \degree symbol
\usepackage{lipsum}% http://ctan.org/pkg/lipsum
\usepackage{flafter}
\usepackage{notoccite}
\usepackage{booktabs}
\usepackage[export]{adjustbox}
\usepackage{graphbox}

%----------------------------------------------------------------------------------------
%	ADD YOUR DEFINITIONS AND COMMANDS
%----------------------------------------------------------------------------------------

% Example of New Commands
\newcommand{\bea}{\begin{eqnarray}} % Shortcut for equation arrays
\newcommand{\eea}{\end{eqnarray}}
\newcommand{\e}[1]{\times 10^{#1}}  % Powers of 10 notation

% Example of Defining a theorem box for Criteria 
\newtheorem{critere}{Criterion}
\newcommand{\crit}[2]{
	\begin{center}
		\fbox{ \begin{minipage}[c]{0.9 \textwidth}
				\begin{critere}
					\textbf{\textup{ #1}} --- #2
				\end{critere}
	\end{minipage}  } \end{center}
}

% Make new figures appear on their own page immediately after their first reference.
\renewcommand\floatpagefraction{0} % default 0.5


%----------------------------------------------------------------------------------------
%	PICK YOUR BIBLIOGRAPHY STYLE
%----------------------------------------------------------------------------------------

\usepackage[square, numbers, sort&compress]{natbib} % for bibliography - Square brackets, citing references with numbers, citations sorted by appearance in the text and compressed (as in [4-7])

% \bibliographystyle{Preasmble/physics_bibstyle} % You may use a different style adapted to your field
% \bibliographystyle{abbrvnat} % You may use a different style adapted to your field
\bibliographystyle{unsrtnat}

%-------------------------------------------------------------------------------
%	TITLE PAGE
%-------------------------------------------------------------------------------

\begin{document}
% \setmainfont{Arial}
\pagestyle{empty} % No page numbers
\frontmatter % Use roman page numbering style (i, ii, iii, iv...) for the preamble pages

\puttitle{
	title=Towards the production of a self-consistent phase space distribution, % Title of the thesis
	name=Austin Hoover, % Author name
	supervisor=Sarah Cousineau, % Supervisor name
	cosupervisor=Nick Evans, % Co-Supervisor name, remove this line if there is none
	submissiondate={May 2022}  % Submission date "Month, year"
}

%-------------------------------------------------------------------------------
%	PREAMBLE PAGES (delete unnecessary pages)
%-------------------------------------------------------------------------------

\startpreamble

% \input{Preamble/declaration}
\unnumberedchapter{Acknowledgments} 
\chapter*{Acknowledgments} 

First, I would like to acknowledge all those at the University of Tennessee who helped me succeed during my graduate studies. I thank my classmates for their camaraderie as we solved difficult problems during our coursework. I also thank my instructors: I have frequently referred to notes from Steven Johnston’s course in mathematical methods, Kenneth Read’s course in numerical methods, and Christian Batista’s course in statistical mechanics. I especially thank Dr. Batista for his dedication to his graduate students during three semesters of study, including holding extra recitations, providing Krispy Kreme donuts to help us through long exams, and scoring the most goals on our intramural soccer team. Outside of Tennessee, I thank Alex Bogacz, Geoffrey Krafft, and Yuri Batygin for their lectures at the United States Particle Accelerator School (USPAS), as well as my instructors at Wheaton College: Darren Craig, Danilo Diedrichs, A. J. Poelarends, Stewart DeSoto, Heather Whitney, and Joshua Whitney. 

Next, I would like to acknowledge the members of the Accelerator Physics Group at the Spallation Neutron Source (SNS). Sarah Cousineau served as my advisor and provided feedback, encouragement, and professional guidance throughout my Ph.D. research. I worked most closely with Nick Evans, who joined me for many problem-solving sessions, provided extensive feedback on my writing and speaking, and allowed me to grow as a researcher. I will always remember our marathon experiments in the SNS control room. I also worked closely with Jeff Holmes, whom I thank for answering many questions about physics and other topics, as well as for his words of encouragement. I also thank Andrei Shishlo and Sasha Zhukov for assistance with the PyORBIT and OpenXAL computer codes, Wim Blockland for assistance with the wire-scanners and target imaging system (and for many free cups of espresso), Charles Peters for leading the setup up the SNS accelerator for our experiments, and Vasiliy Morozov for helpful discussions.

Finally, I could not have completed this dissertation without my family, especially my wife Paris, who provided a steady stream of love and support during the past three and a half years.
\unnumberedchapter{Abstract} 
\chapter*{Abstract} 

Maintaining low levels of beam loss will be increasingly difficult in future high-intensity accelerators. One limitation in circular accelerators is the space charge tune shift — the damping of single-particle oscillations due to the beam's electric field — which, when combined with the periodic electromagnetic fields in the accelerator, can lead to resonant oscillations and eventual particle loss. This problem is minimized if the beam's electric field has a linear dependence on the particle coordinates, but such a beam is difficult to realize and/or maintain in realistic conditions.

Self-consistent phase space distributions are beams in which the electric field has a linear dependence on the particle coordinates, and furthermore, in which the linearity of the electric field is conserved as the beam is transported through arbitrary linear focusing fields. Although such models are often used in theoretical analysis, simulations predict that at least one self-consistent distribution — the Danilov distribution — could be approximately produced in a real machine — the Spallation Neutron Source (SNS) — using a method called elliptical painting. This dissertation contributes to efforts to test this prediction in the SNS. 

First, the beam envelope model was employed to improve understanding of the dynamics of the Danilov distribution with space charge and place constraints on the elliptical painting method. Second, several existing methods to indirectly measure the four-dimensional (4D) phase space distribution of a fully-accumulated beam in the SNS were identified, implemented using existing diagnostics, and optimized, allowing comparison of real beams with the ideal Danilov model. Finally, three initial experiments to produce a Danilov distribution in the SNS were carried out. The experiments were performed under suboptimal conditions due to current hardware constraints; nonetheless, the reduction of the measured 4D beam emittance in the final experiment was not insignificant, indicating that the beam was closer to the desired case than a typical beam in the SNS. Particle-in-cell simulations were included to benchmark the measurements, resulting in qualitative agreement. Small modifications to the SNS are expected to bring the beam closer to a self-consistent state.
% \input{Preamble/abbreviations}
% \unnumberedchapter{Glossary} 
\chapter*{Glossary} 

Because accelerator physics is a small branch of applied physics, and because this dissertation introduces several terms that are not frequently used in the accelerator physics community, the following glossary has been included for the reader.

\textbf{Beta function} — A function that scales the amplitude of single-particle oscillations in the linear approximation.

\textbf{Beam envelope} — The root-mean-square ellipsoid defined by the covariance matrix.

\textbf{Beam perveance} — A dimensionless measure of space charge strength.

\textbf{Circular mode} — A beam with small four-dimensional emittance. It could also refer to the circular motion of the eigenvectors of a coupled transfer matrix.

\textbf{Courant-Snyder ellipse} — Particles move along this ellipse in the linear approximation. Its area is conserved.

\textbf{Danilov distribution} — A self-consistent distribution in two spatial dimensions. It is characterized by an elliptical shape, uniform charge density, and zero four-dimensional emittance.

\textbf{Effective lattice} —

\textbf{Emittance} — The root-mean-square volume or area of a phase space distribution.

\textbf{Kapchinskij-Vladimirskij (KV) distribution} — A self-consistent distribution in two spatial dimensions. Its particles are uniformly distributed on an ellipsoid in four-dimensional phase space.

\textbf{Matched beam} — A beam whose envelope oscillates with the same periodicity as the external focusing.

\textbf{Ring} — A circular accelerator.

\textbf{Painting} — An beam injection method in which the relative transverse distance and angle between the circulating and injected beam is varied.  

\textbf{Phase advance} — The integral of the inverse of the beta function. 

\textbf{Self-consistent distribution} — A phase space distribution which produces linear space charge forces under any linear transformation of the coordinates.

\textbf{Space charge} — The charge density of a beam in free space.

\textbf{Tune} — The number of phase space oscillations per performed by a single particle in one turn around a ring.

\textbf{Space charge tune shift} — The reduction in tune caused by the beam space charge. The linear space charge component results in the same reduction for every particle (tune shift), while the nonlinear component results in an amplitude-dependent reduction (tune spread).
% \input{Preamble/nomenclature}
% \input{Preamble/dedication}

%-------------------------------------------------------------------------------
%	LIST OF CONTENTS/FIGURES/TABLES
%-------------------------------------------------------------------------------

\singlespacing
\unnumberedchapter{Table of Contents}
\tableofcontents % Write out the Table of Contents
\thesisspacing
\unnumberedchapter{List of Figures}
\listoffigures % Write out the List of Figures
% \unnumberedchapter{List of Tables}
% \listoftables % Write out the List of Tables

%-------------------------------------------------------------------------------
%	THESIS MAIN TEXT
%-------------------------------------------------------------------------------
\addtocontents{toc}{\vspace{2em}} % Add a gap in the Contents, for aesthetics
\mainmatter % Begin numeric (1,2,3...) page numbering

\numberedchapter
\chapter{Introduction}\label{chap-1}

The maximum beam intensity in low-energy particle accelerators is limited by nonlinear space charge forces — forces between the charged particles in the beam. In the collisionless approximation, the evolution of a distribution of charged particles is given by the Vlasov equation. Equilibrium solutions to the Vlasov equation are difficult to find in the general case of linear time-dependent external forces. The few known solutions are those that generate linear space charge forces under any linear transformation. These solutions are referred to as self-consistent phase space distributions. This dissertation investigates whether one such distribution — the Danilov distribution — could be physically realized using a method called elliptical painting, particularly in the Spallation Neutron Source (SNS). In addition to the interesting theoretical properties stemming from its self-consistent nature, the vanishing four-dimensional emittance of the Danilov distribution makes it potentially useful for colliders and beam cooling applications.

The structure of this introductory chapter is as follows. The relevant theory of high-intensity beam dynamics is reviewed in sections \ref{sec:Single-particle motion}-\ref{sec:Space charge}. The definition and properties of self-consistent distributions are discussed in \ref{sec:Self-consistent phase space distributions}, starting with the well-known KV distribution and ending with the lesser-known Danilov distribution. A method to generate an approximate Danilov distribution in a ring, as well as the implementation of the method in the SNS, is presented in section \ref{sec:Physically realizing a self-consistent distribution}. The project is further motivated by investigating the connection between the Danilov distribution and the recent interest in circular modes. Finally, the structure and goals of this dissertation are laid out in section \ref{sec:Goals of this dissertation}.




\section{Single-particle motion}\label{sec:Single-particle motion}

We begin by describing the motion of a single particle in a circular accelerator (ring). We assume the existence of a closed orbit and use curvilinear coordinates in which $s$ is the location along the closed orbit and $x$ and $y$ are the horizontal and vertical transverse displacements. We then study oscillations in the transverse plane with the assumption of constant longitudinal velocity $\beta c$, where $c$ is the speed of light.

Magnetic fields are preferred for transverse focusing when the kinetic energy is significant. The magnetic field $\mathbf{B} = (B_x, B_y)$ may be written as an infinite sum:
%
\begin{equation}\label{eq:magnetic_field_expansion}
    B_x - iB_y = \sum_{n = 1}^{\infty}{(b_n - i a_n) \left({\frac{x + i y}{r_0}}\right)^{n - 1}},
\end{equation}
%
where $r_0$ is a constant, $\left\{ b_n \right\}$ are the multipole coefficients, and $\left\{ a_n \right\}$ are the skew multipole coefficients. The $b_n$ term in the expansion is produced by $2n$ symmetrically arranged magnetic poles; the skew terms are obtained by a $45\degree$ rotation. Assuming the transverse velocities are much smaller than $\beta c$, the equations of motion for $x$ and $y$ are
%
\begin{equation}\label{eq:transverse_eom}
\begin{aligned}
    x'' &= -\frac{q}{m \beta c} B_y, \\
    y'' &= -\frac{q}{m \beta c} B_x,
\end{aligned}
\end{equation}
%
where $q$ is the particle charge and $m$ is the particle mass, and the prime represents differentiation with respect to $s$.


\subsection{Linear dynamics}

 Accelerators employ dipole fields ($b_1$) for bending and quadrupole fields ($b_2$) fields for focusing. Keeping only these terms, Eq.~\eqref{eq:transverse_eom} can be written as
%
 \begin{equation}\label{eq:Hill}
     x'' + k(s)x = 0,
 \end{equation}
%
with $k(s) = k(s + L)$ for some $L$. Eq.~\eqref{eq:Hill} is of general interest \cite{Qin2007}. It describes a parametric oscillator — an oscillator whose physical properties change with time. Its solution is described by the Courant-Snyder theory \cite{Courant1958} as
%
\begin{equation}\label{eq:Hill_solution}
    x(s) = \sqrt{2 J \beta(s)} \cos{\left({\mu(s) + \delta}\right)},
\end{equation}
%
with $J$ constant, $\beta(s + L) = \beta(s)$, and the phase advance given by
%
\begin{equation}
    \mu(s) = \int_{0}^{s}{\frac{ds'}{\beta(s')}}.
\end{equation}

It is helpful to view the motion in phase space ($x$-$x'$) at a fixed location in the ring on a turn-by-turn basis as in Fig.~\ref{fig:cs_ellipse}.
%
\begin{figure}
    \centering
    \includegraphics[width=0.6\textwidth]{Images/chapter1/cs_ellipse.png}
    \caption{Courant-Snyder ellipse in horizontal phase space.}
    \label{fig:cs_ellipse}
\end{figure}
%
The particle jumps around the boundary of an ellipse. The so-called Twiss parameters $\beta$, $\alpha = -\beta' / 2$, and $\gamma = (1 + \alpha^2) / \beta$ determine the ellipse dimensions. $J$ is called the Courant-Snyder invariant and is proportional to the area of the ellipse, which is conserved due to Liouville's theorem:
%
\begin{equation}\label{eq:CS invariant}
    J = \frac{x^2 + (\alpha x + \beta x')^2}{\beta}.
\end{equation}
%
We define the tune $\nu$ as the number of phase space oscillations per turn; i.e.,
%
\begin{equation}
    2\pi\nu = \oint{\frac{ds}{\beta(s)}},
\end{equation}
%
where the integral is around the entire ring.

Thus, motion between two locations in the ring is equivalent to an area-preserving linear transformation of a phase space ellipse, plus rotation of the particle around the ellipse. This is more clear in the transfer matrix formulation of the dynamics, writing $\mathbf{x}(s) = \mathbf{M}(s)\mathbf{x}(0)$ where
%
\begin{equation} \label{eq:CS_parameterization}
\begin{aligned}
    \mathbf{M}(s) &= 
    \begin{bmatrix} 
        \sqrt{\beta(s)} & 0 \\
        -\frac{\alpha(s)}{\sqrt{\beta(s)}} & \sqrt{\frac{1}{\beta(s)}}
    \end{bmatrix}
    \begin{bmatrix} 
        \cos\mu(s) & \sin\mu(s) 
        \\ -\sin\mu(s) & \cos\mu(s) 
    \end{bmatrix}
    \begin{bmatrix} 
        \sqrt{\frac{1}{\beta(0)}} & 0 \\
        \frac{\alpha(0)}{\sqrt{\beta(0)}} & \sqrt{\beta(0)}
    \end{bmatrix} \\
    &= \mathbf{V}(s) \, \mathbf{R}(s) \, \mathbf{V}(0)^{-1}. 
\end{aligned}
\end{equation}
%
$\mathbf{V(0)}^{-1}$ transforms the phase space ellipse into a circle while preserving its area, $\mathbf{R(s)}$ rotates the coordinates around the circle according to the phase advance, and $\mathbf{V(s)}$ transforms the circle back into an ellipse \cite{Lee2011}. 

Eq.~\eqref{eq:CS_parameterization} motivates the definition of normalized phase space coordinates $\mathbf{x}_n(s) = \mathbf{V}(s)^{-1} \mathbf{x}(s)$ in which the particle performs simple harmonic oscillations; i.e., rotates in a circle of area $J$ at frequency $2\pi\nu$. 


\subsection{Linear (coupled) dynamics}

In the presence of linear coupling, Eq.~\eqref{eq:Hill} takes the following form:
%
\begin{equation}\label{eq:single_particle_eom_coupled}
\begin{aligned}
    x'' + k_{11}(s)x + k_{13}(s)y + k_{14}(s)y' &= 0, \\
    y'' + k_{33}(s)y + k_{31}(s)x + k_{32}(s)x' &= 0,
\end{aligned}
\end{equation}
%
Liouville's theorem is no longer valid in two-dimensional (2D) phase space; instead, the particle moves along the boundary of an ellipsoid in 4D phase space ($x$-$x'$-$y$-$y'$). For example, Fig.~\ref{fig:skew_quad_single_particle_tbt} shows the turn-by-turn trajectory of a single particle in the presence of linear coupling from a rotated (skew) quadrupole.
%
\begin{figure}[!p]
    \centering
    \includegraphics[width=0.8\textwidth]{Images/chapter1/skew_quad_single_particle_tbt.png}
    \caption{Turn-by-turn trajectory of a particle in a linear lattice with the addition of a skew quadrupole.}
    \label{fig:skew_quad_single_particle_tbt}
\end{figure}
%
The motion is most simply described using transfer matrices. Consider the eigenvectors and eigenvalues of the $4 \times 4$ symplectic one-turn transfer matrix $\mathbf{M}$. There are four eigenvectors — $\mathbf{v}_1$, $\mathbf{v}_2$, $\mathbf{v}_1^*$, $\mathbf{v}_2^*$ — and four eigenvalues — $\lambda_1$, $\lambda_2$, $\lambda_1^*$, $\lambda_2^*$ — with $\lambda_1\lambda_2 = 1$ (* denotes the complex conjugate). The eigenvalue equation is written as
%
\begin{equation} \label{eq:transfer_matrix_eig}
    \mathbf{M} \mathbf{v}_l = e^{-i\mu_l} \mathbf{v}_l,
\end{equation}
%
with $l = 1,2$. The phase space coordinate vector $\mathbf{x} = (x, x', y, y')^T$ at one position in the ring is a linear combination of the eigenvectors:
%
\begin{equation}
    \mathbf{x} = Re \left\{
        \sqrt{2 J_1} \, \mathbf{v}_1 \, e^{-i\psi_1}
        + \sqrt{2 J_2} \, \mathbf{v}_2 \, e^{-i\psi_2}
    \right\},
\end{equation}
%
where $J_{1,2}$ are constant amplitudes and $\psi_{1,2}$ are initial phases. Application of the transfer matrix advances the phases:
%
\begin{equation}\label{eq:eigvec_coords}
    \mathbf{Mx} = Re \left\{
        \sqrt{2 J_1} \, \mathbf{v}_1 \, e^{-i(\psi_1 + \mu_1)}
        + \sqrt{2 J_2} \, \mathbf{v}_2 \, e^{-i(\psi_2 + \mu_2)}
    \right\}.
\end{equation}
%
The old invariants $J_{x,y}$ are replaced by $J_{1,2}$ and the phase advances $\mu_{x,y}$ are replaced by $\mu_{1,2}$. A new normalized phase space is defined by rewriting Eq.~\eqref{eq:eigvec_coords} as $\mathbf{x}_n = \mathbf{V}^{-1} \mathbf{x}$ with
%
\begin{equation}\label{eq:V_from_eigvecs}
    \mathbf{V} = 
    \begin{bmatrix}
        Re\{\mathbf{v}_1\}, & -Im\{\mathbf{v}_1\}, & Re\{\mathbf{v}_2\}, & -Im\{\mathbf{v}_2\}
    \end{bmatrix}.
\end{equation}
%
Particles perform simple harmonic oscillations in normalized phase space, moving in circles of area $J_1$ in the $x_n$-$x_n'$ plane and $J_2$ in the $y_n$-$y_n'$ plane.

We would like to parameterize the eigenvectors as in the uncoupled case. There are currently several parameterizations in existence \cite{Edwards1973, Ripken1989, Wolski2006, Lebedev2010, Qin2009}; we will use the parameterization of Lebedev and Bogacz \cite{Lebedev2010}:
%
\begingroup
\renewcommand*{\arraystretch}{1.1}
\begin{equation}
\begin{aligned}
    \mathbf{v}_1 = 
    \begin{bmatrix}
        \sqrt{\beta_{1x}} \\
        -\frac{\alpha_{1x} + i(1-u)}{\sqrt{\beta_{1x}}} \\
        \sqrt{\beta_{1y}}e^{i\nu_1} \\
        -\frac{\alpha_{1y} + iu}{\sqrt{\beta_{1y}}} e^{i\nu_1} \\
    \end{bmatrix} ,\quad
    \mathbf{v}_2 = 
    \begin{bmatrix}
        \sqrt{\beta_{2x}}e^{i\nu_2} \\
        -\frac{\alpha_{2x} + iu}{\sqrt{\beta_{2x}}}e^{i\nu_2} \\
        \sqrt{\beta_{2y}} \\
        -\frac{\alpha_{2y} + i(1-u)}{\sqrt{\beta_{2y}}} \\
    \end{bmatrix}.
\end{aligned}
\end{equation}
\endgroup
%
The meaning of the new parameters is illustrated in Fig.~\ref{fig:twiss4D}.
%
\begin{figure}[!p]
    \centering
    \includegraphics[width=\textwidth]{Images/chapter1/twiss4D.png}
    \vspace*{0.1cm}
    \caption{Lebedev-Bogacz parameterization of coupled motion. The grey markers are the turn-by-turn trajectory of a single particle. The red and blue lines are the ellipses traced by the transfer matrix eigenvectors.}
    \label{fig:twiss4D}
\end{figure}
%
The motion is simply the sum of two eigenvectors, each of which traces an ellipse when projected onto any 2D subspace. The frequency with which the ellipse is traced differs between the two modes; as a consequence, the horizontal and vertical amplitudes $J_x$ and $J_y$ are exchanged. The parameterization assigns a $\beta$ and $\alpha$ parameter to each ellipse. The parameters $\nu_1$ and $\nu_2$ are the phase differences between the horizontal ($x$-$x'$) and vertical ($y$-$y'$) parts of the eigenvectors, which determines the tilt angle of the ellipses traced in the cross-plane projections ($x$-$y$, $x$-$y'$, $y$-$x'$, $x'$-$y'$). Finally, $u$ determines the area of the ellipse traced by the eigenvectors in horizontal phase space relative to the ellipse in vertical phase space. 


\subsection{Nonlinear resonances}

Nonlinear terms in Eq.~\eqref{eq:magnetic_field_expansion} are generally small but nonzero in reality. Following \cite{LundLecture1}, we return to one-dimensional motion and write
%
\begin{equation}\label{eq:Hill_nonlinear}
    x'' + k(s) x = \Delta B,
\end{equation}
%
where $\Delta B$ represents all the nonlinear terms in the expansion (and also linear deviations from the design fields). The stable solution $x_0$ when $\Delta B = 0$ is given by Eq.~\eqref{eq:Hill_solution}. We now define
%
\begin{equation}
    \phi(s) = \frac{1}{\nu} \oint{\frac{ds}{\beta(s)}},
\end{equation}
%
where $\nu$ is the tune. Moving to the normalized coordinate $u = x / \sqrt{\beta}$, with $\dot{u} = du/d\phi$ we have
%
\begin{equation}\label{eq:pert1}
    \ddot{u} + \nu^2 u = -\nu^2 \sum_{n=0}^{\infty}{\left(\beta^{\frac{n+3}{2}} b_{n+1}\right) u^n}.
\end{equation}
%
$\beta$ (the oscillation amplitude of the unperturbed motion) and $b_n$ (a multipole coefficient) are periodic in $\phi$ since they depend only on the position in the ring. Grouping these terms and Fourier expanding gives
%
\begin{equation}
    \ddot{u} + \nu^2 u = -\nu^2 \sum_{n=0}^{\infty}\sum_{k=-\infty}^{\infty} C_{n,k} \, u^n \, e^{ik\phi}.
\end{equation} 
%
We then perturb around $u_0$, the solution to the homogeneous equation, writing $u = u_0 + \delta u$, and keep only linear powers of $\delta u$. 
%
\begin{equation}
    \ddot{\delta u} + \nu^2 \delta u \approx -\nu^2 \sum_{n=0}^{\infty}\sum_{k=-\infty}^{\infty} C_{n,k} \, u_0^n \, e^{ik\phi}.
\end{equation}
%
Noting that
%
\begin{equation}
    u_0^n \propto \cos^n(\nu\phi) = \frac{1}{2^n}\sum_{m=0}^{n} \binom{n}{m} e^{i(n-2m)\nu\phi},
\end{equation} 
%
leads to
%
\begin{equation}\label{eq:pert2}
    \ddot{\delta u} + \nu^2 \delta u \approx -\nu^2 \sum_{n=0}^{\infty}\sum_{k=-\infty}^{\infty} \sum_{m=0}^{n} {n \choose m} \frac{C_{n,k}}{2^n} e^{i\left[(n - 2m)\nu + k\right]\phi}.
\end{equation}
%
A resonance condition may occur when any of the frequency components of the driving terms are close to the tune $\nu$; i.e., when
%
\begin{equation}
    (n - 2m)\nu + k = \pm \nu.
\end{equation}
%
Dipole terms correspond to integer tunes, quadrupole terms to 1/2 integer tunes, sextupole terms to 1/3 integer tunes, and so on. The same is true in the vertical dimension. The inclusion of coupling between $x$ and $y$ leads to the following resonance conditions:
%
\begin{equation}\label{eq:resonance_lines}
    M_x \nu_x + M_y \nu_y = N,
\end{equation}
%
where $M_x$, $M_y$, and $N$ are integers and $|M_x| + |M_y|$ is the order of the resonance. These resonance lines are plotted in Fig.~\ref{fig:resonance_lines}.
%
\begin{figure}[!p]
    \centering
    \includegraphics[width=\textwidth]{Images/chapter1/resonance_lines.png}
    \caption{Resonance lines in tune space defined by Eq.~\eqref{eq:resonance_lines}.}
    \label{fig:resonance_lines}
\end{figure}
%
\begin{figure}[!p]
    \begin{subfigure}[b]{1.0\textwidth}
        \includegraphics[width=\textwidth]{Images/chapter1/sextupole.png}
        \label{fig:sextupole_a}
    \end{subfigure}
    \vfill
    \vspace*{1.0cm}
    \vfill
    \begin{subfigure}[b]{\textwidth}
        \centering
        \includegraphics[width=\textwidth]{Images/chapter1/sextupole_second_order.png}
        \label{fig:sextupole_b}
    \end{subfigure}
    \caption{Third-order (top) and fourth/fifth-order (bottom) resonances excited by a sextupole perturbation to a linear lattice. (Adapted from \cite{Lee2011}.)}
    \label{fig:sextupole}
\end{figure}
%
The strength of the resonance varies inversely with the order; fourth-order and below are the primary concern in most machines, but higher-order effects may be important when the number of stored turns is large. 

Nonlinear resonances can be studied quickly using mapping equations \cite{Reichl1992}. For illustration, Fig.~\ref{fig:sextupole} shows two numerical experiments from \cite{Lee2011} involving a sextupole perturbation to an otherwise linear lattice. The turn-by-turn trajectories of particles with several different initial amplitudes are plotted for different tunes $\nu_x$. The third-order resonance leads to a well-known triangular region of stability as the tune approaches 2/3. The bottom plot reveals fourth and fifth-order resonances only obtained from second-order perturbation analysis.


\section{Collective beam description}

A beam is a distribution of particles in phase space. In the limit of many particles, we define a distribution function $f(\mathbf{x})$ such that $f(\mathbf{x}) d\mathbf{x}$ gives the number of particles in an infinitesimal volume of phase space $d\mathbf{x}$. The measurable quantities are generally the projections of the distribution; e.g.
%
\begin{equation}
    f(x) = \int_{-\infty}^{\infty}\int_{-\infty}^{\infty}\int_{-\infty}^{\infty} f(x, x', y, y') dx' dy dy'.
\end{equation}
%

It is often sufficient to characterize a distribution by its covariance matrix {$\bm{\Sigma} = \langle{\mathbf{x}\mathbf{x}^T}\rangle$}, where $\langle{\dots}\rangle$ represents the average over the distribution. In the transverse plane:
%
\begin{equation}\label{eq:covariance_matrix}
\begin{aligned}
    \bm{\Sigma} &= 
    \begin{bmatrix}
        \langle{xx}\rangle & \langle{xx'}\rangle & \langle{xy}\rangle & \langle{xy'}\rangle \\
        \langle{xx'}\rangle & \langle{x'x'}\rangle & \langle{x'y}\rangle & \langle{x'y'}\rangle \\
        \langle{xy}\rangle & \langle{x'y}\rangle & \langle{yy}\rangle & \langle{yy'}\rangle \\
        \langle{xy'}\rangle & \langle{x'y'}\rangle & \langle{yy'}\rangle & \langle{y'y'}\rangle 
    \end{bmatrix}
    &= 
    \begin{bmatrix}
        \bm{\sigma}_{xx} & \bm{\sigma}_{xy} \\
        \bm{\sigma}_{xy}^T & \bm{\sigma}_{yy}
    \end{bmatrix}.
\end{aligned}
\end{equation}
%
If a linear transformation $\mathbf{x} \rightarrow \mathbf{M}\mathbf{x}$ is applied to the coordinates, the covariance matrix transforms as
%
\begin{equation}\label{covariance_matrix_transport}
    \bm{\Sigma} 
    \rightarrow 
    \mathbf{M} \, \bm{\Sigma} \, \mathbf{M}^T.
\end{equation}
%
The covariance matrix defines an ellipsoid in phase space. The 4D emittance $\varepsilon_{4D}$ is proportional to the volume of this ellipsoid:
%
\begin{equation} 
    \varepsilon_{4D} = \left|{\Sigma}\right|^{1/2}
\end{equation}
%
where $|...|$ is the determinant. The 4D emittance is conserved under any linear transformation. The horizontal and vertical emittances $\varepsilon_{x,y}$ are individually conserved if the transformation is uncoupled:
%
\begin{equation}
\begin{aligned}
    \varepsilon_x &= \left|{\bm\sigma}_{xx}\right|^{1/2}, \\
    \varepsilon_y &= \left|{\bm\sigma}_{yy}\right|^{1/2}
\end{aligned}
\end{equation}
%
These correspond to the areas in the $x$-$x'$ and $y$-$y'$ planes. In the absence of cross-plane correlations ($\bm{\sigma}_{xy} = 0$), the 4D emittance is equal to the product of the horizontal and vertical emittances.

It is challenging to generate initial distributions for simulations \cite{Lund2009}. For simple applications, a common strategy is to assume elliptical symmetry and construct a distribution from the single-particle invariants $J_{x,y}$. We define the ellipsoid parameter $T$ as 
%
\begin{equation}
    T = \frac{J_x}{\varepsilon_x} + \frac{J_y}{\varepsilon_y}
\end{equation}
%
and stack ellipsoids to create the distribution, writing $f = f(T)$. One option is a Gaussian distribution. In normalized coordinates:
%
\begin{equation}
    f_{gauss} \propto \exp(-T/2).
\end{equation}
%
Another is the Waterbag distribution, which is a uniformly filled ellipsoid:
%
\begin{equation}
    f_{wb} \propto \Theta(1 - T),
\end{equation}
%
where $\Theta$ is the Heaviside step function. Another is the KV distribution, which is a uniformly populated ellipsoidal shell:
%
\begin{equation}
    f_{kv} \propto \delta(1 - T)
\end{equation}
%
The 1D and 2D projections of these 4D distributions are shown in Fig.~\ref{fig:distributions_gaussian}, Fig.~\ref{fig:distributions_waterbag}, and Fig.~\ref{fig:distributions_kv}. The black ellipse shows the covariance matrix ellipsoid projected onto the planes, multiplied by a factor of four. Since the distributions share the same covariance matrix, they are said to be rms-equivalent.
%
\begin{figure}[!p]
    \begin{subfigure}{0.49\textwidth}
        \includegraphics[width=\textwidth]{Images/chapter1/Gaussian_dist.png}
        \caption{Gaussian distribution}
        \label{fig:distributions_gaussian}
    \end{subfigure}
    \hfill
    \begin{subfigure}{0.49\textwidth}
        \includegraphics[width=\textwidth]{Images/chapter1/Waterbag_dist.png}
        \caption{Waterbag distribution}
        \label{fig:distributions_waterbag}
    \end{subfigure}
    \vfill
    \begin{subfigure}{0.49\textwidth}
        \includegraphics[width=\textwidth]{Images/chapter1/KV_dist.png}
        \caption{KV distribution}
        \label{fig:distributions_kv}
    \end{subfigure}
    \hfill
    \begin{subfigure}{0.49\textwidth}
        \includegraphics[width=\textwidth]{Images/chapter1/Danilov_dist.png}
        \caption{Danilov distribution}
        \label{fig:distributions_danilov}
    \end{subfigure}
    \caption{1D and 2D projections of various 4D phase space distributions. Black ellipses are defined by four times the distribution covariance matrix.}
    \label{fig:distributions}
\end{figure}
%





\section{Space charge}\label{sec:Space charge}

Particle motion is also influenced by the beam space charge — the charge density of the beam in free space. The beam electric field $\mathbf{E} = (E_x, E_y)$ modifies the single-particle equation of motion:
%
\begin{equation} \label{eq:eom_with_spacecharge}
\begin{aligned}
    x'' + k_x(s)x &= \frac{q}{m \gamma^3 \beta^2 c^2} E_x, \\
    y'' + k_y(s)y &= \frac{q}{m \gamma^3 \beta^2 c^2} E_y,
\end{aligned}
\end{equation}
%
where $\gamma = \left( 1 - \beta^2 \right)^{-1/2}$. The Lorentz factors in the denominator of the space charge term are due to the attractive magnetic force between co-moving charges in the lab frame which cancels the electric force as $\beta \rightarrow 1$. We will make the coasting beam approximation — infinite length, uniform density, and constant momentum in the longitudinal plane — to reduce the problem to two dimensions. (This is generally invalid for linacs but locally valid for a transverse slice of a long distribution in a ring. It is equivalent to replacing particles with infinitely long uniform density charged rods.) 

Following Hofmann \cite{Hofmann2017Book}, we divide space charge effects into two categories: incoherent effects involving the motion of single particles, and coherent effects involving the self-consistent motion of the entire beam. Although the two effects may be difficult to isolate during the beam evolution \cite{Hofmann2021}, the distinction is clear in some cases. 


\subsection{Incoherent effects}

We first assume that the beam is matched — i.e., oscillates with the same periodicity as the external focusing — and track a particle in the field of the matched beam. This may be justified if space charge is weak. The electric field may be written as a sum of products of $x$ and $y$:
%
\begin{equation}\label{eq:space_charge_multipoles}
    E_x(x, y) = \sum_{i, j}{c_{i, j} x^i y^j}.
\end{equation}
%
These terms can be treated in a similar way to the magnetic multipoles of Eq.~\eqref{eq:magnetic_field_expansion}.


\subsubsection{Space charge tune shift}

The linear terms in Eq.~\eqref{eq:space_charge_multipoles} modify the external linear focusing; therefore, the single-particle tune is reduced in both planes. A primary concern in rings is that the depressed tunes are located near one of the low-order resonance lines in Fig.~\ref{fig:resonance_lines}. Approximate analytical formulas for the tune shift can be obtained \cite{Ng2005} but are not presented here.

The nonlinear terms in Eq.~\eqref{eq:space_charge_multipoles} result in a tune shift that depends on the particle amplitude, hence a spread of tunes. An intuitive explanation is that large-amplitude particles experience a weaker average electric field throughout one turn in the ring \cite{Franchetti2017}. A recent study of the space charge tune spread in rings is found in \cite{Hotchi2020}. Fig.~\ref{fig:jparc_montague}, taken from the paper, shows the simulated effect of $x^2y^2$ terms in the space charge potential on the particle tunes in the Japan Proton Accelerator Research Center (J-PARC).
%
\begin{figure}[!p]
    \centering
    \includegraphics[width=0.6\textwidth]{Images/chapter1/montague.png}
    \caption{Simulated tune footprint in the JPARC accelerator. (From \cite{Hotchi2020})}
    \label{fig:jparc_montague}
\end{figure}
%

Thus, the beam intensity in rings is fundamentally limited by the incoherent space charge tune shift. A rough guideline is that the maximum tune shift should be kept below 0.25 to avoid fourth-order resonance lines \cite{book:Reiser}, but specific requirements depend on the application.


\subsubsection{Incoherent space charge resonances}

%
It is also possible for space charge itself to drive particle resonances \cite{Holmes1999, Jeon1999, Li2014, Kojima2019, Asvesta2020}. For illustration, we reproduce a numerical study from \cite{Hofmann2017Book} using PyORBIT \cite{Shishlo2015}, a particle-in-cell (PIC) code. Fig.~\ref{fig:incoherent_instability} shows a simulation of a truncated Gaussian distribution in a FODO lattice as the zero-current tune is decreased from 100\degree to 90\degree over 500 cells. The initial distribution has equal emittances in both planes and is matched to the lattice with a depressed tune of 92\degree.
%
\begin{figure}[!p]
    \centering
    \begin{subfigure}{\textwidth}
        \includegraphics[width=\textwidth]{Images/chapter1/incoherent_resonance_fourth_order.png}
        \label{fig:incoherent_instability_a}
        \caption{}
    \end{subfigure}
    \begin{subfigure}{0.5\textwidth}
        \includegraphics[width=\textwidth]{Images/chapter1/incoherent_resonance_fourth_order_emittance.png}
        \label{fig:incoherent_instability_b}
        \caption{}
    \end{subfigure}
    \caption{Simulation of a truncated Gaussian distribution in a FODO lattice. The zero-current tune is decreased from 100\degree to 90\degree over 500 cells. (a) $x$-$x'$ distribution. (b) RMS horizontal emittance. (Reproduced from \cite{Hofmann2017Book}.)}
    \label{fig:incoherent_instability}
\end{figure}
%
A fourth-order resonance is excited as the depressed tune approaches 90 degrees. The smooth emittance growth during most of the simulation shows that the core of the beam remains matched, justifying the use of ``incoherent" to describe the effect. Higher-order resonances can also for different combinations of beam intensity and focusing strength.


\subsection{Coherent effects}

Coherent space charge effects involve self-consistent oscillations of the entire beam and are relevant for strong space charge \cite{book:Reiser, Wangler2008}. We may model the beam as a smooth distribution in phase space $f(\mathbf{x})$; neglecting collisions between particles, the evolution of $f$ is given by the Vlasov equation \cite{Vlasov1961}:
%
\begin{equation} \label{eq:Vlasov}
    \frac{df}{ds} = \frac{\partial{f}}{\partial{s}} + \{H, f\} = 0,
\end{equation}
%
where \{...\} denotes the Poisson bracket and $H$ is the Hamiltonian. The Hamiltonian for a particle in an uncoupled linear lattice is given by
%
\begin{equation}
    H = \frac{1}{2}p_x^2 + \frac{1}{2}p_y^2 + \frac{1}{2}k_{x}x^2 + \frac{1}{2}k_{y}y^2 + \frac{q}{m\gamma^3\beta^2c^2}\Phi
\end{equation}
%
where the space charge potential $\Phi$ is determined from the Poisson equation:
%
\begin{equation} \label{eq:Poisson}
    \nabla^2 \Phi = -\frac{q}{\epsilon_0}\iint{f}dx'dy'.
\end{equation}
%

Equilibrium solutions to the Vlasov equation in the presence of time-dependent linear forces are discussed in the following section. For now, we take it for granted that there exists one such solution, called the KV distribution, that produces linear space charge forces ($E_x \propto x$, $E_y \propto y$). In \cite{Hofmann1983}, Hofmann et al. analytically studied perturbations of a round ($\varepsilon_x = \varepsilon_y$) KV distribution using the Vlasov equation in one of the simplest time-dependent cases: a FODO lattice with equal horizontal and vertical tunes. The result is shown in Fig.~\ref{fig:stopbands}, which plots the depressed tune as a function of beam intensity. Each thin line represents a different zero-current tune. The thick lines represent regions of instability.
%
\begin{figure}[p]
    \centering
    \includegraphics[width=\textwidth]{Images/chapter1/stopbands_hor.png}
    \caption{Instability stopbands obtained from perturbations of a KV distribution with equal emittances in a FODO lattice. (From \cite{Hofmann1983}).\label{fig:stopbands}}
    \vfill
    \vspace*{2cm}
    \vfill
    \includegraphics[width=\textwidth]{Images/chapter1/envelope_instability.png}
    \caption{Integrated KV envelope equations in a FODO lattice as the depressed KV tune $\nu_x$ is decreased. The zero-current tune is 100\degree. \label{fig:envelope_instability}}
\end{figure}
%

The KV distribution produces a closed set of differential equations for its envelope — the elliptical boundary containing the beam particles — due to its linear space charge forces. The second-order instabilities involve linear forces only, so they appear in the envelope equations. We use a FODO cell with a zero-current tune of $100\degree$ corresponding to the second-to-bottom line on the left-most plot in Fig.~\ref{fig:stopbands}. The initial distribution is first matched to the lattice, then tracked for 500 cells by integrating the KV envelope equations. Fig.~\ref{fig:envelope_instability} shows the horizontal and vertical envelopes as the depressed KV tune is decreased from $90\degree$ to $71\degree$, crossing the stopband. This is known as the envelope instability. It is assumed to be real, but its existence has not yet been experimentally confirmed.

Observation of the higher-order stopbands requires PIC simulation. We choose a zero-current tune of 90\degree; according to Fig.~\ref{fig:stopbands}, a third-order and fourth-order instability should occur at a depressed tune of 45\degree and 30\degree, respectively. Fig.~\ref{fig:coherent_instabilities} shows the simulated evolution in PyORBIT for three different distributions: KV, Waterbag, and Gaussian. (Note that while the simulations in \cite{Hofmann2017Book} used a bunched beam, coasting beams were used in this simulation.)
%
\begin{figure}[!p]
    \begin{subfigure}[b]{0.45\textwidth}
        \includegraphics[width=\textwidth]{Images/chapter1/coherent_instability_fourth_order.png}
        \label{fig:coherent_instabilities_a}
    \end{subfigure}
    \hfill
    \begin{subfigure}[b]{0.45\textwidth}
        \includegraphics[width=\textwidth]{Images/chapter1/coherent_instability_third_order.png}
        \label{fig:f2}
    \end{subfigure}
    \vfill
    \begin{subfigure}[b]{\textwidth}
        \centering
        \includegraphics[width=0.8\textwidth]{Images/chapter1/coherent_instability_emittances.png}
        \label{fig:coherent_instabilities_b}
    \end{subfigure}
    \caption{Simulated Gaussian, Waterbag, and KV distributions in a FODO lattice with a zero-current tune of 90\degree and depressed KV tunes of $\nu_x$ = 45\degree (top left) and 30\degree (top right).}
    \label{fig:coherent_instabilities}
\end{figure}
%
The instabilities violently affect the KV distribution, but their effect is less pronounced in the other distributions. Thus, it is assumed that high-order coherent instabilities, while interesting, are not important in real beams with a large tune spread. 



\section{Self-consistent phase space distributions}\label{sec:Self-consistent phase space distributions}

\subsection{Definition and properties}

Any function constructed from single-particle invariants $\{C_i\}$ is an equilibrium solution to the Vlasov equation:
%
\begin{equation}\label{eq:vlasov_equilibria}
    \frac{d}{ds} f(\{C_i\}) = \sum_{i}{\frac{df}{dC_i}\frac{dC_i}{ds}} = 0.
\end{equation}
%
One example of a single-particle invariant when the focusing is linear and time-dependent is the Courant-Snyder invariant of Eq.~\eqref{eq:CS invariant}. The inclusion of space charge complicates the identification of invariants; the only known equilibrium distributions are those that produce linear space charge forces. We label such distributions as \textit{self-consistent}: a self-consistent distribution produces linear space charge forces, and the linearity of the space charge force is conserved under any linear transformation. 
Self-consistent distributions possess several notable properties. First, the integro-differential system of equations in Eq.~\eqref{eq:Vlasov} is reduced to a system of ordinary differential equations. Second, nonlinear space charge effects are minimized: the emittance is conserved, the maximum space charge tune shift is minimized, and the space charge tune spread is eliminated. Third, higher-order coherent instabilities may be present in self-consistent distributions due to their small tune spread. Fourth, self-consistent distributions give rise to a uniform charge density.



\subsection{Known solutions}

Danilov et al. enumerated a class of self-consistent distributions in $2d$-dimensional phase space \cite{Danilov2003}: 
%
\begin{equation}\label{eq:scdist_general_form}
    f\left({\mathbf{x}, \mathbf{x}'}\right) = 
    g\left({H - H_b}\right)
    \prod_{i=1}^{m}\delta\left({\mathbf{e}_i \cdot \mathbf{x} 
    + \mathbf{e}'_i \cdot \mathbf{x}'}\right),
\end{equation}
%
where $\mathbf{x}$, $\mathbf{x}'$ are the $d$ dimensional vector coordinates and momenta, $g$ is a function of $H$ — a quadratic positively defined function of the phase space coordinates — and $H_b$ — an upper bound on $H$ — $\delta$ is the Dirac delta function, and $\mathbf{e}_i$, $\mathbf{e}'_i$ are vectors of constants. This is referred to as the $\{n,m\}$ case. It was proven that (a) the electric field within any uniformly filled ellipsoid is linear, and (b) any distribution of the form of $\eqref{eq:scdist_general_form}$ that produces linear a linear electric field will do so under any linear transformation. 

In summary, one class of self-consistent distributions are those that generate linear space charge forces and are constructed from single-particle invariants. We now focus on the \{2, 0\} and \{2, 2\} cases. (Qin and Davidson derived a self-consistent distribution for $n = 2$ in \cite{Qin2013} using their recent parameterization of coupled motion, but made no reference to \cite{Danilov2003}. The connection between Danilov's work and Qin and Davidson's work should thus be explored in the future.)



\subsubsection{The KV distribution}

The $\{2, 0\}$ case corresponds to the KV distribution derived by Kapchinskij and Vladimirskij in 1959 \cite{Kapchinskij1959}. The distribution is a function of the Courant-Snyder invariants $J_x$ and $J_y$:
%
\begin{equation}
    f(\mathbf{x}) = \frac{\lambda}{\pi^2 \varepsilon_x\varepsilon_y}
    \delta \left(\frac{J_x}{\varepsilon_x} + \frac{J_y}{\varepsilon_y} -1 \right),
\end{equation}
%
where $\lambda$ is the longitudinal particle density. Particles in the KV distribution are evenly distributed on an ellipsoidal shell in 4D phase space. As shown in Fig.~\ref{fig:distributions_kv}, any 2D projection of the distribution is a uniform density ellipse. Of particular importance is the $x$-$y$ projection, which remains upright and uniform density under any uncoupled transformation. The electric field within such an ellipse is
%
\begin{equation}  \label{eq:field_in_upright_ellipse}
    \mathbf{E}(x, y) =
    \frac{\lambda}{\pi\epsilon_0}
    \left[ 
        \frac{x}{c_x\left(c_x+c_y\right)} \hat{x}
        + \frac{y}{c_y\left(c_x+c_y\right)} \hat{y}
    \right],
\end{equation}
%
where $c_x$ and $c_y$ are the horizontal and vertical semi-axes and $\epsilon_0$ is the permittivity of free space. Since the space charge force is linear and uncoupled, $J_{x,y}$ remains invariant for every particle and the emittances $\varepsilon_{x,y}$ are conserved. The KV distribution does not exist in 3D [Ref: Sacherer].

As mentioned in the previous section, the preservation of the linearity of the space charge force leads to a self-consistent set of differential equations for the evolution of the beam envelope. The KV envelope equations read:
%
\begin{align} \label{eq:KV_envelope}
    \tilde{x}'' + k_{0x}(s)\tilde{x} - \frac{\varepsilon_x^2}{\tilde{x}^3} - \frac{Q}{2\left(\tilde{x} + \tilde{y}\right)} &= 0, \\
    \tilde{y}'' + k_{0y}(s)\tilde{y} - \frac{\varepsilon_y^2}{\tilde{y}^3} - \frac{Q}{2\left(\tilde{x} + \tilde{y}\right)} &= 0. \nonumber
\end{align}
%
The RMS widths of the beam $\tilde{x} = \sqrt{\langle{{x^2}}}$ and $\tilde{y} = \sqrt{\langle{{y^2}}}$ are used instead of the true widths $c_x$ and $c_y$. They are simply related by a factor of two for a uniform density ellipse. The perveance $Q$ is a dimensionless measure of space charge strength:
%
\begin{equation}\label{eq:perveance}
    Q = \frac{2\lambda r_0}{\beta^2\gamma^3},
\end{equation}
%
where $r_0 = e^2 / 4\pi\epsilon_0mc^2$ is the classical proton radius. Eqs.~\eqref{eq:KV_envelope} provide many insights into dynamical beam behavior and an analytic benchmark for computer simulations.  

A remarkable fact is that Eqs.~\eqref{eq:KV_envelope} are exact for distributions with elliptical symmetry even if the space charge force is nonlinear \cite{Sacherer1968}. They are not closed, however, since the emittances would then depend on time. Thus, the KV envelope equations provide a good approximation to the evolution of more realistic distributions in the limit of elliptical symmetry and small emittance growth \cite{Lund2004}.


\subsubsection{The Danilov distribution}

The focus of this dissertation is on the $\{2, 2\}$ case of Eq.~\eqref{eq:scdist_general_form} which will be referred to as the Danilov distribution:
%
\begin{equation}
    f(\mathbf{x}, \mathbf{x}') \propto 
    \Theta\left({1 - \mathbf{x}^T\bm{\mathbf{\sigma}^{-1}}\mathbf{x}}\right)
    \delta\left({\mathbf{x} - \mathbf{D}\mathbf{x}'}\right)
\end{equation}
%
with 
%
\begin{equation}
    \bm{\sigma} = 
    4
    \begin{bmatrix}
        \langle{xx}\rangle & \langle{xy}\rangle \\
        \langle{xy}\rangle & \langle{yy}\rangle
    \end{bmatrix}
\end{equation}
%
and $\mathbf{D}$ a $2 \times 2$ matrix. Similar to the KV distribution, any 2D projection of the Danilov distribution is a uniform density ellipse; however, the ellipses in the cross-plane projections ($x$-$y$, $x$-$y'$, $y$-$x'$, $x'$-$y'$) are not necessarily upright and may collapse to lines. For example, the projections are shown in Fig.~\ref{fig:distributions_danilov} for $\mathbf{D}_{11}=\mathbf{D}_{22}=0$ and $\mathbf{D}_{12} = -\mathbf{D}_{21}=1$, which corresponds to a rigidly rotating disk. The electric field in a uniform density ellipse with semi-axes $c_{x,y}$ tilted at an angle $\phi$ in the $x$-$y$ plane is:
%
\begin{equation}
\begin{aligned}
    E_x &\propto 
    \left({\frac{\cos^2\phi}{c_x} + \frac{\sin^2\phi}{c_y}}\right) \frac{x}{c_x + c_y}
    +
    \sin\phi\cos\phi \left({\frac{1}{c_y} - \frac{1}{c_x}}\right) \frac{y}{c_x + c_y}, \\
    E_y &\propto 
    \left({\frac{\cos^2\phi}{c_y} + \frac{\sin^2\phi}{c_x}}\right) \frac{y}{c_x + c_y}
    +
    \sin\phi\cos\phi \left({\frac{1}{c_y} - \frac{1}{c_x} }\right) \frac{x}{c_x + c_y}.
\end{aligned}
\end{equation}
%
The field is linear in $x$ and $y$, as required. And it is worth repeating: the linearity of the electric field is maintained under any linear transformation. The key difference from the KV distribution is that space charge linearly couples the horizontal and vertical motion of individual particles. The Courant-Snyder invariants $J_{x,y}$ are therefore replaced by the more general invariants $J_{1, 2}$, which are conserved even with the inclusion of space charge. 

Due to the cross-plane correlation in the Danilov distribution, the 4D emittance is not the simple product of the horizontal and vertical emittances $\varepsilon_x$ and $\varepsilon_y$, which we refer to as the \textit{apparent} emittances from now on. Instead, the 4D emittance is the product of the \textit{intrinsic} emittances $\varepsilon_1$ and $\varepsilon_2$:
%
\begin{equation} \label{eq:mode_emittances1}
    \varepsilon_{4D} = \left|{\bm{\Sigma}}\right|^{1/2} = \varepsilon_1\varepsilon_2.
\end{equation}
%
The intrinsic emittances are found by a symplectic diagonalization of $\bm{\Sigma}$ and are nicely expressed as \cite{Xiao2013}
%
\begin{equation}
    \varepsilon_{1, 2} = \frac{1}{2}\sqrt{
      -tr\left[(\bm{\Sigma} \mathbf{U})^2\right] \pm \sqrt{tr^2\left[(\bm{\Sigma} \mathbf{U})^2\right] - 16|{\bm{\Sigma}}|},
    }
\end{equation}
%
where $\mathbf{U}$ is the unit symplectic matrix:
%
\begin{equation}
    \mathbf{U} = 
    \begin{bmatrix}
        0 & 1 & 0 & 0 \\
        -1 & 0 & 0 & 0 \\
        0 & 0 & 0 & 1 \\
        0 & 0 & -1 & 0
    \end{bmatrix}.
\end{equation}
%
The intrinsic emittances are constants of the motion for any linear focusing system. Their product is less than or equal to the product of the apparent emittances \cite{Buon1993}. The delta functions in the Danilov distribution function cause the 4D emittance, and therefore one of the intrinsic emittances, to be zero. The following relationship holds:
%
\begin{equation} \label{eq:mode_emittances2}
    \varepsilon_1 = \varepsilon_x + \varepsilon_y, \quad
    \varepsilon_2 = 0
\end{equation}
%
or vice versa. Discussion of the Danilov envelope equations is reserved for chapter \ref{chap-2}.




\section{Physically realizing a self-consistent distribution}\label{sec:Physically realizing a self-consistent distribution}


\subsection{Motivation}

The following points motivate the physical realization of a self-consistent distribution.

\begin{enumerate}
\item 
The properties listed in section~\ref{sec:Self-consistent phase space distributions} — conservation of emittance, reduced space charge tune shift, and reduced space charge tune spread — have the potential to increase the maximum intensity in a ring.

\item
It is an interesting challenge to bring a real distribution closer to a self-consistent analytical model which is generally taken to be unrealistic. 

\item
Beams with a uniform charge density are ideal for fixed-target applications such as spallation neutron production. SNS targets are complex entities that cost \$$1.5 \times 10^6$ to replace, and considerable research and development goes into reducing the peak density on the target. This issue will become even more important if future machines are built on the intensity horizon with similar targets.

\item
There has been recent interest in generating circular modes, where a circular mode is a beam with small 4D emittance. Such a beam can be transformed to a round state ($\varepsilon_x = \varepsilon_y$) or a flat state ($\varepsilon_x \ll \varepsilon_y$) using coupled linear optics that preserve $\varepsilon_{1,2}$. In \cite{Burov2002}, several potential applications of circular modes are listed. Consider first a round-flat transformation. The fractional increase in beam luminosity — a figure of merit in colliders [\ref{}] — in this case is
%
\begin{equation}
    C = \sqrt{\frac{\varepsilon_x\varepsilon_y}{\varepsilon_1\varepsilon_2}},
\end{equation}
%
which approaches $\infty$ as $\varepsilon_1\varepsilon_2 \rightarrow 0$. Flat beams may also increase the possible beam intensity in a ring by suppressing incoherent space charge resonances in one dimension, freeing large areas of tune space. The possible uses of flat beams in the Large Hadron Collider (LHC) are suggested by Burov in \cite{Burov2013}. Alternatively, round beams may be helpful for the suppression of beam-beam effects at collider interaction points \cite{Danilo1996}. Circular modes may also find use in relativistic electron cooling \cite{Burov2000}, low-energy hadron cooling \cite{Derbenev2000}, muon ionization cooling, and radiation generation \cite{Corlett2001}. The connection between the Danilov distribution and circular modes is straightforward: the Danilov distribution is a circular mode with uniform charge density. 

\end{enumerate}

\subsection{Previous experimental work}

Luiten et al. proposed a method to create a \{3, 3\} distribution (a 3D uniform density ellipsoid) of electrons \cite{Luiten2004}. They observed that since a uniform density oblate spheroid ($(x/A)^2 + (y/B)^2 + (z/C)^2 = 1$ with $A = B > C$) will collapse under its own gravity into a flat disk \cite{Lin1965}, a flat transverse disk of electrons will longitudinally expand into a uniform density ellipsoid. This ``pancake" distribution can be created using ultrashort pulsed-laser photoemission with an appropriate radial intensity profile. Musumeci et al. experimentally demonstrated this method in \cite{Musumeci2008}. Their measurements are shown in Fig.~\ref{fig:Musumeci}.
%
\begin{figure}[!p]
    \centering
    \begin{subfigure}{0.75\textwidth}
        \includegraphics[width=\textwidth]{Images/chapter1/Musumeci_fig2.png}
        \label{fig:Musumeci_a}
        \caption{}
    \end{subfigure}
    \vfill
    \vspace*{1.5cm}
    \vfill
    \begin{subfigure}{0.75\textwidth}
        \centering
        \includegraphics[width=\textwidth]{Images/chapter1/Musumeci_fig3.png}
        \label{fig:Musumeci_b}
        \caption{Lower image in (a), displayed with a different color map, and 1D projections of the image.}
    \end{subfigure}
    \caption{Measured electron beam images in the  $x$-$z$ plane from \cite{Musumeci2008}.}
    \label{fig:Musumeci}
\end{figure}
%

Unfortunately, these methods do not apply to high-intensity proton rings. Distributions in these machines are built up over many turns and can often be approximated as 2D coasting beams. The rest of this section describes a method to generate a Danilov distribution in a high-intensity proton ring using the concept of phase space painting, as well as the implementation of the method in the SNS.


\subsection{Phase space painting}

Accelerators are often broken into stages; a common pattern is the injection of a beam from a linac into a ring followed by eventual extraction to a different section of the machine. Injection and extraction are accomplished using kicker magnets — dipole magnets with fast rise times.

In multi-turn injection, multiple beam pulses are injected into the same stable region of longitudinal phase space in the ring. This process is limited by Liouville's theorem in the sense that the phase space density in the ring cannot be increased. In charge-exchange injection, an ion beam from the linac is stripped of its electrons upon entering the ring, leaving only protons. This produces a higher phase space density because charge-exchange is a non-Liouvillian process. While multi-turn injection is constrained to tens of turns, charge-exchange injection is performed over hundreds of turns \cite{Bracco2017}.

Phase space painting — or simply ``painting" — is the time-dependent variation of the transverse distance and angle between the injected beam and the circulating beam; as such, it allows time-dependent control over the phase space distribution in the ring. Painting is a vital tool for the mitigation of space charge effects in high-intensity rings. The free parameters are the painting path — the path of the injection point in phase space — and the speed at which this path is traversed. After discussing the two most popular painting methods, we will introduce a new method called elliptical painting that theoretically produces a Danilov distribution.


\subsection{Square root painting methods}

Although particles move along elliptical trajectories in $x$-$x'$ and $y$-$y'$ phase space in the linear approximation, they explore a rectangular region in the $x$-$y$ plane due to differences in the horizontal and vertical tunes. Square root painting methods make the best of this situation by theoretically generating uniform density ellipses in $x$-$x'$ and $y$-$y'$ phase space. 

\subsubsection{Correlated painting}

Let $x$, $x'$, $y$, and $y'$ be the coordinates of the injected beam in the phase space of the circulating beam, and $t$ be a time variable normalized to the range [0, 1]. In its simplest form, correlated painting proceeds as
%
\begin{equation}
\begin{aligned}
    {x}(t) &= {x}_{max}\sqrt{t}, \\
    {y}(t) &= {y}_{max}\sqrt{t}.
\end{aligned}
\end{equation}
%
In the linear approximation and without space charge, correlated painting generates uniform density ellipses in the $x$-$x'$ and $y$-$y'$ planes and a rectangular distribution in the $x$-$y$ plane. This is illustrated on the left side of Fig.~\ref{fig:painting_graphic}.
%
\begin{figure}[!p]
    \centering
    \includegraphics[width=\textwidth]{Images/chapter1/painting_graphic.png}
    \caption{Illustration of correlated painting (left), anti-correlated painting (center), and elliptical painting (right) in the linear approximation without space charge.}
    \label{fig:painting_graphic}
\end{figure}
%
But note that the 1D projection of a uniform density ellipse is a parabola, but the 1D projection of a uniform density rectangle is flat-topped. So the density in the $x$-$y$ plane is non-uniform, leading to nonlinear space charge forces.

A primary concern in the SNS is the size and peak density of the $x$-$y$ distribution on the spallation target \cite{Riemer2010}. A modified correlated painting scheme is employed to this end:
%
\begin{equation}
\begin{aligned}
    {x}(t) &= x_0 + x_{max}\sqrt{t}, \\
    {y}(t) &= y_0 + y_{max}\sqrt{t}. \\
\end{aligned}
\end{equation}
%
Initially, a donut is created in phase space. Space charge and other nonlinear forces eventually cause the distribution to fill in its hollow center, reducing the peak density.


\subsubsection{Anti-correlated painting}

Anti-correlated painting is equivalent to correlated painting reversed in one of the planes:
%
\begin{equation}
\begin{aligned}
    {x}(t) &= x_{max}\sqrt{t}, \\
    {y}(t) &= y_{max}\sqrt{1 - t}. \\
\end{aligned}
\end{equation}
%
Initially, the $x$-$x'$ distribution is a point while the $y$-$y'$ distribution is a donut. The painting path in this method follows the line $J_x + J_y = constant$, which is the condition of particles in the KV distribution. Thus, in the linear approximation without space charge, the final distribution is a KV distribution. This is illustrated in Fig.~\ref{fig:painting_graphic}. However, the space charge force is nonlinear throughout injection and the KV structure will not be maintained \cite{Crosbie1996}. How to overcome this limitation is an open question. Nonetheless, anti-correlated painting has benefits over correlated painting in some cases \cite{Hotchi2020}.


\subsection{Elliptical painting}

Refer back to Eq.~\eqref{eq:eigvec_coords} in which the single-particle motion is written as the sum of two modes. In the elliptical painting method, the injection point $\mathbf{x} = (x, x', y, y')$ is scaled along one of the eigenvectors:
%
\begin{equation}\label{eq:elliptical_painting}
    \mathbf{x}(t) =  
    Re \left\{ \sqrt{2 J_l} \, \mathbf{v}_l \, e^{-i\psi_l} \right\} \sqrt{t},
\end{equation}
%
with $l = 1,2$. The first injected pulse does not move since it is injected onto the closed orbit. The second pulse traces a small elliptical path in every 2D projection of the phase space on a turn-by-turn basis. The third pulse traces a slightly larger elliptical path enclosing the second pulse, and so on. The square root time-dependence ensures that the beam is a uniform density ellipse in every 2D projection of the 4D phase space at every point during injection. Thus, in the linear approximation, a Danilov distribution is maintained at all times, even with space charge. This is illustrated on the right side of Fig.~\ref{fig:painting_graphic}. Of course, the method is limited even in the linear approximation due to the finite emittance of the beam from the linac.

Elliptical painting can be carried out in any ring. If the ring is uncoupled, the two elliptical modes reduce to planar modes and injection into one of the modes results in a flat beam. Coupled optics change the shape of the matched beam at the injection point and produce a non-flat beam. Alternatively, the horizontal and vertical tunes can be equated, in which case the transfer matrix has degenerate eigenvalues and any linear combination of eigenvectors is itself an eigenvector. 


\subsection{Implementation of elliptical painting in the Spallation Neutron Source}

Elliptical painting requires time-dependent control of the transverse ring orbit position and slope in both planes at the injection point. The SNS is highly-optimized for correlated painting, not elliptical painting, but elliptical painting is possible in principle. A software application to perform the painting method has recently been developed by SNS physicists; before describing the application, a brief description of the SNS is warranted. 

\subsubsection{Description of the Spallation Neutron Source}

The SNS is a neutron scattering facility. Sixty times per second, a microsecond-long proton beam collides with a liquid mercury target at 1 GeV kinetic energy, producing neutrons by the process of spallation \cite{Russell1990}. The original beam is a continuous wave of H$^-$ ions which is then bunched in a radio-frequency quadrupole (RFQ) and chopped into microsecond-long minipulses. Each minipulse is accelerated to 1 GeV through a normal-conducting and superconducting linac, then transported to the injection region through the high-energy beam transport (HEBT). The electrons are then stripped using a carbon foil, and the remaining protons continue their journey in the ring. One thousand minipulses — $1.5 \times 10^{14}$ protons — are accumulated over $10^{-3}$ seconds before the beam is extracted and guided through the ring-beam transport line (RTBT) to the target. A comprehensive description of the SNS is given in \cite{Henderson2014}. Fig.~\ref{fig:SNS} shows an overview of the machine.
%
\begin{figure}[!p]
    \centering
    \includegraphics[angle=-90, width=0.5\textwidth]{Images/chapter1/SNS.png}
    \caption{Overview of the Spallation Neutron Source.}
    \label{fig:SNS}
\end{figure}
%

Fig.~\ref{fig:SNS_injection_region} zooms in on the injection region.
%
\begin{figure}[!p]
    \centering
    \begin{subfigure}{\textwidth}
        \includegraphics[width=\textwidth]{Images/chapter1/SNS_injection_region1.png}
        \label{fig:SNS_injection_region_a}
        \caption{}
    \end{subfigure}
    \vfill
    \vspace*{1.5cm}
    \vfill
    \begin{subfigure}{\textwidth}
        \centering
        \includegraphics[width=\textwidth]{Images/chapter1/SNS_injection_region_2b.png}
        \caption{}
        \label{fig:SNS_injection_region_b}
    \end{subfigure}
    \caption{SNS injection region. (a) Overhead view of H$^-$ beam trajectory. (b) Schematic layout of the horizontal plane of the injection region: red = chicane dipoles, blue = quadrupoles, green = horizontal kickers, yellow = vertical kickers. (From \cite{Henderson2014}.)}
    \label{fig:SNS_injection_region}
\end{figure}
%
Four dipole magnets align the horizontal orbit with the beam from the linac at the foil. The injected beam trajectory is held fixed so that any remaining H$^0$ or H$^-$ particles can be reliably guided to a dump. Eight kicker magnets — four per plane — are available for time-dependent control of the position and slope of the orbit at the injection point. 


\subsubsection{Ring Injection Control application}

Each injection kicker magnet is given a waveform that scales the kicker voltage during injection. Once the initial and final voltages are known, they can be connected with a square root waveform to satisfy Eq.~\eqref{eq:elliptical_painting}. The time between the initial and final voltages (painting time) controls the final beam intensity. 

To control the position and angle of the orbit at the injection point, it is first necessary to measure the position and angle of the orbit at the injection point. The position and angle can be measured indirectly as follows. A single minipulse is injected and stored in the ring, and its turn-by-turn mean transverse position is measured at one position using a beam-position-monitor (BPM). This is repeated for several minipulses and the average is taken. In the linear approximation, the mean position performs the pseudo-harmonic oscillations of Eq.~\eqref{eq:Hill_solution}; however, energy spread in the minipulse leads to decoherence — the mean position goes to zero. A Gaussian-damped sine wave is an accurate model of this process [\ref{}]:
%
\begin{equation}\label{eq:damped_sinusoid}
    x(t) = A_0 + A e^{kt^2} \cos{\left(\mu + \mu_0\right)},
\end{equation}
%
where $t$ is the turn number. The parameter $A$ gives the betatron amplitude, $\mu / 2\pi$ gives the fractional tune, and $\mu_0$ gives the particle phase at the BPM. The phase space coordinates recovered by combining these parameters with the linear ring model:
%
\begin{equation}
\begin{aligned}
    x_{bpm} &= A \cos\mu_0 \\ 
    x'_{bpm} &= -A\left({\sin\mu_0 + \frac{\alpha}{\beta}\cos\mu_0}\right).
\end{aligned}
\end{equation}
%
The coordinates are then transported to the injection point using the model transfer matrix. Repeating this for each BPM gives an estimated mean and standard deviation of the phase space coordinates at the injection point. Examples of measured individual and averaged BPM signals in the SNS ring are shown in Fig.~\ref{fig:bpm_avg} along with the damped-sinusoid fit. Additionally, a simulated minipulse in the (linearized) SNS ring is shown in Fig.~\ref{fig:minipulse}.
%
\begin{figure}[!p]
    \centering
    \includegraphics[width=\textwidth]{Images/chapter1/bpm_avg.png}
    \caption{Measured turn-by-turn BPM signal in the SNS ring — averaged over 50 pulses and fit with Eq.~\eqref{eq:damped_sinusoid}.}
    \label{fig:bpm_avg}
\end{figure}
%
%
\begin{figure}
    \centering
    \includegraphics[width=\textwidth]{Images/chapter1/minipulse_chromaticity.png}
    \caption{Simulated minipulse in the (linearized) SNS ring. The $x$-$x'$ distribution is plotted at the injection point along with the Courant-Snyder ellipse.}
    \label{fig:minipulse}
\end{figure}
%

The next issue is how to control phase space coordinates at the injection point. Each kicker magnet is calibrated by applying a voltage difference to the magnet and measuring the orbit response using the ring BPMs; the angular kick associated with the magnet is varied until the model orbit agrees with the measured orbit. It was found that slight quadrupole corrections are necessary for this to occur. The standard deviation of the measured phase space coordinates is small after this calibration. One can then ask the model for a change in coordinates, update the kickers accordingly, and measure the new coordinates, iterating if necessary. This is currently done manually, and kicker power supplies need to be visually checked to make sure they are not beyond physical limits. Once this setup is complete, the kicker voltages are saved to a file and fed to a different script which scales sets the kicker waveforms. These steps were implemented as part of the Ring Injection Control (RIC) application in the OpenXAL framework \cite{Milas2021}. 

SNS injection kicker magnets have limited strengths and are unipolar, which limits the minimum distance between the ring orbit and the foil as well as the maximum angle at the injection point. In fact, the closed orbit cannot reach the foil at production energy (1 GeV). As will be discussed later, the beam energy can be lowered to increase the effective kicker strength; however, this is a significant task for SNS operators due to issues related to the SNS timing system. Initial attempts to lower the energy to 0.6 GeV were unsuccessful, but an energy of 0.8 GeV was recently achieved. 



\section{Goals of this dissertation}\label{sec:Goals of this dissertation}.

The goal of this dissertation is to contribute to efforts to generate a Danilov distribution in the SNS ring. An additional goal is to improve the current understanding of the dynamics of the Danilov distribution with space charge.

In chapter \ref{chap-2}, envelope equations describing the linear transport of the Danilov distribution are studied. In particular, insight is gained by computing the matched beam envelope in the presence of space charge and linear external coupling. In chapter \ref{chap-3}, particle-in-cell (PIC) simulations of elliptical painting in the SNS are carried out, building on previous work. In chapter \ref{chap-4}, methods are proposed to measure the similarity between a painted distribution in the SNS ring and a Danilov distribution. The methods are simulated and implemented in the SNS. In chapter \ref{chap-5}, elliptical painting is carried out in the SNS for the first time; measurements compared with simulation. Finally, implications and extensions of this work are discussed in chapter \ref{chap-6}.

 \chapter{Beam envelope equations} \label{chap-2}

This chapter investigates the dynamics of the Danilov distribution using the envelope model. In addition to satisfying theoretical interest, the study of the envelope equations has the potential to elucidate previous simulation results and guide future experiments. A brief outline: First, the envelope equations for the Danilov distribution are presented. Second, a method to find the matched beam envelope with space charge is developed. The method is demonstrated in several simple focusing systems, and the properties of the solutions are examined. Third, the matched beam envelope in the linearized SNS ring is calculated and the practical importance of the calculation is discussed, namely that knowledge of the matched solution with space charge is critical to understanding the conditions under which elliptical painting will produce a Danilov distribution.\footnote{The majority of this chapter has been published in \cite{Hoover2021}.}


\section{The Danilov envelope equations}

We seek a self-consistent set of differential equations for the elliptical boundary containing the beam particles, similar to Eq.~\eqref{eq:KV_envelope}. Although Chernin's equations satisfy this requirement \cite{Chernin1988}, we adopt the equations derived in \cite{Danilov2003}. There, the coordinates of the ellipse in real space are parameterized as $\mathbf{x} = \mathbf{W}\mathbf{c}$ where $\mathbf{x} = (x, y)^T$, $\mathbf{W}$ is the $2 \times 2$ envelope matrix, $\mathbf{c} = (\cos\psi, \, \sin\psi)^T$, and $\psi$ is a free parameter running from $0$ to $2\pi$. The envelope matrix evolves according to
%
\begin{equation}\label{eq:danilov_envelope}
    \mathbf{W}'' + \left({\mathbf{K_0} - \mathbf{K}_{sc}}\right)\mathbf{W} + \mathbf{K}_1 \mathbf{W}'= 0,
\end{equation}
%
where $\mathbf{K}_0$, $\mathbf{K}_1$, and $\mathbf{K}_{sc}$ are time-dependent $2 \times 2$ matrices. Linear external focusing is encompassed by $\mathbf{K}_{0, 1}$, and linear space charge defocusing is encompassed by $\mathbf{K}_{sc}$. If the beam ellipse in real space has semi-axes $c_x$ and $c_y$ and is tilted at an angle $\phi$ below the $x$ axis, then
%
\begin{equation}
    \mathbf{K}_{sc} = \frac{2Q}{c_x + c_y} 
        \mathbf{R}(\phi) \begin{bmatrix} 1/c_x & 0 \\ 0 & 1/c_y \end{bmatrix} \mathbf{R}(\phi)^T. 
\end{equation}
%
$\mathbf{R}$ is the rotation matrix and $Q$ is the beam perveance (Eq.~\eqref{eq:perveance}). When Eq.~\eqref{eq:danilov_envelope} is expanded, the evolution of each envelope matrix element resembles that of a single particle in a linear system with a nonlinear driving term proportional to $Q$. Thus, the equations can be easily integrated using an existing tracking code such as PyORBIT \cite{Shishlo2015}. The four elements of $\mathbf{W}$ are represented using two bunch particles, and nodes are added to the accelerator model to perform the nonlinear kicks. 

The integration of Eq.~\eqref{eq:danilov_envelope} reveals that the Danilov distribution tilts in real space even without coupled forces; this is illustrated in Fig.~\ref{fig:fodo_zerosc}. 
%
\begin{figure}[!p]
    \centering
    \includegraphics[width=0.8\textwidth]{Images/chapter2/fodo_zerosc.png}
    \caption{Relationship between the tilt angle in real space and the $x$ and $y$ phase advances for a Danilov distribution in a FODO cell of length $L$. The beam is tracked without space charge by integrating the envelope equations.}
    \label{fig:fodo_zerosc}
\end{figure}
%
The tilt angle is a function of the difference between the horizontal and vertical phase advances, which are calculated from
\begin{equation} \label{eq:phase_advance}
    \mu_x(s) = \int_{0}^{s}
    {\frac{\varepsilon_x(s')}{{\tilde{x}(s')}^2} \, ds'},
\end{equation}
where $\tilde{x}^2 = \langle{x^2}\rangle$ and $x$ and $y$ can be interchanged.



\section{Matched envelope computation}

The distribution function $f$ of a matched beam in a lattice of period length $L$ satisfies $f(s) = f(s + L)$ for all $s$. In practice, matching usually refers only to the second-order moments contained in the beam covariance matrix: $\bm{\Sigma}(s) = \bm{\Sigma}(s + L)$. The two notions are equivalent for the Danilov distribution, for which all higher-order moments vanish.

The problem of computing the matched beam is as follows:
%
\begin{equation}
\begin{aligned}
    & \underset{\bm{\Sigma}}{\text{minimize}}
    & & C(\bm{\Sigma}) = \left\Vert{\mathbf{M} \bm{\Sigma} \mathbf{M}^T - \bm{\Sigma}}\right\Vert^2 \\
    & \text{subject to}
    & & |\bm{\Sigma}| = 0,
\end{aligned}
\end{equation}
%
where $\mathbf{M} = \mathbf{M}(\bm{\Sigma})$ is a linear transfer matrix connecting the initial and final covariance matrix, and $\Vert\dots\Vert$ is the matrix norm. The constraint that the covariance matrix is singular comes from the definition of the Danilov distribution. Additionally, we would like to hold the nonzero intrinsic emittance fixed so that a unique solution is found for a given lattice and beam perveance. An iterative approach is needed since space charge causes $\mathbf{M}$ to depend on $\bm{\Sigma}$ in a potentially complicated way which is unknown before tracking the beam. 


\subsection{Motivation}

The calculation of the matched beam envelope is of practical importance for space charge dominated beams. It is a common first step in lattice design \cite{Lund2006}. The beam current able to be transported through a periodic focusing channel with a given aperture is maximum when the beam is matched \cite{book:Reiser}. The free energy available in a mismatched beam may result in emittance growth and halo formation \cite{book:Reiser}. Calculation of the matched envelope is generally the first step in a stability analysis of the KV envelope equations \cite{Lund2006}. The task has been performed for the KV distribution in both uncoupled and coupled focusing systems \cite{Hofmann1983, Chernin1988, Ryne1995, Lund2006, Anderson2007, Goswami2016}, and it would be beneficial to extend this analysis to the Danilov distribution. Furthermore, the calculation of the matched envelope of the Danilov distribution is of critical importance to the experimental realization of such a distribution: the end-result of the elliptical painting method only approaches a Danilov distribution if the circulating beam is matched.

The purpose of this section is to calculate and describe the properties of the matched envelope of the Danilov distribution in simple focusing systems as space charge is increased. In addition to the ability to calculate the matched envelope in more complicated focusing systems such as the SNS accumulator ring, a better understanding of the beam dynamics under the influence of coupled internal and external forces should fall out of this analysis.

Before presenting our solution, we highlight the space-charge-driven mismatch oscillations that are to be corrected. Fig.~\ref{fig:fodo_mismatch_tbt} shows the turn-by-turn evolution of a beam that is matched to a lattice without space charge and then tracked with space charge. 
%
\begin{figure}[!p]
    \centering
    \includegraphics[width=\textwidth]{Images/chapter2/fodo_mismatch_tbt.png}
    \caption{Turn-by-turn mismatch oscillations of the Danilov distribution at the entrance of an uncoupled FODO lattice. The beam would be matched to the lattice without space charge. In the right column, blue (orange) corresponds to $x$ ($y$).}
    \label{fig:fodo_mismatch_tbt}
\end{figure}
%
There are two frequencies in the mismatch oscillations: a larger frequency near twice the zero-current tune corresponding to the breathing oscillation of the beam sizes, and a smaller frequency corresponding to the emittance exchange from space-charge-driven linear coupling. 


\subsection{Solution}

The problem of computing the matched envelope can be approached in the following way. The effect of the linear beam space charge is to modify the linear focusing strength at every position; we call this modified linear focusing system the \textit{effective lattice}. Generating a beam that is matched to a lattice with space charge is equivalent to generating a beam that is matched to an effective lattice without space charge. The latter task is straightforward using an existing parameterization of coupled motion. The correct effective lattice is unknown a priori, so a search must be performed over the lattice parameters. Fig.~\ref{fig:effective_lattice} illustrates the concept of the effective lattice.
%
\begin{figure}
    \centering
    \includegraphics[width=1.0\textwidth]{Images/chapter2/effective_lattice.png}
    \caption{Illustration of the effective lattice — the net linear focusing after space charge is included. The coefficients $k_{ij}$ are defined by $x'' + k_{11}x + k_{13}y = 0$; $y'' + k_{33}y + k_{31}x = 0$.}
    \label{fig:effective_lattice}
\end{figure}
%

\subsubsection{Zero space charge}

We begin by demonstrating how to compute the matched envelope without space charge. We write Eq.~\eqref{eq:eigvec_coords} again for convenience:
%
\begin{equation}
    \mathbf{Mx} = Re \left\{
        \sqrt{2 J_1} \, \mathbf{v}_1 \, e^{-i(\psi_1 + \mu_1)}
        + \sqrt{2 J_2} \, \mathbf{v}_2 \, e^{-i(\psi_2 + \mu_2)}
    \right\}.
\end{equation}
%
The turn-by-turn trajectory of a particle with a given $J_{1,2}$ forms a closed surface in phase space, and a group of particles distributed uniformly over this surface will appear to be invariant. A matched distribution is a collection of these surfaces with different amplitudes.

We now switch to the collective description of the beam using its covariance matrix. The symplectic normalization matrix $\mathbf{V}$ (from Eq.~\eqref{eq:V_from_eigvecs}) can be used to express the matched beam covariance matrix as
%
\begin{equation}\label{eq:sigma_n}
    \bm{\Sigma}_n = 
    \mathbf{V}
    \begin{bmatrix}
        \varepsilon_1 & 0 & 0 & 0 \\
        0& \varepsilon_1 & 0 & 0 \\
        0 & 0 & \varepsilon_2 & 0 \\
        0 & 0 & 0 & \varepsilon_2
    \end{bmatrix}
    \mathbf{V}^T.
\end{equation}
%
In the uncoupled case, $\mathbf{V}^{-1}$ transforms a tilted ellipse in the $x$-$x'$ plane into a circle. In the coupled case, $\mathbf{V}^{-1}$ transforms a ``tilted" 4D ellipsoid into an ``upright" 4D ellipsoid.

A parameterization of $\mathbf{V}$ was introduced in Fig.~\ref{fig:twiss4D}. The number of parameters can be reduced to six by observing that the Danilov distribution is a function of only one eigenvector. We now set one of the intrinsic emittances to zero in Eq.~\eqref{eq:sigma_n} and display the connection between the parameters and the covariance matrix. The beta functions give the ratios between the beam size and intrinsic emittance:
%
\begin{equation}
    \beta_{lx} = \frac{\langle{x^2}\rangle}{\varepsilon_l}, \quad
    \beta_{ly} = \frac{\langle{y^2}\rangle}{\varepsilon_l}
\end{equation}
%
where again $l = 1$ or $2$. Similarly, the alpha functions give the ratios between the beam divergence and the intrinsic emittance:
%
\begin{equation}
    \alpha_{lx} = -\frac{\langle{xx'}\rangle}{\varepsilon_l}, \quad
    \alpha_{ly} = -\frac{\langle{yy'}\rangle}{\varepsilon_l}
\end{equation}
%
Next, $u$ gives the ratio between the apparent emittance and the intrinsic emittance. When $l = 1$, $u = \varepsilon_y / \varepsilon_l$, or when $l = 2$, $u = \varepsilon_x / \varepsilon_2$.
% %
% \begin{equation} \label{eq:u}
%      u = \frac{\varepsilon_y}{\varepsilon_1},
% \end{equation}
% %
% or when $l = 2$:
% %
% \begin{equation}
%      u = \frac{\varepsilon_x}{\varepsilon_2}.
% \end{equation}
% %
Finally, $\nu_l$, is related to the $x$-$y$ correlation coefficient:
%
\begin{equation}
    \cos\nu_l = \frac{\langle{xy}\rangle}{\sqrt{\langle{x^2}\rangle\langle{y^2}\rangle}}.
\end{equation}
%
The subscript will be dropped from now on since it has no effect. As mentioned in Fig.~\ref{fig:fodo_zerosc}, $\nu$ will vary even without the presence of coupled forces. For example, Fig.~\ref{fig:splittunes_tbt} shows the turn-by-turn $x$-$y$ projection of a beam whose $x$-$x'$ and $y$-$y'$ ellipses are matched to an uncoupled lattice with a tune separation of 0.01, along with the value of $\nu$ at each frame. 
%
\begin{figure}[!p]
    \centering
    \includegraphics[width=0.8\textwidth]{Images/chapter2/splittunes_tbt.png}
    \caption{Turn-by-turn $x$-$y$ projections of a Danilov distribution in a lattice with a tune split of 0.01. The horizontal and vertical phase space projections are matched to the lattice.}
    \label{fig:splittunes_tbt}
\end{figure}
%
In summary, the matched beam is described by a vector of parameters $\mathbf{p}$, where
%
\begin{equation} \label{eq:twiss_params_4D}
    \mathbf{p} = (\alpha_{lx}, \alpha_{ly}, \beta_{lx}, \beta_{ly}, u, \nu)
\end{equation}
%
with $l = 1$ or $2$ depending on which intrinsic emittance is zero.



\subsubsection{Nonzero space charge}

We denote the choice $\varepsilon_2 = 0$ as solution $1$ and $\varepsilon_1 = 0$ as solution 2. Once this emittance is chosen, $\mathbf{p}$ is initialized using the bare lattice parameters. We then perform the following procedure: (1) generate a beam envelope from $\mathbf{p}$, (2) track the beam through one lattice period and compute the cost function, (3) update $\mathbf{p}$, (4) stop if the relative change in $C$ or $|\mathbf{p}|$ is below a given tolerance, otherwise repeat from step 1. A trust-region minimization algorithm \cite{Branch1999} is used to determine the update strategy for $\mathbf{p}$. If necessary, the process can be repeated at multiple steps so that the seed envelope remains close to the matched solution. In one case during our studies, this optimizer failed to converge and it became necessary to use a custom update method; in this method, the beam is tracked for a number of turns and $\mathbf{p}$ is updated to its average over those turns. The method does not need to worry about bounds on the parameters since every update is based on an existing beam. An example of the progress of this method is shown in Fig.~\ref{fig:optimizer_iters}. 

\begin{figure}[!p]
    \centering
    \includegraphics[width=0.8\textwidth]{Images/chapter2/optimizer_iters.png}
    \caption{Turn-by-turn oscillations of the beam parameters after the first few iterations of the matching routine. The custom update method is used. Faint horizontal lines give the average of the oscillations. Blue (orange) corresponds to $x$ ($y$).}
    \label{fig:optimizer_iters}
\end{figure}








\subsection{Method demonstration}

The matching routine was applied to an equally spaced, periodic quadrupole (FODO) lattice. The horizontal focusing strength in this lattice is shown in Fig.~\ref{fig:fodo_lattices}a as a function of $s$. Several variants of the FODO lattice were also considered to include external coupling: in Fig.~\ref{fig:fodo_lattices}b, the focusing and defocusing quadrupoles are rotated by $3\degree$ in opposite directions in the transverse plane, and in Fig.~\ref{fig:fodo_lattices}c solenoid magnets are inserted in the drift spaces between the quadrupoles. This section examines the matched solutions in each lattice as space charge is increased. Previous studies indicate that the KV envelope equations have a unique matched solution for each choice of lattice, beam perveance, and apparent emittances. Although there is no known proof of this conjecture, it seems to be true based on numerical evidence \cite{Lund2006}. Thus, for the Danilov distribution, it was expected that each choice of lattice and perveance will lead to two matched solutions depending on which intrinsic emittance is set to zero. No evidence to the contrary was found.
%
\begin{figure}[!p]
    \centering
    \includegraphics[width=0.8\textwidth]{Images/chapter2/fodo_lattices.png}
    \caption{Horizontal focusing strength as a function of $s$ in a FODO lattice with period length $L$. Quadrupoles have length $L/4$ and are equally spaced. (a) Upright quadrupoles with $80\degree$ phase advance in both planes. (b) The quadrupoles are rotated by $\phi = 3\degree$ in the transverse plane ($QF$ and $QD$ are rotated in opposite directions). (c) Solenoid magnets are inserted between the quadrupoles in (a).}
    \label{fig:fodo_lattices}
\end{figure}


\subsubsection{Uncoupled lattice}

We now apply the matching routine to an uncoupled FODO lattice. First, a note about the matched solution for without space charge. The eigenvectors of the transfer matrix are uncoupled, meaning that $\mathbf{v}_1$ has no component in the $y$-$y'$ plane and $\mathbf{v}_2$ has no component in the $x$-$x'$ plane. A matched beam is formed by generating particles uniformly in phase along either of these eigenvectors; therefore, the matched beam is flat ($\varepsilon_x = 0$ or $\varepsilon_y = 0$). An exception occurs when the transfer matrix has degenerate eigenvalues, i.e., when the horizontal and vertical tunes are equal. In this case, any linear combination of eigenvectors forms another eigenvector. Thus, without space charge, there are an infinite number of matched solutions in a lattice with equal tunes. The free parameters from Eq.~\eqref{eq:twiss_params_4D} are $u$ and $\nu$.

We now include the self-force of the beam in the matching routine. Fig.~\ref{fig:matched_vs_sc_fodo} shows the matched beam sizes, apparent emittances, and $\nu$ parameter within the lattice for a range of linearly increasing space charge strengths.\footnote{For the zero space charge solution, we chose $u = 0.5$ and $\nu = \pi/2$.}
%
\begin{figure}[!p]
    \begin{subfigure}{1.0\textwidth}
        \includegraphics[width=\textwidth]{Images/chapter2/matched_vs_sc_fodo_mode1.png}
        \caption{Solution 1}
        \label{fig:matched_vs_sc_fodo_a}
    \end{subfigure}
    \vfill
    % \vspace*{1.0cm}
    \vfill
    \begin{subfigure}{1.0\textwidth}
        \centering
        \includegraphics[width=\textwidth]{Images/chapter2/matched_vs_sc_fodo_mode2.png}
        \caption{Solution 2}
        \label{fig:matched_vs_sc_fodo_b}
    \end{subfigure}
    \caption{Matched envelope of the Danilov distribution in an uncoupled FODO lattice as space charge is increased. Left: phase space projections at the lattice entrance. Right: beam parameters within the lattice. Solid lines are for $x$ and dashed lines are for $y$ in the top two plots.}
    \label{fig:matched_vs_sc_fodo}
\end{figure}
%
It also shows the phase space projections at the lattice entrance. The following properties of the matched solutions are worth noting. First, except for the oscillatory apparent emittances, the beam evolution within the lattice when space charge is nonzero is very similar to the case of zero space charge. Second, two solutions are found which differ in the sign of their angular momentum. This is seen in the opposite signs of the slopes defining the linear relationships between the position in one plane and the slope in the other; it is a consequence of the opposite directions of rotation of the two eigenvectors. The third property to note is how the solutions scale with increased space charge: the average width and height of the matched beam within the lattice grow approximately linearly, and the variation in the difference between the horizontal and vertical phase advances is suppressed — hence the decreased oscillation of the $\nu$ parameter. 

The same analysis can be performed when the horizontal and vertical tunes are split. We chose to increase the horizontal phase advance and decrease the vertical phase advance, both by $5\degree$. The results are displayed in Fig.~\ref{fig:matched_vs_sc_fodo_split}.\footnote{$\nu$ is undefined when the beam is flat, but we chose to draw a flat line at $\nu = \pi/2$.}
%
\begin{figure}[!p]
    \begin{subfigure}{1.0\textwidth}
        \includegraphics[width=\textwidth]{Images/chapter2/matched_vs_sc_fodo_split_mode1.png}
        \caption{Solution 1}
        \label{fig:matched_vs_sc_fodo_split_a}
    \end{subfigure}
    \vfill
    % \vspace*{1.0cm}
    \vfill
    \begin{subfigure}{1.0\textwidth}
        \centering
        \includegraphics[width=\textwidth]{Images/chapter2/matched_vs_sc_fodo_split_mode2.png}
        \caption{Solution 2}
        \label{fig:matched_vs_sc_fodo_split_b}
    \end{subfigure}
    \caption{Matched envelope of the Danilov distribution in an uncoupled FODO lattice with unequal tunes as space charge is increased. Left: phase space projections at the lattice entrance. Right: beam parameters within the lattice. Solid lines are for $x$ and dashed lines are for $y$ in the top two plots.}
    \label{fig:matched_vs_sc_fodo_split}
\end{figure}
%
The beam dynamics with space charge are very similar to the previous case of equal tunes, and the two solutions are again related by opposite signs of their angular momentum. The most notable effect of space charge on the matched beam is to significantly split the apparent emittances. This is expected since space charge now needs to decrease the horizontal and vertical tunes by different amounts such that they are equal in the end. When space charge is weak, the apparent emittances are maximally split; as space charge is increased, the emittances move closer together. We also note that the horizontal emittance in both solutions is always greater than the vertical emittance when space charge is turned on, which follows from the fact that the zero-current tune is larger in the horizontal plane than in the vertical plane, so space charge must provide more defocusing in the vertical plane.





\subsubsection{Coupled lattice}

The same information as in Fig.~\ref{fig:matched_vs_sc_fodo} and Fig.~\ref{fig:matched_vs_sc_fodo_split} is plotted in Fig.~\ref{fig:matched_vs_sc_fodo_skew} for the skew quadrupole lattice in Fig.~\ref{fig:fodo_lattices}b.
%
\begin{figure}[!p]
    \begin{subfigure}{1.0\textwidth}
        \includegraphics[width=\textwidth]{Images/chapter2/matched_vs_sc_fodo_skew_mode1.png}
        \caption{Solution 1}
        \label{fig:matched_vs_sc_fodo_skew_a}
    \end{subfigure}
    \vfill
    % \vspace*{1.0cm}
    \vfill
    \begin{subfigure}{1.0\textwidth}
        \centering
        \includegraphics[width=\textwidth]{Images/chapter2/matched_vs_sc_fodo_skew_mode2.png}
        \caption{Solution 2}
        \label{fig:matched_vs_sc_fodo_skew_b}
    \end{subfigure}
    \caption{Matched envelope of the Danilov distribution in a coupled FODO lattice as space  charge is increased. The lattice is coupled due to skew quadrupoles. Left: phase space projections at the lattice entrance. Right: beam parameters within the lattice. Solid lines are for $x$ and dashed lines are for $y$ in the top two plots.}
    \label{fig:matched_vs_sc_fodo_skew}
\end{figure}
%
The locations of the skew quadrupoles are evident from the small arcs in the emittance curves. Without space charge, the matched beam at the center of the quadrupoles projects to a diagonal line in real space ($\nu = 0\degree$ or $180\degree$) with zero angular momentum, and the two solutions differ in the sign of the tilt angle of this line. The inclusion of space charge pulls $\nu$ away from these extreme values, resulting in a nonzero beam area. The cross-plane correlations between the positions and slopes also become nonzero, again revealing the opposite directions of the angular momentum between the two solutions. 
The relationship between the two solutions is now more complicated than in the previous cases. The presence of space charge leads to two solutions with the same tilt angle in the $x$-$y$ plane, as opposed to opposite tilt angles without space charge. This abrupt change in the matched beam orientation in solution 1 causes the optimizer to struggle for low space charge, often terminating due to a lack of progress. Fig.~\ref{fig:optimizer_comparison_a} shows the final cost as a function of the beam perveance, and Fig.~\ref{fig:optimizer_comparison_b} shows the turn-by-turn oscillations of the $\nu$ parameter for a subset of these cases. 
%
\begin{figure}[!p]
    \centering
    \begin{subfigure}{0.8\textwidth}
        \includegraphics[width=\textwidth]{Images/chapter2/optimizer_comparison_costfunc.png}
        \caption{}
        \label{fig:optimizer_comparison_a}
    \end{subfigure}
    \vfill
    \vspace*{1cm}
    \vfill
    \begin{subfigure}{1.0\textwidth}
        \includegraphics[width=\textwidth]{Images/chapter2/optimizer_comparison_tbt.png}
        \caption{}
        \label{fig:optimizer_comparison_b}
    \end{subfigure}
    \caption{Performance of the matching algorithm in a skew quadrupole lattice, corresponding to solution 1 in Fig.~\ref{fig:matched_vs_sc_fodo_skew_a}. (a) Final value of the cost function as a function of the beam perveance. (b) Turn-by-turn oscillations of the $\nu$ parameter after running each algorithm.}
    \label{fig:optimizer_comparison}
\end{figure}
%
The matching routine is not run when $Q = 0$ since the beam is already matched to the bare lattice; this corresponds to the bottom line in both panels of Fig.~\ref{fig:optimizer_comparison_b} at $\nu = 0$. The final cost is therefore the same for the two algorithms at this point. For small but nonzero perveance, the optimizer converges to a beam with $\nu \approx 0$ which exhibits very small mismatch oscillations (yellow region). The oscillations become more severe as the perveance is increased (red region), which corresponds to lines starting at $\nu \approx 25\degree$ in Fig.~\ref{fig:optimizer_comparison_b}. An exact match is eventually found (green region) when the perveance is sufficiently large with $\nu \approx 150\degree$. The averaging method, on the other hand, finds the exact match in nearly all cases. (Note that the circles and crosses on the far right of Fig.~\ref{fig:optimizer_comparison_b} represent the same beam; the algorithms have just terminated at different final costs.) This discussion simply illustrates that some care must be taken for certain values of the beam perveance when skew quadrupoles are present.

As a final demonstration of the method, coupling was included by the insertion of solenoid magnets in Fig.~\ref{fig:fodo_lattices}c. The results are shown in Fig.~\ref{fig:matched_vs_sc_fodo_sol}.
%
\begin{figure}[!p]
    \begin{subfigure}{1.0\textwidth}
        \includegraphics[width=\textwidth]{Images/chapter2/matched_vs_sc_fodo_sol_mode1.png}
        \caption{Solution 1}
        \label{fig:matched_vs_sc_fodo_sol_a}
    \end{subfigure}
    \vfill
    % \vspace*{1.0cm}
    \vfill
    \begin{subfigure}{1.0\textwidth}
        \centering
        \includegraphics[width=\textwidth]{Images/chapter2/matched_vs_sc_fodo_sol_mode2.png}
        \caption{Solution 2}
        \label{fig:matched_vs_sc_fodo_sol_b}
    \end{subfigure}
    \caption{Matched envelope of the Danilov distribution in a coupled FODO lattice as space  charge is increased. The lattice is coupled due to solenoid magnets inserted between the quadrupoles. Left: phase space projections at the lattice entrance. Right: beam parameters within the lattice. Solid lines are for $x$ and dashed lines are for $y$ in the top two plots.}
    \label{fig:matched_vs_sc_fodo_sol}
\end{figure}
%
The matched beam resembles that of the uncoupled FODO lattice in Fig.~\ref{fig:matched_vs_sc_fodo}; most notably, it is round at the symmetry points in the lattice. The differences are found in the apparent emittances: their oscillation amplitude is larger within the drift spaces and quadrupoles, and there is a significant additional emittance exchange within the solenoids. 


\subsubsection{Effective transfer matrices}

The effective linear transfer matrix generated by the lattice and matched beam can be calculated by tracking test particles subject to the internal fields of the matched beam.\footnote{The matched beam is a function of \textit{one} of the transfer matrix eigenvectors; the second eigenvector does not necessarily correspond to a matched solution in the same lattice. For example, the two solutions in the uncoupled FODO lattice share the same effective transfer matrix, but the two solutions in the skew quadrupole lattice do not share the same effective transfer matrix. In the latter case, the unused eigenvector is a matched solution in a lattice in which the sign of each skew term is reversed (a mirror reflection in one plane).} The eigenvalues of each effective transfer matrix are plotted in the complex plane as space charge is increased in Fig.~\ref{fig:effective_transfer_matrix_eigvals}.
%
\begin{figure}[!p]
    \centering
    \includegraphics[width=\textwidth]{Images/chapter2/eigvals.png}
    \caption{Eigenvalues of the transfer matrix of the effective lattice generated by the matched beam, plotted in the complex plane.}
    \label{fig:effective_transfer_matrix_eigvals}
\end{figure}
%
We observe that the difference between the phase advances $\Delta = |\mu_2 - \mu_1|$, which is a measure of the coupling strength in the effective lattice, is never zero when space charge is nonzero. (In the top two rows, $\Delta$ is very small and the plotted eigenvalues lie nearly on top of one another.) We also observe that the matched beam space charge cancels out some of the bare lattice coupling; for example, in the skew quadrupole lattice $\Delta$ is large in the left column (zero space charge) but nearly zero in the far right column. This is not true when coupling is included using solenoid magnets; $\Delta$ instead remains relatively constant.  


\section{Relevance to experiment}

These studies pave the way for future research on the stability of the Danilov distribution using perturbations around the matched envelope \cite{Goswami2016}, as well as halo formation using the particle-core model \cite{Wangler1998, Gluckstern1994, Gluckstern1998}. Our findings are also relevant to the elliptical painting method, which aims to produce a Danilov distribution in a ring. The elliptical painting method is optimized when the painting proceeds along an eigenvector of the \textit{effective} transfer matrix generated by the matched beam.\footnote{When a beam is painted into a ring, the $x$-$x'$ and $y$-$y'$ projections are more-or-less matched by the end of injection. This is because new particles are continuously added to the distribution as the circulating particles move along their Courant-Snyder ellipse, filling in the available phase space.}
%

Holmes et al performed PIC simulations of elliptical painting in the SNS, concluding that the method is feasible \cite{Holmes2018}. These simulations are described in more detail in Chapter \ref{chap-3}; for now, we comment on two findings:
%
\begin{quote}
    \textbf{Finding 1}: The ideal painting path in the SNS is a line in the $x$-$y'$ plane.
\end{quote}
%
The reasoning in \cite{Holmes2018} was that, due to the geometry of the injection region, variation of $x'$ leads to scraping losses. We have found another reason why this painting path is ideal: in general, the matched solution with space charge at a symmetry point ($\alpha_x = \alpha_y = 0$) in an uncoupled lattice is upright ($\nu \approx \pi/2$). The injection point in the SNS ring has $\beta_x \approx \beta_y$ and $\alpha_x \approx \alpha_y = 0$, so this general principle should apply. And it does: Fig.~\ref{fig:matched_env_SNS} shows the matched envelope at the SNS injection point with realistic beam parameters: $\varepsilon_2 = 20$ mm mrad, energy = 0.8 GeV, bunch length $\approx$ 3/4 ring length, and intensity = $0.75 \times 10^{14}$.
%
\begin{figure}[!p]
    \centering
    \includegraphics[width=0.8\textwidth]{Images/chapter2/matched_env_SNS.png}
    \caption{Matched beam envelope at the SNS injection point. Parameters: $\varepsilon_1 = 0$, $\varepsilon_2 = 20$ mm~mrad, energy = 0.8 GeV, intensity = $0.75 \times 10^{14}$, bunch length $\approx$ 3/4 ring length.}
    \label{fig:matched_env_SNS}
\end{figure}
%
Variation of $\nu$ has the potential to generate large mismatch oscillations due to the linear coupling from the beam. During painting, this means that the ideal cross-plane correlations will be blurred, and it is also likely that a uniform density ellipse will not be maintained since particles will not be injected onto the beam edge. In Fig.~\ref{fig:mismatched_env_SNS}, the same beam as in Fig.~\ref{fig:matched_env_SNS} is tracked, but the $\nu$ parameter is changed by $\pi / 4$. 
%
\begin{figure}[!p]
    \centering
    \includegraphics[width=0.8\textwidth]{Images/chapter2/mismatched_env_SNS.png}
    \caption{Mismatched beam envelope at the SNS injection point. The parameters are the same as in Fig.~\ref{fig:matched_env_SNS}. The red ellipse is the initial beam, and black ellipses show 100 subsequent turns overlayed on the same plot.}
    \label{fig:mismatched_env_SNS}
\end{figure}
%
The red line shows the initial beam; the beam ellipses over the next 100 turns are overlayed on the same plot. The mismatch causes a large space-charge-driven emittance exchange, causing the beam to rotate turn-by-turn and blur the $x$-$y'$ correlation. The superposition of these ellipses is not a self-consistent distribution.
%
\begin{quote}
    \textbf{Finding 2}: When a solenoid is added to an uncoupled lattice with tune split $\nu_x$ - $\nu_y$, the final beam quality is, to a degree, insensitive to the tune split. 
\end{quote}
%
The reasoning in \cite{Holmes2018} was that the beam adjusts its shape such that the depressed horizontal and vertical tunes are equal. Using the envelope model and a simple lattice, we have confirmed that such a process can occur. It is possible to find a matched beam even for a significant tune split in an uncoupled lattice: the horizontal and vertical emittances adjust to produce an asymmetric tune depression between the horizontal and vertical planes. This may be a contributing factor to Finding 2. But there seems to be a more simple reason why the beam is insensitive to the tune split: the transfer matrix of the ring with the solenoid produces two eigenvectors, each of which rotates in a circle in the $x$-$y$ plane at the injection point. This is also true when the linear space charge of the Danilov distribution is included in the transfer matrix. This fact does not depend on the tune split in the original lattice, so Finding 2 is no surprise.

Finally, a yet-unconsidered modification to the SNS is to turn on the skew quadrupole correctors in the ring. It would be ideal to avoid the large transverse momentum kicks required to paint a rotating beam, which introduce technical challenges as well as opportunities for beam loss, and instead, vary only the relative positions of the injected and circulating beams. It is thus worth exploring whether skew quadrupoles can be used to change the shape of the matched beam so that the required angular kick is minimized.



\section{Summary}

The evolution of the Danilov distribution is given by its envelope equations. An iterative procedure to calculate the matched envelope was developed by observing that the matched beam is a function of a single eigenvector of an unknown coupled transfer matrix. 

The method was demonstrated in a simple FODO lattice, which was then modified to study the effects of unequal tunes and linear coupling. Two matched solutions, one for each choice of intrinsic emittance, were obtained for each lattice and space charge strength. The primary difference between these solutions was the sign of their angular momentum. A common finding among nearly all the cases was that the shape of the matched beam in phase space remained approximately the same as space charge was increased; the main effect of space charge was to increase the average beam area within the lattice, as well as to introduce an exchange of the horizontal and vertical emittances.

Several observations from simulations in \cite{Holmes2018} were illuminated by these studies. In particular, by computing the matched beam envelope at the SNS injection point, further justification was given to the choice of the $x$-$y'$ painting path, which produces a circular beam at the injection point.
\chapter{Simulations} \label{chap-3}

Holmes et al. recently performed PIC simulations in the original ORBIT code to determine the feasibility of elliptical painting in the SNS \cite{Holmes2018}. Their findings are reviewed in this chapter. Additionally, the simulations are extended in PyORBIT to include updated experimental constraints. First, the computational model is briefly described. 


\section{Computational model}

The PyORBIT code tracks a Bunch object containing the 6D phase space coordinates. The accelerator is modeled as a series of nodes, each of which modifies the phase space coordinates in some way. Following the TEAPOT approach [\ref{}], single particles are transported using symplectic maps derived from the Hamiltonian. Here, we briefly describe the approach to nonlinear magnetic fringe fields and collective effects. 

\subsection{Space charge}

A major component of beam physics simulations is the calculation of the space charge force. Direct Coloumb sums are currently infeasible. The Vlasov equation can be solved directly, but this is difficult in 2D and 3D. The particle-in-cell (PIC) method is a “best of both worlds” approach in which an $N$ particle bunch is represented by $M$ macroparticles, where $M \ll N$. The macroparticles are tracked according to Eq.~\eqref{eq:eom_with_spacecharge}. The electric field is obtained by solving Eq.~\eqref{eq:Poisson} on a grid. The key step is transforming between the discrete and continuous representation. The PIC loop is shown in Fig.~\ref{fig:pic_loop}. 
%
\begin{figure}[!p]
    \centering
    \includegraphics[width=\textwidth]{Images/chapter3/pic_loop.png}
    \caption{\label{fig:pic_loop}The particle-in-cell loop.}
    \vfill
    \vspace*{2.5cm}
    \vfill
    \includegraphics[width=\textwidth]{Images/chapter3/poisson.png}
    \caption{\label{fig:poisson}Solution of Poisson's equation on the doubled grid.}
    
\end{figure}
%

First, the charge density $\rho_{i,j}$ is obtained on a grid. A common method is to treat each macroparticle as a rectangular, uniform density cloud of charge with dimensions equal to the grid spacing, assigning a fractional charge to each bin according to the fraction of the cloud overlapping with that bin \cite{Birdsall1975}. Second, Poisson’s equation is solved on the grid. The method used in PyORBIT follows \cite{Hockney1981}. The potential is written as the convolution of a Green's function $G(\mathbf{x})$ with the charge density $\rho(\mathbf{x})$:
%
\begin{equation}
    \Phi(\mathbf{x}) = G(\mathbf{x}) * \rho(\mathbf{x}).
\end{equation}
%
We then exploit the convolution theorem \cite{Arfken1985} to write
%
\begin{equation}
    \mathcal{F}[\Phi(\mathbf{x})]
    =
    \mathcal{F}[G(\mathbf{x})] \cdot \mathcal{F}[\rho(\mathbf{x})]
\end{equation}
%
where $\mathcal{F}$ represents the Fourier transform. For a grid with $N$ bins per dimension, the time-complexity of the convolution is $O(N^2)$. The Fourier transform reduces this to $O(N \log N)$. To create periodic boundary conditions, the grid is doubled in each dimension. The Green's function is mirror-reflected onto these new regions, while the charge density is set to zero. The potential is solved for on the extended grid, after which the unphysical regions are discarded. An example is shown in Fig.~\ref{fig:poisson}. Third, using the same weighting method as the first step, the gradient of the potential is interpolated at the particle positions. Finally, the particle momenta are updated using an appropriate integration scheme.

Care must be taken when choosing the number of macroparticles, grid size, and integration step size. In the following simulations, 128 bins are used in each dimension. The number of macroparticles changes during injection, but the final number of particles is usually at least $3 \times 10^{5}$.

In rings where the coasting beam approximation is valid, the longitudinal and transverse dimensions are treated separately. In PyORBIT, a longitudinal space charge node acts on the bunch once per turn. Two models are included for the transverse space charge calculation: the 2.5D model and the sliced model. In the 2.5D model, Poisson’s equation is solved once for a charge density obtained by projecting the entire bunch onto the $x$-$y$ plane; the transverse space charge forces are then weighted according to the longitudinal density. In the sliced model, the bunch is longitudinally sliced, and Poisson’s equation is solved for each slice. 

\subsection{Wake fields}

In the discussion of space charge thus far, the beam was assumed to be in free space. In reality, the beam is in a conducting vacuum chamber. Charged particles leave so-called wake fields on the conducting surface, which then act on other particles or on the same particle in a ring, possibly leading to instability. The treatment of wake fields can be challenging and is introduced in \cite{Chao1993}. No details are described here; we just mention that PyORBIT takes these effects into account.


\subsection{Fringe fields}

Quadrupole, dipole, and solenoid magnets are finite in length; the magnetic fields outside the core are nonlinear. These are referred to as fringe fields. [...]


\subsection{Other effects}

An important effect during charge-exchange injection is Coulomb scattering during passage through the stripper foil. [...].

Finally, longitudinal focusing in the ring is provided by two RF cavities. The harmonic frequency $h$ is defined as the RF frequency divided by the revolution frequency of the beam; one cavity operates at $h = 1$ and the other operates at $h = 2$, both at an amplitude near 5 kV. The energy gain $\Delta \epsilon$ for a particle passing through the cavity is approximated as  
%
\begin{equation}
    \Delta \epsilon = q V \sin((h \phi + \phi_0).
\end{equation}
%
The particle phase $\phi$ is zero for the synchronous particle.



\section{Fringe field correction}

It was found in \cite{Holmes2018} that in an otherwise linear lattice, fringe fields tend to eliminate any cross-plane correlations in the beam when the tunes are near the difference resonance $\nu_x \approx \nu_y$. To demonstrate this, we track a Danilov distribution in a linearized version of the SNS ring. Fig.~\ref{} shows the results when fringe fields are included as the horizontal and vertical tunes move toward each other. There is clearly nonlinear coupling between the horizontal and vertical motion; in the end, the distribution is a superposition of rotating and counter-rotating modes and the intrinsic emittances reduce to the apparent emittances. [...]

In Fig.~\ref{} the simulation is repeated with a solenoid magnet inserted in the ring. The Danilov distribution now maintains its form.



\section{Painting simulations}

The injection process is simulated by adding particles to the bunch at the foil location on each turn. The minipulse from the linac has an RMS emittance of approximately 0.3 mm~mrad. A so-called JOHO distribution is used:
%
%
The beam Twiss parameters are usually assumed to be matched to the lattice ($\beta_x \approx \beta_y \approx 10$ m/rad, $\alpha_x \approx \alpha_y \approx 0$ rad), but recent measurements indicate that they could be significantly different, particularly the $\alpha$ parameter which determines the beam divergence. To be safe, we use values close to these measurements. A Gaussian distribution is used for the longitudinal distribution with RMS energy spread of a few MeV. Since the minipulse is much smaller than the final accumulated pulse, the exact details of the minipulse distribution are not important.


\begin{figure}[!p]
    \centering
    \begin{subfigure}{\textwidth}
        \includegraphics[width=\textwidth]{Images/chapter3/snapshots.png}
    \end{subfigure}
    \vfill
    \vspace*{1.0cm}
    \vfill
    \begin{subfigure}{0.7\textwidth}
        \includegraphics[width=\textwidth]{Images/chapter3/emittances.png}
    \end{subfigure}
    \caption{Simulation of elliptical painting.}
    \label{fig:my_label}
\end{figure}
\chapter{Diagnostics} \label{chap-4}

Determining the similarity between a painted distribution in the SNS and a Danilov distribution requires measurement of the 4D transverse phase space distribution. A direct measurement using a slit-scan \cite{Cathey2018} is not possible at high energy, so the distribution must be reconstructed from lower-dimensional projections. In this chapter, we first describe the available hardware to measure such projections in the SNS. We then describe several methods to perform the reconstruction using 1D and/or 2D projections, as well as the implementation of these methods in the SNS.


\section{Available hardware and constraints}

The phase space measurement must be performed in the ring-target beam transfer (RTBT) section of the SNS after the beam has been accumulated in the ring. The RTBT is effectively an extension of the ring that is traversed once. It is straightforward to vary the number of accumulated turns to measure the beam at any time during injection. 

The RTBT optics are shown in Fig.~\ref{fig:rtbt_optics} along with the locations of four wire-scanners — WS20, WS21, WS23, and WS24 — near the target.
%
\begin{figure}[!p]
    \includegraphics[width=\textwidth]{Images/chapter4/rtbt_optics.pdf}
    \caption{$\beta$ functions, phase advances, and quadrupole/wire-scanner locations in the second half of the RTBT. The plot ends at the spallation target.}
    \label{fig:rtbt_optics}
\end{figure}
%
Each wire-scanner consists of three thin tungsten wires mounted on a fork — one wire is vertical, another is horizontal, and another is tilted at a forty-five-degree angle. The 1D projections of the distribution onto axes perpendicular to the wires are generated by moving the fork across the beam and measuring the charge induced by secondary emission from the wires \cite{Henderson2014}. The four wire-scanners can be run in parallel and take approximately five minutes to move across the beam and return to their original positions. Their step size is 3 mm and their dynamic range is approximately 100. They are run at a beam pulse frequency of 1 Hz.\footnote{Each data point corresponds to a separate beam pulse, so the measurement relies on pulse-to-pulse stability.} The measured profiles can be used to estimate $\langle{xx}\rangle$, $\langle{yy}\rangle$, and $\langle{uu}\rangle$, where the $u$ axis is tilted at angle $\phi = \pi/4$ above the $x$ axis, as well as $\langle{xy}\rangle$ from
%
\begin{equation}
    \langle{xy}\rangle = \frac{\langle{uu}\rangle - \langle{xx}\rangle \cos^2\phi - \langle{yy}\rangle \sin^2\phi}{2\sin\phi\cos\phi}
    .
\end{equation}
%

The SNS employs a target imaging system (TIS) to measure the 2D projection of the distribution on the target \cite{Blokland2010}. The SNS target is a stainless steel vessel containing liquid mercury. Its nose is prepared with a Cr:Al2O3 coating that releases light when impacted by the proton beam. Due to the high-radiation environment, the light is collected by a mirror, deflected, and focused onto an optical fiber bundle which guides the light to a camera some distance away. The TIS configuration is shown in Fig.~\ref{fig:tis}.
%
\begin{figure}[!p]
    \centering
    \includegraphics[width=\textwidth]{Images/chapter4/tis1.png}
    \caption{Configuration of the SNS target imaging system. (From \cite{Blokland2010}.)}
    \label{fig:tis}
\end{figure}
%

The optics in the RTBT can be modified, but there are constraints. The $\beta$ functions should be kept below $\approx$ 30 m/rad in the wire-scanner region and below $\approx$ 100 m/rad closer to the target to avoid excess beam loss. At the target, it is best to keep the $\beta$ functions near their default values of $\beta_x \approx$ 60 m/rad and $\beta_y \approx$ 6 m/rad to satisfy peak density and beam size requirements on the target. In addition to these constraints, quadrupoles in the wire-scanner region share power supplies: there is a horizontal group \{QH18, QH20, QH22, QH24\} and a vertical group \{QV19, QV21, QV23, QV25\}. The last five magnets — QH26, QV27, QH28, QV29, and QH30 — are individually controlled.


\section{4D phase space reconstruction from 1D projections}\label{sec:Phase space reconstruction from 1D projections}

\subsection{Method description}

The covariance matrix $\bm{\Sigma}$ can be reconstructed from 1D projections \cite{book:Minty2003, Woodley2000, Prat2014}. We seek to reconstruct $\bm{\Sigma}$ at position $a$ by measuring $\langle{xx}\rangle$, $\langle{yy}\rangle$ and $\langle{xy}\rangle$ at position $b$, downstream of $a$. Assuming linear transport, the two covariance matrices are related by
%
\begin{equation}
    \bm{\Sigma}_b = \mathbf{M} \bm{\Sigma}_a \mathbf{M}^T,
\end{equation}
%
where $\mathbf{M}$ is the linear transfer matrix from $a$ to $b$. We repeat the measurement at least four times with different transfer matrices — either by changing the measurement location or by changing the machine optics — and write
%
\begin{equation}
    \begin{bmatrix}
        {\langle{xx}\rangle}^{(1)} \\
        {\langle{xy}\rangle}^{(1)} \\
        {\langle{yy}\rangle}^{(1)} \\
        {\langle{xx}\rangle}^{(2)} \\
        {\langle{xy}\rangle}^{(2)} \\
        {\langle{yy}\rangle}^{(2)} \\
        {\langle{xx}\rangle}^{(3)} \\
        {\langle{xy}\rangle}^{(3)} \\
        {\langle{yy}\rangle}^{(3)} \\
        \vdots
    \end{bmatrix}_b
    = \mathbf{A}
    \begin{bmatrix}
        \langle{xx}\rangle \\
        \langle{xx'}\rangle \\
        \langle{xy}\rangle \\
        \langle{xy'}\rangle \\
        \langle{x'x'}\rangle \\
        \langle{x'y}\rangle \\
        \langle{x'y'}\rangle \\
        \langle{yy}\rangle \\
        \langle{yy'}\rangle \\
        \langle{y'y'}\rangle \\
    \end{bmatrix}_a
    .
\end{equation}
%
The superscripts represent the measurement index. The transpose of the coefficient matrix $\mathbf{A}$ for a single measurement is
%
\begin{equation}
    \mathbf{A}^T = 
    \begin{bmatrix}
        M_{11}M_{11} & M_{11}M_{31} & M_{31}M_{31} \\
        2M_{11}M_{12} & M_{12}M_{31} + M_{11}M_{32} & 2M_{31}M_{32} \\
        2M_{11}M_{13} & M_{13}M_{31} + M_{11}M_{33} & 2M_{31}M_{33} \\
        2M_{11}M_{14} & M_{14}M_{31} + M_{11}M_{34} & 2M_{31}M_{34} \\
        M_{12}M_{12} & M_{12}M_{32} & M_{32}M_{32} \\
        2M_{12}M_{13} & M_{13}M_{32} + M_{12}M_{33} & 2M_{32}M_{33} \\
        2M_{12}M_{14} & M_{14}M_{32} + M_{12}M_{34} & 2M_{32}M_{34} \\
        M_{13}M_{13} & M_{13}M_{33} & M_{33}M_{33} \\
        2M_{13}M_{14} & M_{14}M_{33} + M_{13}M_{34} & 2M_{33}M_{34} \\
        M_{14}M_{14} & M_{14}M_{34} & M_{34}M_{34}
    \end{bmatrix}
\end{equation}
%
where $M_{ij}$ is the $i$,$j$ element of the transfer matrix for that measurement. The system is solved using linear least squares (LLSQ). 

The measurement has a geometric interpretation, illustrated in Fig.~\ref{fig:ws_emittance_measurement} for the 2D case.
%
\begin{figure}[!p]
    \centering
    \includegraphics[width=0.9\textwidth]{Images/chapter4/ws_emittance_measurement2.png}
    \caption{Illustration of 2D emittance measurement.}
    \label{fig:ws_emittance_measurement}
\end{figure}
%
Each measured beam size defines two lines in the $x$-$x'$ plane at $b$; when the lines are transported back to $a$, their intersection bounds the phase space ellipse. In the general case, each measurement at $b$ defines a 2D surface in 4D phase space and the intersection of these surfaces at $a$ bounds the phase space ellipsoid.


\subsection{Implementation in the SNS}

To perform the wire scans, the beam is set to a pulse frequency of 1 Hz, and the beam loss monitors in the wire-scanner region are masked due to the higher-than-normal losses when the wires cross the beam core. Wire-scanner data acquisition is performed by the Profile Tools and Analysis (PTA) application. After the four wire-scanners complete their scan, a time-stamped file is produced containing the measured profiles.

The four wire-scanners produce four equations, exactly determining the cross-plane moments, so no optics changes are needed in principle; however, additional measurements should reduce the error. In the 2D case, it is typically recommended to space $n$ measurements by $\pi / n$ in phase advance \cite{book:Minty2003}. This may be due to the geometric interpretation of Fig.~\ref{fig:ws_emittance_measurement}: in normalized phase space, the rotation angle of the measurement lines is equivalent to the phase advance and the lines are evenly spaced around the phase space ellipse. The four wire-scanners in the RTBT are already somewhat evenly spaced in phase advance, and it was determined that a 30$\degree$ window around each wire-scanner would provide sufficient coverage. 

Due to the shared power supplies of the quadrupoles in the wire-scanner region, there is limited control of the phase advances between the wire-scanners. We instead vary the phase advances from QH18 (the first varied quadrupole) to WS24 (the last wire-scanner), which changes the phase advances at WS20, WS21, and WS23 by similar amounts. To set the phase advances at WS24 while constraining the beam size in the wire-scanner region, two power supplies (eight quadrupoles) upstream of WS24 were varied using an optimizer that minimizes the following cost function:
%
\begin{equation}
    C(\mathbf{g}) = \left\Vert{\tilde{\bm{\mu}} - \bm{\mu} }\right\Vert^2
    + 
    \epsilon
    \left\Vert
    \Theta\left(
        \tilde{\bm{\beta}}_{max} - \bm{\beta}_{max}
    \right)
    \right\Vert^2
    .
\end{equation}
%
The quadrupole field strengths are contained in the vector $\mathbf{g}$. The calculated and desired phase advances at WS24 are $\bm{\mu} = (\mu_x, \mu_y)$, and $\tilde{\bm{\mu}} = (\tilde{\mu}_x, \tilde{\mu}_y)$, respectively. The maximum calculated and allowed $\beta$ functions in the wire-scanner region are $\bm{\beta}_{max} = (\beta_{x_{max}}, \beta_{y_{max}})$ and $\tilde{\bm{\beta}}_{max} = (\tilde{\beta}_{x_{max}}, \tilde{\beta}_{y_{max}})$, respectively. $\Theta$ is the Heaviside step function. Finally, $\epsilon$ is a constant.\footnote{We are assuming that the beam is approximately matched to the lattice optics so that the calculated phase advances are close to the true phase advances.}

After the model optics are computed, the live quadrupole settings must be changed. The SNS employs a machine protection system (MPS) that will cause the machine to trip if the RTBT quadrupole strengths wander outside a certain window, so this window is extended beforehand. Additionally, the MPS will activate if the fractional change in field strength is too large; to solve this problem, the field strength is changed in small steps. 

A GUI application to perform the above tasks was developed in the OpenXAL framework for use in the SNS control room. In the first pane of the application, the user can set the phase advances at WS24 and view the model optics and phase advances throughout the RTBT. In the second pane, the user can load wire-scanner output files and choose the reconstruction location. These files contain the wire-scanner profiles, RMS parameters, and Gaussian fit parameters. They also contain an integer that defines the machine state at the time of the measurement. The application reads this number, synchronizes the model with the machine state, and computes the transfer matrices from the wire-scanners to the reconstruction location. The RMS moments and transfer matrices are then used to reconstruct the covariance matrix. The resulting beam parameters are printed and compared to the model lattice parameters. The 2D projections of the covariance ellipsoid are plotted along with the measurement lines, with the option to view in normalized coordinates. 



\subsection{Measurement of a production beam}

The multi-optics method was tested on a fully accumulated production beam. The phase advances at WS24 were varied in a 30$\degree$ range over ten steps: the first half of the scan held the vertical phase advance fixed while varying the horizontal phase advance, and the second half of the scan held the horizontal phase advance fixed while varying the vertical phase advance. The result of the reconstruction is shown in Fig.~\ref{fig:prod_meas} for a location just before QH18.
%
\begin{figure}[!p]
    \centering
    \begin{subfigure}{0.9\textwidth}
        \centering
        \includegraphics[width=\textwidth]{Images/chapter4/prod_meas_lines.pdf}  
    \end{subfigure}
    \par\medskip
    \begin{subfigure}{0.6\textwidth}
        \centering
        \begin{tabular}{lll}
            \small\textbf{Parameter} & \small\textbf{Measurement} & \small\textbf{Model} \\
            \midrule
            \small$\beta_x$ [m/rad] & \small22.06 $\pm$ 0.29 & \small22.00 \\
            \small$\beta_y$ [m/rad] & \small4.01 $\pm$ 0.02 & \small3.81 \\
            \small$\alpha_x$ & \small2.33 $\pm$ 0.04 & \small2.37 \\
            \small$\alpha_y$ & \small-0.49 $\pm$ 0.01 & \small-0.60 \\
            \small$\varepsilon_1$ [mm~mrad] & \small33.02 $\pm$ \small0.05 & - \\
            \small$\varepsilon_2$ [mm~mrad] & \small25.67 $\pm$ \small1.03 & - \\
            \small$\varepsilon_x$ [mm~mrad] & \small32.85 $\pm$ \small0.05 & - \\
            \small$\varepsilon_y$ [mm~mrad] & \small25.87 $\pm$ \small0.12 & - \\
          \end{tabular}
    \end{subfigure}
    \par\medskip
    \caption{Reconstructed beam parameters and graphical output from a multi-optics emittance measurement of a production beam.}
    \label{fig:prod_meas}
\end{figure}
%
The best-fit ellipses in the $x$-$x'$ and $y$-$y'$ planes are normalized by the reconstructed Twiss parameters. The uncertainties in the beam parameters were calculated by propagating the standard deviations of the ten reconstructed moments obtained from the LLSQ estimator.\footnote{See Appendix A of \cite{Faus-Golfe2016}.} The reconstructed Twiss parameters are close to the model parameters computed from the linear transfer matrices of the ring and RTBT, showing that the beam is matched. The intrinsic emittances are almost equal to the apparent emittances, showing that there is very little cross-plane correlation in the beam. This is expected for a production beam.



\subsection{Sensitivity to errors}

A comprehensive study of errors in the multi-optics 4D emittance measurement was completed at the SwissFEL Injector Test Facility (SITF) by Prat and Aiba in \cite{Prat2014}. They considered errors in the measured moments, quadrupole field and alignment errors, beam energy errors, beam mismatch at the reconstruction point, and dispersion/chromaticity \cite{Mostacci2012}, concluding that the multi-optics measurement remained accurate, reporting $< 5\%$ uncertainty in the intrinsic emittances. We initially performed similar studies in PyORBIT using envelope tracking to estimate the reconstruction errors in the RTBT, also concluding that the method should remain accurate \cite{Hoover2021-IPAC}. Space charge forces, which can render the method invalid for high-perveance beams \cite{Anderson2002}, can be neglected: the space charge tune shift in the ring is around 3\%, and the distance between the reconstruction and measurement locations is much smaller than the length of the ring. 

In summary, the multi-optics emittance measurement is feasible in the SNS. The only downside is the long measurement time, for the following reasons. First, we are not only interested in the beam emittances at a single time but are also interested in the growth and evolution of the emittances throughout accumulation. For example, it could be possible for $\varepsilon_2$ to remain small (as desired) in the first half of accumulation before growing to a much larger value in the second half of accumulation. Measurement of this emittance evolution would convey valuable information about the beam dynamics and allow for qualitative comparison with computer simulation. Second, it would be beneficial to quickly evaluate various machine states, the best of which will be unknown during initial experiments. Finally, a practical point: the time reserved for accelerator physics (AP) experiments at the SNS is limited, and the setup for initial experiments will be much longer than typical AP experiments for reasons discussed in Chapter \ref{chap-5}. Thus, the fixed-optics method — in which only four profiles are used in the reconstruction — is preferred.\footnote{An additional benefit of the fixed-optics method is that there is no potential for steering errors in the RTBT, which could enhance beam loss.} A modest reduction in accuracy for the increase in speed is warranted since weak cross-plane correlations ($\varepsilon_1 \varepsilon_2 \approx \varepsilon_x \varepsilon_y$) are uninteresting for our purposes and do not need to be resolved.

Using only one set of optics from the previous scan resulted in a covariance matrix that was not positive-definite, producing imaginary intrinsic emittances. We label this a failed fit. A nonlinear solver \cite{Raimondi1993} or Cholesky decomposition \cite{Agapov2007} can be used to ensure a valid covariance matrix, but we found that the answer depended strongly on the initial guess provided to the solver and on which measurement in the scan was used in the reconstruction.

To investigate the failure of the fixed-optics method, a covariance matrix was generated with cross-plane moments set to zero and within-plane moments matched to the lattice optics, then tracked to the wire-scanners using the known transfer matrices. The reconstruction was then performed 1000 times with 3\% random noise added to the moments.\footnote{To select the proper noise level, the wire-scanners were run seven times without changing the machine optics, producing seven estimated moments for each of the three wires on each of the four wire-scanners. The profiles are highly reproducible: for each wire, the maximum difference between any two moments was always less than 3\% of the mean moment. Therefore, the noisy moments were sampled within $\pm$ 3\% of the true moments.} Failed fits were discarded. Fig.~\ref{fig:prod_sensitivity} shows that the reconstructed intrinsic emittances in the successful trials are very sensitive to changes to the ``measured" moments. 
%
\begin{figure}[!p]
    \vspace*{5cm}
    \includegraphics[width=\textwidth]{Images/chapter4/prod_sensitivity.pdf}
    \caption{Monte Carlo simulation of a fixed-optics emittance measurement in the RTBT. Trials were repeated until several thousand successful fits were obtained. 3\% noise was assumed for the measured moments. Transfer matrix errors were ignored. The correct values are $\varepsilon_1$ = $\varepsilon_x$ = 32 mm~mrad, $\varepsilon_2$ = $\varepsilon_y$ = 20 mm~mrad.}
    \label{fig:prod_sensitivity}
    \vspace*{5cm}
\end{figure}
%
Unlike the apparent emittances, the intrinsic emittances are strongly correlated and are not centered on the correct values. We refer to the difference between the mean emittances and the true emittances as the \textit{bias}.\footnote{The bias and strong correlation between the intrinsic emittances stems from the fact that $\varepsilon_1\varepsilon_2 \le \varepsilon_x\varepsilon_y$. In the case at hand, the reconstructed $\varepsilon_x$ and $\varepsilon_y$ are essentially constant in each trial.}

Sensitivity of fixed-optics 4D emittance measurements was observed by Woodley and Emma \cite{Woodley2000} and studied more recently by Agapov, Blair, and Woodley \cite{Agapov2007} as well as Faus-Golfe et al. \cite{Faus-Golfe2016}, all in the context of design studies for a future International Linear Collider (ILC). The motivation for these studies is that a flat beam ($\varepsilon_y \ll \varepsilon_x$) would be ideal for the ILC, and that $\varepsilon_y$ can be minimized by measuring and removing any cross-plane correlation in the beam. Woodley and Emma proposed to abandon the fixed-optics method due to the bias in the reconstructed intrinsic emittances introduced by large errors in the measured moments, suggesting to instead measure the 2D emittance and iteratively minimize $\varepsilon_y$. 

Agapov, Blair, and Woodley revisited this problem and showed that the linear system used to reconstruct the cross-plane moments can easily become ill-conditioned. The sensitivity of a linear system $\mathbf{A} \mathbf{x} = \mathbf{b}$ to errors in $\mathbf{b}$ is determined by the condition number $C = \Vert \mathbf{A} \Vert \Vert \mathbf{A}^{-1} \Vert$ (or the pseudo-inverse $\mathbf{A}^\dagger = (\mathbf{A}^T\mathbf{A})^{-1} \mathbf{A}^T$ if $\mathbf{A}$ is not square) where $\Vert \dots \Vert$ is a matrix norm \cite{Golub1985}. As an example, consider four wire-scanners that are evenly spaced in phase advance and connected by rotation matrices. Since the transfer matrices are uncoupled, there are three independent subsystems to solve: $x$-$x'$, $y$-$y'$, and the cross-plane moments. Let the coefficient matrices for these subsystems be $\mathbf{A}_{xx}$, $\mathbf{A}_{yy}$, and $\mathbf{A}_{xy}$, respectively, and the condition numbers be $C_{xx}$, $C_{yy}$, and $C_{xy}$. Recall that the within-plane moments are overdetermined while the cross-plane moments are exactly determined. Fig.~\ref{fig:fodo_condition_number} plots the inverse of these condition numbers as a function of the wire-scanner spacing.
%
\begin{figure}[!p]
    \centering
    \vspace*{2cm}
    \includegraphics[width=\textwidth]{Images/chapter4/fodo_condition_number.pdf}
    \caption{Condition numbers of the coefficient matrices produced by four wire-scanners that are evenly spaced in phase advance and connected by rotation matrices.}
    \label{fig:fodo_condition_number}
    \vspace*{2cm}
\end{figure}
%
$C_{xx}$ and $C_{yy}$ approach $\infty$ when the spacing is $\pi/2$ in their respective planes, in which case two pairs of measurements provide degenerate information, while $C_{xy}$ depends on the difference between the phase advances. The pattern will be more complicated for different optics and/or additional wire-scanners. The error and uncertainty in the emittance reconstruction, as well as the number of failed fits, mirrors these condition numbers. Using this framework, Faus-Golfe et al. developed analytical formulas to determine whether a given system can accurately measure the intrinsic emittances. They also suggested that the planned ILC emittance measurement station, which contained four wire-scanners, could be modified to reduce the sensitivity. 

We performed a similar modification to the RTBT wire-scanner region. To find a new set of optics, the phase advances at WS24 ($\mu_x$, $\mu_y$) were varied in a $90\degree$ window centered on their nominal values ($\mu_{x0}$, $\mu_{y0}$); at each setting, the condition numbers were calculated and the reconstruction was simulated with true emittances $\varepsilon_x  = \varepsilon_y = \varepsilon_1 = \varepsilon_2$ = 20 mm~mrad. Failed trials were discarded. The resulting biases and standard deviations of the reconstructed emittances are plotted in Fig.~\ref{fig:rtbt_montecarlo_emittances}.
%
\begin{figure}[!p]
    \centering
    \includegraphics[width=0.9\textwidth]{Images/chapter4/rtbt_montecarlo_emittances.pdf}
    \caption{Simulated 4D emittance reconstruction errors as a function of the phase advances at WS24.}
    \label{fig:rtbt_montecarlo_emittances}
\end{figure}
%
Settings that produced no successful trials appear as white cells. The apparent emittances are not displayed because they remained within 1\% of their true values at every optics setting. Modifying the optics so that $\mu_x = \mu_{x0} + 45\degree$, $\mu_y = \mu_{y0} - 45\degree$ reduces the bias to $\approx 7\%$ and the standard deviation to $\approx 5\%$. The fraction of failed fits, which is very large along the diagonal in the figure, is reduced to zero.

It is also important to examine the effect of mismatched beam parameters on the accuracy of the reconstruction. Recall that the phase advance is the integral of the inverse of the $\beta$ function. In a periodic system, there is a unique periodic solution for the $\beta$ function, but this is not true in a transfer line such as the RTBT; thus, the phase advance in the RTBT depends on the Twiss parameters at the ring extraction point — the RTBT entrance. 

All previous phase advance calculations have assumed that the beam Twiss parameters are the same as the ring Twiss parameters at extraction. This is generally a safe assumption since turn-by-turn mismatch oscillations are washed out during painting. It is possible, however, for space charge to effectively modify the ring Twiss parameters, resulting in mismatch when entering the RTBT. This modification is small during production painting, as shown in Fig.~\ref{fig:prod_meas}, but it is expected (from simulations) that more significant mismatch could occur if the space charge density is increased and/or if the beam energy is decreased.

To examine the effect of mismatch, we first moved the operating point to $\mu_x = \mu_{x0} + 45\degree$, $\mu_y = \mu_{y0} - 45\degree$, then varied the initial Twiss parameters at BPM17 in the RTBT and repeated the Monte Carlo trials. There are four parameters: $\alpha_x$, $\alpha_y$, $\beta_x$, and $\beta_y$. We based the range of each parameter on a measurement in which the reconstructed Twiss parameters were different than the nominal Twiss parameters, shown in Table~\ref{tab:mismatch}.  
%
\begin{table}[!p]
    \centering
    \caption{Reconstructed and model Twiss parameters at BPM 17 in the RTBT (see Experiment 2 in Chapter \ref{chap-5}.)}
    \begin{tabular}{lll}
    \midrule
    \textbf{Parameter} & \textbf{Measured} & \textbf{Model} \\
    \midrule
    $\beta_x$ [m/rad] & 6.26 & 5.49 \\
    $\beta_y$ [m/rad] & 20.82 & 19.25 \\
    $\alpha_x$ & -0.89 & -0.78 \\
    $\alpha_y$ & 1.17 & 1.91 \\
    \midrule    
    \end{tabular}
    \label{tab:mismatch}
\end{table}
%
The beam mismatch is unlikely to exceed these values in future experiments.\footnote{Details about the measurement are left for Chapter \ref{chap-5}.} Therefore, to examine the effect of mismatch, we first moved the operating point to $\mu_x = \mu_{x0} + 45\degree$, $\mu_y = \mu_{y0} - 45\degree$, then varied $\beta_x$ and $\beta_y$ within a $\pm 20\%$ window around their model values, $\alpha_x$ within a $\pm 15\%$ window, and $\alpha_y$ within a $-40\%, +10\%$ window to extend beyond the measured discrepancies, and repeated the Monte Carlo trials for each initial beam, thus producing a collection of means and standard deviations for the reconstructed intrinsic emittances. The left plot in Fig.~\ref{fig:mismatch} displays the standard deviations and biases for $\varepsilon_1$ (pink) and $\varepsilon_2$ (blue).
%
\begin{figure}[!p]
    \centering
    \includegraphics[width=\textwidth]{Images/chapter4/mismatch2.pdf}
    \caption{Bias and standard deviation of $\varepsilon_1$ (pink) and $\varepsilon_2$ (blue) from simulated reconstructions in the RTBT. In each plot, the collection of points is generated by varying the initial beam Twiss parameters. The true values of the emittances are printed on the top of the figures.}
    \label{fig:mismatch}
\end{figure}
%
Although most of the points are clustered near the original bias and standard deviation of 7\% and 5\%, respectively, the bias increases to nearly 15\% in some cases, which may make it difficult to resolve weak cross-plane correlation; however, the measurement should still resolve strong cross-plane correlation. This is demonstrated in the rest of the plots in Fig.~\ref{fig:mismatch}, in which the entire process is repeated with $\varepsilon_1 / \varepsilon_2 > 1$. The bias in the reconstruction quickly decreases — the emittances are clustered around their true values. 

We conclude that with small modifications to the RTBT optics, the fixed-optics method should be sufficient for fast 4D emittance measurements in the SNS. As detailed in Chapter 5, such measurements will be needed to evaluate various machine settings within a single study period, especially in initial experiments, as well as to measure the emittance growth during accumulation for qualitative comparison with simulation. The multi-optics method should be used once a promising machine state is found (or if time allows) to reduce the uncertainty.


\subsubsection{Other uses of 1D projections}

To close this section, we mention that there is information to be gained from 1D projections in addition to the root-mean-square reconstruction just described. First, the measured projections can be compared to the ideal ``half-circle" projections of a uniform density ellipse.\footnote{The best expected case is a uniform density core with small nonlinear tails, the 1D projection of which is distinguishable from a Gaussian curve, but it may be difficult to distinguish intermediate cases with larger tails. The method we employ in the next chapter is to calculate the standard deviation of the measured profile, plot the projections of an ideal Gaussian and uniform density elliptical distribution with the same standard deviation, and visually compare the three curves. More quantitative methods may be used in the future.} Second, the projections can be used to reconstruct the $x$-$x'$ or $y$-$y'$ distribution using the tomographic methods described in the next section; it may be possible to include cross-plane information in the reconstruction using diagonal projections. 


\section{4D phase space reconstruction from 2D projections}

Tomographic methods are well-established for the reconstruction of 2D phase space distributions from 1D projections in transverse phase space \cite{Hock2014} and longitudinal phase space \cite{Evans2014}. The concept has recently been extended to the reconstruction of the 4D transverse phase space, both in theory and in practice \cite{Hock2013b, Wang2019, Wolski2020}. This section begins with a brief discussion of tomography in two dimensions as applied to beam diagnostics, then moves on to describe the accuracy and limitations of several 4D reconstruction algorithms. Finally, the use of tomography to reconstruct the 4D phase space distribution from beam images on the SNS target is discussed. 



\subsection{Tomography for beam diagnostics}

Several algorithms exist to reconstruct 2D images from 1D projections, such as filtered back-projection (FBP), algebraic reconstruction (ART) \cite{Slaney1988}, and maximum entropy (MENT) \cite{Minerbo1979}. Projections of an object are normally obtained by illuminating the object at different angles. Although the measured projections of a 2D phase space distribution ($x$-$x'$) are always along the $x$ axis, we can take advantage of the known transfer matrix $\mathbf{M}$ between the measurement location $b$ and the reconstruction location $a$ to obtain the projections at different angles in $x$-$x'$ at the reconstruction location. The measured projection of the $x$-$x'$ distribution at $b$ is
%
\begin{equation}
    p_b(x_b) = \int_{-\infty}^{\infty} f(x_b, x'_b) dx'_b.
\end{equation}
%
When the distribution is transported back to $a$, the projection will be along axis $\tilde{x}_a$, which is rotated at angle $\theta$ above the $x_a$ axis. The projection angle is computed from the transfer matrix \cite{Hock2013a}:
%
\begin{equation}\label{eq:proj_trans_1}
    \tan\theta = \frac{M_{12}}{M_{11}}.
\end{equation}
%
The distance along the projection axis will be scaled:
%
\begin{equation}\label{eq:proj_trans_2}
    r = \frac{x_b}{\tilde{x}_a} = \sqrt{M_{11}^2 + M_{12}^2}.
\end{equation}
%
The projection must then be scaled to conserve its area. The projections at $a$ and $b$ are related by 
%
\begin{equation}\label{eq:proj_trans_3}
    p_a(\tilde{x}_a) = r p_b(r \tilde{x}_a).
\end{equation}
%
Standard tomography algorithms can be applied to the scaled projections.

Reconstructing the distribution in normalized phase space can reduce errors \cite{Hock2011}. Recall the normalization matrix $\mathbf{V}$ from Eq.~\eqref{eq:CS_parameterization}. Note that
%
\begin{equation}
\begin{aligned}
    \mathbf{x}_b 
    = \mathbf{M} \mathbf{x}_a
    = \mathbf{M} \mathbf{V} (\mathbf{V}^{-1} \mathbf{x}_a)
    ,
\end{aligned}
\end{equation}
%
where $\mathbf{V}$ depends on the Twiss parameters at $a$. Eq.~\eqref{eq:proj_trans_1}, Eq.~\eqref{eq:proj_trans_2}, and Eq.~\eqref{eq:proj_trans_3} can be applied to the matrix $\mathbf{M} \mathbf{V}$ to obtain the projections in the normalized phase space at $a$. After the image is reconstructed, the true distribution can be obtained by transforming the grid coordinates using $\mathbf{V}$ and interpolating at the transformed coordinates. Any Twiss parameters can be used to form $\mathbf{V}$; if the Twiss parameters are matched to the distribution, the rotation angle of the projection will be the phase advance from $a$ to $b$, and the reconstructed distribution will be circular in the normalized phase space. 


\subsection{4D reconstruction as a series of 2D reconstructions}

Recent work by Hock et al. reduces 4D reconstruction to a series of 2D reconstructions when the $x$-$y$ projections are available \cite{Hock2013a}. The method, which we refer to as Hock's method, is as follows. Assume that the rotation angles in $x$-$x'$ and $y$-$y'$ can be independently controlled. Let the angles in $x$-$x'$ be \{$\theta_{x_1}$, $\dots$, $\theta_{x_k}$, $\dots, \theta_{x_K}$\} and the angles in $y$-$y'$ be \{$\theta_{y_1}$, $\dots$, $\theta_{y_l}$, $\dots$, $\theta_{y_L}$\}. The projections are stored in an array $\mathbf{S}$, where $\mathbf{S}_{i,j,k,l}$ is the intensity at point ($x_i$, $y_j$) on the screen for angles $\theta_{x_k}$, $\theta_{y_l}$. Consider a single row of an image, fixing $y_j$, which gives a 1D projection of a slice of the distribution onto the $x$-axis at the screen. If we fix $\theta_{y}$ and vary $\theta_{x}$, we produce a set of 1D projections that can be used to reconstruct the $x$-$x'$ phase space distribution for this slice using any 1D $\rightarrow$ 2D reconstruction method. This is repeated for each $y_j$ and $\theta_{y_l}$. Now, for each $x$ and $x'$ in the reconstruction grid, we have set of projections of the $y$-$y'$ distribution onto the $y$-axis at the screen for different $\theta_{y}$; thus, for each $x$ and $x'$ in the reconstruction grid, we can reconstruct the $y$-$y'$ distribution. This completes the reconstruction.

To test the method, the 600,000-particle distribution from Fig.~\ref{fig:Holmes} was used. In \cite{Hock2013a}, filtered back-projection (FBP) was used for the 2D reconstructions. FBP requires many projections — something that is not always possible in the context of beam diagnostics. Simultaneous algebraic reconstruction (SART) is a possible alternative when the number of projections is small. The accuracy of SART will depend on the number of projections and the range of projection angles. Fig.~\ref{fig:tomo_sim_art2D} demonstrates this by reconstructing the $y$-$y'$ distribution from 1D projections as these numbers are varied.
%
\begin{figure}[!p]
    \centering
    \vspace*{3.0cm}
    \includegraphics[width=0.7\textwidth]{Images/chapter4/tomo_sim_art2d.png}
    \caption{SART accuracy as a function of number of projections and range of projection angles.}
    \label{fig:tomo_sim_art2D}
    \vspace*{3.0cm}
\end{figure}
%
It appears that if the projection angles are distributed over a significant range, the accuracy does not improve much beyond 10-15 projections. As discussed later, 15 projections are likely near the maximum possible in the SNS for each 2D reconstruction if using Hock's method. In the following simulated 4D reconstruction, the phase advances in both planes were scanned over $180\degree$ in 12 steps; at each step, the distribution was transported to and then binned on a virtual screen. The reconstruction was performed in normalized phase space, and it was assumed that the distribution was matched to the lattice parameters. Fig.~\ref{fig:tomo_sim_target_scan} shows the simulated images in normalized space with a screen resolution of $75 \times 75$.
%
\begin{figure}[!p]
    \centering
    \includegraphics[width=\textwidth]{Images/chapter4/tomo_sim_target_scan_full.png}
    \caption{Simulated $x$-$y$ projections as the horizontal (rows) and vertical (columns) phase advances are varied.}
    \label{fig:tomo_sim_target_scan}
\end{figure}
%
Three SART iterations were used for each 2D reconstruction. The 2D projections of the reconstructed distribution are compared to those of the original distribution in Fig.~\ref{fig:tomo_sim_rec_hock_proj_2D} in normalized phase space, which shows good agreement.
%
\begin{figure}[!p]
    \centering
    \includegraphics[width=0.8\textwidth]{Images/chapter4/tomo_sim_rec_hock_proj_2D_ver.png}
    \caption{Simulated reconstruction using Hock's method (normalized phase space).}
    \label{fig:tomo_sim_rec_hock_proj_2D}
\end{figure}
%
Notice that the projections of the reconstructed distribution, such as $x$-$y'$, are present in the simulated $x$-$y$ images in Fig.~\ref{fig:tomo_sim_target_scan}. The reason is straightforward: if the phase advance in the vertical plane is $\pi$/2, then $y \rightarrow y'$ and $f(x, y) \rightarrow f(x, y')$ \cite{Hock2013a}. 

This method is preferred because it leverages 2D reconstruction algorithms. Open-source implementations of these algorithms are widely available and the conditions needed for accurate reconstructions are well-understood, primarily due to the use of tomography in medical imaging.


\subsection{Direct 4D reconstruction}

If the phase advances cannot be independently controlled or if only a very small number of projections can be collected, 2D reconstruction algorithms must be generalized to 4D. Several algorithms generalize to any number of dimensions, but they may be difficult to implement, the conditions for an accurate reconstruction may be unclear, and the time and space complexity may make the method infeasible. Here, we focus on one method that has recently been experimentally demonstrated, then mention a few more that could be explored in future work.


\subsubsection{ART}

Each measured projection on the screen produces the following set of equations:
%
\begin{equation}\label{eq:art}
    \bm{\rho} = \mathbf{P} \bm{\psi}.
\end{equation}
%
$\bm{\rho}$ is a vector of the pixel intensities on the screen and $\bm{\psi}$ is a vector of the phase space coordinates on the reconstruction grid. To form $\mathbf{P}$, we place a particle at the center of each bin in the reconstruction grid and track the particles to the screen using the transfer matrix. $\mathbf{P}_{i, j} = 1$ if particle $j$ landed in bin $i$ on the screen; otherwise, $\mathbf{P}_{i, j} = 0$. The equations produced by subsequent measurements are stacked, and the resulting system of equations is solved using a sparse least squares solver. This method has been used to reconstruct the phase space distribution in the Compact Linear Accelerator for Research and Applications (CLARA), a low-energy test facility \cite{Wolski2020}.

For an $N \times N \times N \times N$ reconstruction grid, an $N \times N$ measurement grid, and $n$ measurements, $\bm{\rho}$ has $nN^2$ elements, $\bm{\psi}$ has $n N^4$ elements, and $\mathbf{P}$ has $n N^2 \times N^4$ elements. In practice, these significant storage requirements limit the resolution of the reconstruction grid to $N \approx 50$ \cite{Wolski2020}. In Fig.~\ref{fig:tomo_sim_rec_art_proj_2D}, the method was applied to the same simulated distribution, but $8 \times 8$ projections were used instead of $15 \times 15$.
%
\begin{figure}[!p]
    \centering
    \includegraphics[width=0.8\textwidth]{Images/chapter4/tomo_sim_rec_art_proj_2D_ver.png}
    \caption{Simulated reconstruction using algebraic reconstruction (normalized phase space).}
    \label{fig:tomo_sim_rec_art_proj_2D}
\end{figure}
%
Although the main features of the distribution are present in the reconstruction, there are streaking artifacts outside the beam core that are not present in Fig.~\ref{fig:tomo_sim_rec_hock_proj_2D}, although is likely that the performance could improve if a larger number of projections were used. Unfortunately, the algorithm took hours to execute as opposed to minutes for the previous example, even with the reduced grid resolution.


\subsubsection{Additional methods}

Here are two additional methods to reconstruct the 4D phase space distribution that could be explored in future work.

Among the distributions consistent with the measured projections, MENT selects the distribution with the maximum entropy. For example, without any measurements constraining the solution, MENT will produce a uniform distribution. It can perform well with few projections and has been used for 2D reconstruction in particle accelerators \cite{Hock2013a}. The downside is that the iterative numerical solution is difficult to implement and may struggle to converge when the number of projections is large. For 4D reconstruction from $x$-$y$ projections, the MENT could be used to perform the 2D reconstructions in Hock's method, which may result in improved performance over SART. Alternatively, just as ART was generalized to four dimensions in the previous section, MENT could be generalized to four dimensions.\footnote{In principle, MENT can perform a 4D reconstruction using 1D projections \cite{Sander1979}; however, it seems a priori unlikely for this to produce an accurate result: imagine reconstructing a 3D image from 1D projections.} An analytic MENT solution has recently been derived and used for the 4D reconstruction of an SNS minipulse using $x$-$x'$ and $y$-$y'$ projections from a laser wire \cite{Wong-forthcoming}. 

Another method is to generate a particle bunch, track the bunch to the screen, weight each particle by the measured signal at the bin where it fell on the screen, and generate new particles in the region of that particle according to its weight. The advantage of this method is that it does not assume linear transport and that it can perform well with few projections. It was experimentally demonstrated by Wang et. al. in the Xi’an Proton Application Facility (XiPAF) using six projections \cite{Wang2019}. 



\subsection{Implementation in the SNS}

The idea to use SNS target images for tomographic reconstruction of the phase space distribution was proposed late in this research. Due to this fact, as well as time constraints and unexpected machine downtime, the methods described in the previous subsection were not able to be applied to real data; this is left for future work. Nonetheless, the following paragraphs describe how the reconstruction can be performed in the SNS.


\subsubsection{Optics control}

We desire independent control of the horizontal and vertical phase advances. The optics control developed for the wire-scanner measurement can be used here. The constraints are now that the $\beta$ functions remain below 30 m/rad in the wire-scanner region, below 100 m/rad before the target, and stay within 15\% of their nominal values at the target. Fig.~\ref{fig:target_phase_scan_1} shows that both the horizontal and vertical phase advances can be independently scanned in a 180$\degree$ range. Fig.~\ref{fig:target_phase_scan_2} overlays the $\beta$ functions and phase advances throughout the RTBT for every step in the scan, showing that the beam size constraints are not violated. The horizontal axis starts at the first varied quadrupole and ends at the target.
%
\begin{figure}[!p]
    \centering
    \vspace*{2.0cm}
    \includegraphics[width=\textwidth]{Images/chapter4/target_phase_scan1.png}
    \caption{Scan of the phase advances at the target.}
     \label{fig:target_phase_scan_1}
    \vspace*{2.0cm}
\end{figure}
%
\begin{figure}[!p]
    \centering
    \includegraphics[width=\textwidth]{Images/chapter4/target_phase_scan2.png}
    \caption{$\beta$ functions and phase advances vs. position for the scan in Fig.~\ref{fig:target_phase_scan_1}.}
    \label{fig:target_phase_scan_2}
\end{figure}
%

Computing each optics setting takes approximately sixteen seconds using an OpenXAL solver. It also takes time to change the magnet strengths in the machine, trigger the beam, and collect a batch of target images. The time available in most accelerator physics studies is eight to ten hours at a maximum, so we place an upper limit on the number of images collected during the scan at $15 \times 15$, for which it takes around one hour to calculate the optics and one hour to collect the images. 


\subsubsection{Image acquisition and processing}

Target image acquisition is handled entirely by the target imaging system software. Live target images are displayed in the SNS control room. It is straightforward to access the image from an OpenXAL script as an 80,000 element array. The script to perform the target scan repeatedly modifies the RTBT quadrupoles, triggers the beam, and saves the image array to a file.

The unprocessed target images are not ideal. First, to reduce pulse-to-pulse variation, the images can be averaged over a few pulses. Second, the beam passes through 2 meters of Helium at atmospheric pressure before the target; due to radiation damage, light from the gas appears as a streaking artifact on the lower-right of the image \cite{Blokland2010}. Although this has been corrected by delaying the shutter opening by a few microseconds, the issue has occasionally resurfaced when the beam energy is different than 1 GeV. If these images are collected, they can be identified later by placing a maximum value on the pixels far from the image center, particularly in the lower-right region. Third, there are visible grid lines from the fiber bundle. A Gaussian blur is therefore applied to the image as in Fig.~\ref{fig:target_image}. Finally, there are four dark spots on the image that serve as fiducial markers; they are visible when the beam is large. In this work, the dark spots are left in the image.
%
\begin{figure}[!p]
    \centering
    \vspace*{5cm}
    \includegraphics[width=1.0\textwidth]{Images/chapter4/target_image.png}
    \caption{Image of the beam on the target.}
    \label{fig:target_image}
     \vspace*{5cm}
\end{figure}
%


\subsubsection{Other uses of 2D projections}

There is information to be gained from 2D projections of the distribution in addition to the tomographic 4D reconstruction just described. The projections can be compared to a uniform density ellipse. Additionally, one can observe the variation in the $x$-$y$ correlation coefficient as the difference between the horizontal and vertical phase advances is varied. This reveals any ``hidden" cross-plane correlations, as in Fig.~\ref{fig:tomo_sim_target_scan}. Finally, by computing the RMS moments of the images, the covariance matrix can be reconstructed using the least squares method described in Section \ref{sec:Phase space reconstruction from 1D projections}; the advantage would be that data collection is much faster than for the wire-scanners and that the $\langle xy \rangle$ moment is computed directly.

\chapter{Experiments} \label{chap-5}

[Grammar is unchecked...]
This chapter presents the results of initial experimental studies of elliptical painting at the SNS. The simulations in chapter \ref{chap-2}-\ref{chap-3} were used to guide the experiments, and the diagnostics described in chapter \ref{chap-4} were used to measure the painted distribution. Recall from chapter \ref{chap-3} that solenoid magnetic fields should help the distribution remain close to a Danilov distribution during injection; solenoid magnets were planned to be installed in the SNS ring in 2021, but their installation was delayed until mid-2022, outside the dogmatic of this dissertation. Thus, the beams created in the following experiments were not expected to be optimal. Nonetheless, it was hoped that the beams would be clearly distinguishable from a production beam produced by correlated painting. 


\section{Procedure}

Accelerator physics experiments are performed in the SNS control room using the OpenXAL framework. OpenXAL provides a high-level interface to perform tasks such as changing magnet strengths, triggering the beam, etc. It can also perform single-particle or envelope tracking using an online model of the accelerator. OpenXAL scripts are written in Java or Jython and are executed from the command line. Many graphical user interface (GUI) OpenXAL applications have been developed over the history of the SNS and are available for use in the control room. 

The following steps are taken during the experimental setup:
%
\begin{enumerate}
    \item 
    The beam energy is lowered from 1.0 GeV to 0.8 GeV by turning off several RF cavities at the end of the linac. This is performed by the SNS operations team. Generally, lowering the energy causes other accelerator components to trip or malfunction due to the modified timing system; these must be corrected one-by-one. The first attempt to lower the energy to 0.8 GeV took over six hours.
    %
    \item
    The horizontal and vertical tunes are set to the same value using the Ring Optics Control (ROC) application. ROC varies several quadrupoles until the model tunes are equal to the desired tunes. The tunes are measured using turn-by-turn BPM readings from a single minipulse in the ring. Generally, the measured and model tunes are not quite equal; we therefore shift the ROC input tunes until they agree with the measured tunes. The measured tunes are assumed to be accurate to at least two decimal places.
    %
    \item
    (Optional: Modify the injection region in some way to increase the effective kicker strength.)
    %
    \item
    The eight injection kicker magnets are calibrated using the Ring Injection Control (RIC) application, as described in chapter \ref{chap-1}. 
    %
    \item
    The initial/final kicker voltages are determined to obtain the desired closed-orbit coordinates at the foil, as described in \ref{chap-1}.
    %
    \item
    The initial/final voltages are connected with a square root waveform; the waveform is applied to the injection kickers. The duration of the waveform — the painting time — is chosen at this step but can be easily changed later on. The painting time determines the number of minipulses in the final distribution; i.e., the beam intensity.
    %
    \item
    The number of injected turns before extraction is chosen. This allows the distribution to be measured at different times during injection.  It is also possible to store the beam in the ring, although the SNS normally extracts the beam immediately after accumulation.
    %
\end{enumerate}
%
The next task is to prepare for the measurements. For the wire-scanner measurement, the first step is to modify the RTBT optics using the application developed as part of this dissertation. If the fixed-optics method is used, the optics are changed immediately. If the multi-optics method is used, the optics are pre-computed and stored for later use. The SNS employs a sophisticated machine protection system (MPS) that will cause the machine to trip if the RTBT quadrupole strengths wander outside a certain window, so this window is extended beforehand. Additionally, MPS will activate if the fractional change in field strength is too large; to solve this problem, the field strength is changed in small steps. Wire-scanner data acquisition is performed by the Profile Tools and Analysis (PTA) application. The beam is set to a pulse frequency of 1 Hz, and the beam loss monitors in the wire-scanner region are masked due to the higher-than-normal losses when the wires cross the beam core. After the four wire-scanners complete their scan, a time-stamped file containing the measured profiles along with their statistical properties (mean, standard deviation, etc.). 

The second measurement is the tomographic reconstruction from $x$-$y$ projections on the target. Since the optics calculation is time-consuming, it is generally run in the background while wire-scans are collected; the quadrupole strengths are saved to a file. Since the use the target images for tomographic reconstruction was proposed late in this research, the target scan was only performed in the last of the following experiments.


\section{Experiment 1}

The SNS reserves approximately one day per month for accelerator physics experiments, and various experiments must compete for time within this twenty-four hour period. At the time of our first experiment, setup of the injection region using the RIC application had not yet been completed; although simulations indicated that the kickers were not strong enough to perform elliptical painting at 1 GeV kinetic energy, this had not been tested in reality. And the SNS energy had not yet been decreased — a time-consuming and possibly error-prone task. The goal of Experiment 1 was therefore to push the injected coordinates $x$ and $y'$ to their limits at 1 GeV.

Simulations predict that the distribution will undergo significant change during injection, especially without the presence of solenoid magnets in the ring; therefore, it is interesting to measure the distribution not only at its final state, but also at the intermediate states. Using the fixed-optics method, ten measurements can be performed within one hour. 

\subsection{Correlated painting}

Before setting up for elliptical painting, a production beam was measured for comparison. Recall that a production beam is produced using correlated painting: the displacements at the foil are increased from an initial offset to their final value, and the slope at the foil is always zero. The number of injected turns was reduced from 1000 to 500, and the beam was measured every 50 turns. The measured wire-scanner profiles are shown in Fig.~\ref{fig:exp1a_wsmeas}, and the reconstructed emittances and covariance ellipses are shown in Fig.~\ref{fig:exp1a_emittances}.
%
\begin{figure}[!p]
    \centering
    \begin{subfigure}{\textwidth}
        \includegraphics[width=\textwidth]{Images/chapter5/exp1a/waterfall.png}
    \end{subfigure}
    \vfill
    \vspace*{1.25cm}
    \vfill
    \begin{subfigure}{\textwidth}
        \includegraphics[width=\textwidth]{Images/chapter5/exp1a/rms.png}
    \end{subfigure}
    \caption{Measured wire-scanner profiles during injection for a 1 GeV production beam, 500 injected turns.}
    \label{fig:exp1a_wsmeas}
\end{figure}
%
%
\begin{figure}[!p]
    \centering
    \begin{subfigure}{0.6\textwidth}
        \includegraphics[width=\textwidth]{Images/chapter5/exp1a/emittances.png}
    \end{subfigure}
    \vfill
    \vspace*{0.0cm}
    \vfill
    \begin{subfigure}{0.8\textwidth}
        \includegraphics[width=\textwidth]{Images/chapter5/exp1a/corner.png}
    \end{subfigure}
    \caption{Reconstructed emittances and covariance ellipses of a 1 GeV production beam during injection (500 turns). In this and subsequent figures, light/dark ellipses correspond to the start/end of injection.}
    \label{fig:exp1a_emittances}
\end{figure}
%

Each subplot in Fig.~\ref{fig:exp1a_wsmeas} shows the evolution of the projection onto a single wire during injection; each row corresponds to a different wire-scanner and each column corresponds to a different projection axis — $x$, $y$, or $u$. That the closed orbit starts offset from the foil is evident from initial two peaks in the $x$ and $y$ projections. The distribution forms a ring or donut in $x$-$x'$ and $y$-$y'$, and the hollow center eventually becomes partially filled due to space charge and other nonlinear effects.

The main feature of Fig.~\ref{fig:exp1a_emittances} is that there is very little measured cross-plane correlation in the beam, as expected for the correlated painting method. The error bars are instead calculated by repeating the reconstruction many times with noise added to the measured moments, then taking the mean and standard deviation over the trials. Previous measurements indicate that the moments estimated from the profiles have less than 2\% variation if repeated multiple times. This can lead to asymmetric error bars, which will be shown in the next sections. 

[Simulation]


\subsection{Elliptical-ish painting}

Next, we attempted to carry out elliptical painting. First, [as shown in Fig.~\ref{},] the horizontal and vertical tunes were set to 6.18. Then, next was to move the closed orbit to the foil. This was found to be possible in the vertical plane but impossible in the horizontal plane; the minimum distance from the foil was 10 mm. Additionally, the maximum possible vertical slope was 0.7 mrad. It was decided to continue with the painting method using initial coordinates: ($x$, $x'$, $y$, $y'$) $\approx$ (10 mm, 0 mrad, 0 mm, 0 mrad) and final coordinates: ($x$, $x'$, $y$, $y'$) $\approx$ (21 mm, 0 mrad, 0 mm, 0.7 mrad). The initial beam would be a donut in $x$-$x'$ and a point in $y$-$y'$. In real space, it would be a flat horizontal line with higher density on the two ends of the line. In other words, injected particles would move along an ellipse in the $x$-$y$ plane with zero vertical size. As time progressed, the horizontal and vertical size of the ellipse would grow at different rates depending on the maximum painting $x$ and $y'$ coordinates. This is all assuming linear transport and non-interacting particles. Without a computer, it is unclear what would happen with the inclusion of space charge.

[... We should probably show some screenshots of the RIC application. Maybe not here, but somewhere. ...]

The measured wire-scanner profiles are shown in Fig.~\ref{fig:exp1b_wsmeas}, and the reconstructed emittances and covariance ellipses are shown in Fig.~\ref{fig:exp1a_emittances}.
%
\begin{figure}[!p]
    \centering
    \begin{subfigure}{\textwidth}
        \includegraphics[width=\textwidth]{Images/chapter5/exp1b/waterfall.png}
    \end{subfigure}
    \vfill
    \vspace*{1.25cm}
    \vfill
    \begin{subfigure}{\textwidth}
        \includegraphics[width=\textwidth]{Images/chapter5/exp1b/rms.png}
    \end{subfigure}
    \caption{Measured wire-scanner profiles during injection for a 1 GeV beam. Initial injected coordinates: ($x$, $x'$, $y$, $y'$) $\approx$ (10 mm, 0 mrad, 0 mm, 0 mrad). Final injected coordinates: ($x$, $x'$, $y$, $y'$) $\approx$ (21 mm, 0 mrad, 0 mm, 0.7 mrad).}
    \label{fig:exp1b_wsmeas}
\end{figure}
%
%
\begin{figure}[!p]
    \centering
    \begin{subfigure}{0.6\textwidth}
        \includegraphics[width=\textwidth]{Images/chapter5/exp1b/emittances.png}
    \end{subfigure}
    \vfill
    \vspace*{-0.2cm}
    \vfill
    \begin{subfigure}{0.8\textwidth}
        \includegraphics[width=\textwidth]{Images/chapter5/exp1b/corner.png}
    \end{subfigure}
    \caption{Reconstructed emittances and covariance ellipses for a 1 GeV beam. Initial injected coordinates: ($x$, $x'$, $y$, $y'$) $\approx$ (10 mm, 0 mrad, 0 mm, 0 mrad). Final injected coordinates: ($x$, $x'$, $y$, $y'$) $\approx$ (21 mm, 0 mrad, 0 mm, 0.7 mrad).}
    \label{fig:exp1b_emittances}
\end{figure}
%
Notice the difference in the growth of the vertical beam size in comparison with the horizontal size. The $x$-$x'$ distribution is a donut that fills in over time while also growing in radius, hence the relative lack of growth in the horizontal beam size. The vertical beam size, on the other hand, starts at a small value and increases throughout injection, as desired. Additionally, there is now a clear separation between the intrinsic emittances.

[Some thoughts: Might want to mention something about the comparison with the target images. We seem to be over-estimating the horizontal size and under-estimating the vertical size. Additionally, although the initial image is significantly tilted, the wire-scanner measurement gets the tilt angle wrong. This decreases my trust in these measurements. On the other hand, the $x$-$x'$ reconstruction has produced values very close to the model prediction for a production beam. Also, the intrinsic emittances are a complicated function of the beam moments, so they are probably not so sensitive to these systematic errors. Maybe what we have is some systematic error in the beam size due to the small number of points in the wire-scans?]



\section{Experiment 2}

In Experiment 2, the beam energy was lowered to 0.8 GeV for the first time. The closed orbit was successfuly reached to the foil, and a maximum vertical slope of 1.1 mrad was achieved. 

Before beginning the emittance measurement, we attempted to change the orbit corrector dipoles in the injection region to increase the maximum vertical slope. 

[... Mention failed attempt to optimize the orbit corrector dipoles in the injection region. ...]

%
\begin{figure}[!p]
    \centering
    \begin{subfigure}{\textwidth}
        \includegraphics[width=\textwidth]{Images/chapter5/exp2/waterfall.png}
    \end{subfigure}
    \vfill
    \vspace*{1.25cm}
    \vfill
    \begin{subfigure}{\textwidth}
        \includegraphics[width=\textwidth]{Images/chapter5/exp2/rms.png}
    \end{subfigure}
    \caption{Measured wire-scanner profiles during injection for a 0.8 GeV beam. Initial injected coordinates: ($x$, $x'$, $y$, $y'$) $\approx$ (0 mm, 0 mrad, 0 mm, 0 mrad). Final injected coordinates: ($x$, $x'$, $y$, $y'$) $\approx$ (21 mm, 0 mrad, 0 mm, 1.1 mrad).}
    \label{fig:exp2_wsmeas}
\end{figure}
%

%
\begin{figure}[!p]
    \centering
    \begin{subfigure}{0.6\textwidth}
        \includegraphics[width=\textwidth]{Images/chapter5/exp2/emittances.png}
    \end{subfigure}
    \vfill
    \vspace*{-0.2cm}
    \vfill
    \begin{subfigure}{0.8\textwidth}
        \includegraphics[width=\textwidth]{Images/chapter5/exp2/corner.png}
    \end{subfigure}
    \caption{Reconstructed emittances and covariance ellipses of a 0.8 GeV beam during injection. Initial injected coordinates: ($x$, $x'$, $y$, $y'$) $\approx$ (0 mm, 0 mrad, 0 mm, 0 mrad). Final injected coordinates: ($x$, $x'$, $y$, $y'$) $\approx$ (21 mm, 0 mrad, 0 mm, 1.1 mrad).}
    \label{fig:exp2_emittances}
\end{figure}
%

\section{Experiment 3}


%
\begin{figure}[!p]
    \centering
    \begin{subfigure}{\textwidth}
        \includegraphics[width=\textwidth]{Images/chapter5/exp3/waterfall.png}
    \end{subfigure}
    \vfill
    \vspace*{1.25cm}
    \vfill
    \begin{subfigure}{\textwidth}
        \includegraphics[width=\textwidth]{Images/chapter5/exp3/rms.png}
    \end{subfigure}
    \caption{Measured wire-scanner profiles during injection for a 0.8 GeV beam. Initial injected coordinates: ($x$, $x'$, $y$, $y'$) $\approx$ (0 mm, 0 mrad, 0 mm, 0 mrad). Final injected coordinates: ($x$, $x'$, $y$, $y'$) $\approx$ (31 mm, 0 mrad, 0 mm, 1.1 mrad).}
    \label{fig:exp3_wsmeas}
\end{figure}
%

%
\begin{figure}[!p]
    \centering
    \begin{subfigure}{0.6\textwidth}
        \includegraphics[width=\textwidth]{Images/chapter5/exp3/emittances.png}
    \end{subfigure}
    \vfill
    \vspace*{-0.2cm}
    \vfill
    \begin{subfigure}{0.8\textwidth}
        \includegraphics[width=\textwidth]{Images/chapter5/exp3/corner.png}
    \end{subfigure}
    \caption{Reconstructed emittances and covariance ellipses of a 0.8 GeV beam during injection. Initial injected coordinates: ($x$, $x'$, $y$, $y'$) $\approx$ (0 mm, 0 mrad, 0 mm, 0 mrad). Final injected coordinates: ($x$, $x'$, $y$, $y'$) $\approx$ (31 mm, 0 mrad, 0 mm, 1.1 mrad).}
    \label{fig:exp3_emittances}
\end{figure}
%



\section{Experiment 4}



\section{Summary}
\chapter{Conclusion} \label{chap-6}

The following is a summary of this work.
%
\begin{itemize}
    %
    \item The attractive properties of the self-consistent Danilov distribution were described, as well as elliptical painting — a method to produce an approximate Danilov distribution in a circular accelerator.
    %
    \item The dynamics of the Danilov distribution in linear focusing systems with space charge were investigated using envelope equations. An iterative algorithm was developed to find the matched beam in coupled or uncoupled focusing systems. The matched beam in the SNS was calculated, placing constraints on the elliptical painting method in the SNS. 
    %
    \item A previous simulation of elliptical painting in the SNS was revisited, more clearly defining the best expected case in the SNS in terms of the measurable beam parameters such as the intrinsic emittances.
    %
    \item Methods to measure the 4D transverse phase space distribution were identified and implemented in the SNS using existing diagnostics. The first method was to use 1D projections from wire-scanners; the machine optics were adjusted to reduce the sensitivity to errors and minimize the measurement time. The second method was to use 2D projections on the target to reconstruct the phase space distribution. 
    %
    \item Elliptical painting was carried out for the first time in the SNS. The painted distribution was measured throughout accumulation and compared to simulations. In the final experiment, the measured four-dimensional emittance was reduced relative to a distribution produced by normal injection methods. Simulations indicated that the current setup is sensitive to the tune split in the ring, but that future modifications to the ring should bring the beam closer to a self-consistent state.
\end{itemize}
%

Work will continue at the SNS. It may be possible to increase the maximum injection angle by modifying the foil position, orbit corrector dipoles, Chicane dipoles, or HEBT trajectory, which would allow a larger, rounder beam to be painted and potentially allow elliptical painting at the nominal beam energy of 1 GeV. There may also be some use for the skew quadrupole correctors in the ring to reduce the required angular kicks. Additionally, solenoids, which will be installed in the SNS ring in late 2022, should stabilize the distribution against nonlinearities and reduce the sensitivity to the tune split in the ring. If the beam is measured and found to be close to a Danilov distribution, it will be measured as it is stored in the ring after accumulation to determine its stability. Such experiments could address some of the questions in \cite{Burov2013}.

Diagnostics can be improved. The electron-scanner, which can measure turn-by-turn 1D projections of the distribution in real-time, will be recommissioned soon; although the phase space distribution cannot be reconstructed from these projections, they can be compared to the projections of a uniform density ellipse. For the beam images on the target, it would be ideal to use a camera instead of a fiber bundle and to eliminate the effect of the fiducial markers to produce a cleaner image. Additionally, the methods described in Chapter \ref{chap-4} can be used to reconstruct the 4D phase space distribution from the target images; unfortunately, a quality set of beam images was not collected in this work. 

Related research problems may be pursued in the future. For example, it may be possible to design a lattice with so-called ``circular mode optics" that keep the beam as round as possible throughout the ring, with a very small 4D emittance, using locally rotational-invariant optics \cite{Burov2002}. This problem is being studied in a collaboration between the SNS, Argonne National Laboratory, and Fermi National Accelerator Laboratory \cite{Morozov-forthcoming}. Parallel design studies could be performed to optimize the injection region for elliptical painting. Realistic simulations of elliptical painting should be carried out with such an injection region, and losses should be compared to simulations of correlated and/or anti-correlated painting, especially as the beam intensity is scaled, to see if elliptical painting could outperform these methods.

There are also theoretical problems related to the Danilov distribution and/or self-consistent distributions in general. The stability properties of the envelope equations, studied in detail for the KV distribution in \cite{Lund2004}, have not yet been studied specifically for the Danilov distribution. There will be so-called odd modes, or tilting modes, due to the cross-plane correlation in the beam. A framework to study the stability of all second-order modes was presented in \cite{Yuan2017}. 3D envelope equations have also been studied using the KV model \cite{Qiang2018}, but they are not closed since the KV distribution does not exist in three spatial dimensions; several self-consistent distributions were derived in three spatial dimensions in \cite{Danilov2003}, and the resulting closed set of 3D envelope equations may be of interest. Finally, the high-order coherent instabilities present in the KV distribution can be investigated numerically for the Danilov distribution and/or other self-consistent distributions.


%-------------------------------------------------------------------------------
%	BIBLIOGRAPHY
%-------------------------------------------------------------------------------

\addtocontents{toc}{\vspace{2em}} % Add a gap in the Contents, for aesthetics
\unnumberedchapter{Bibliography} % Title of the unnumbered chapter
\bibliography{Preamble/Thesis_bibliography} % The references information are stored in the file named "Thesis_bibliography.bib"

%-------------------------------------------------------------------------------
%	APPENDICES (optional)
%-------------------------------------------------------------------------------

\addtocontents{toc}{\vspace{2em}} % Add a gap in the Contents, for aesthetics
\appendix

\numberedchapter % Regular chapters following
\chapter{Nonlinear resonances} \label{app-A}

The study of nonlinear resonances is important in many areas of physics \cite{Reichl1992}. A derivation of the nonlinear resonance condition in Eq.~\ref{eq:resonance_lines} (in one dimension) is included in this appendix. The derivation follows \cite{LundLecture1} closely.

We return to one-dimensional motion and write
%
\begin{equation}\label{eq:Hill_nonlinear}
    x'' + k(s) x = \Delta B,
\end{equation}
%
where $\Delta B$ represents all the nonlinear terms in the magnetic field expansion (and also linear deviations from the design fields). The stable solution $x_0$ when $\Delta B = 0$ is given by Eq.~\eqref{eq:Hill_solution}. We now define
%
\begin{equation}
    \phi(s) = \frac{1}{\nu} \oint{\frac{ds}{\beta(s)}},
\end{equation}
%
where $\nu$ is the tune. Moving to the normalized coordinate $u = x / \sqrt{\beta}$, with $\dot{u} = du/d\phi$ we have
%
\begin{equation}\label{eq:pert1}
    \ddot{u} + \nu^2 u = -\nu^2 \sum_{n=0}^{\infty}{\left(\beta^{\frac{n+3}{2}} b_{n+1}\right) u^n}.
\end{equation}
%
$\beta$ (the oscillation amplitude of the unperturbed motion) and $b_n$ (a multipole coefficient) are periodic in $\phi$ since they depend only on the position in the ring. Grouping these terms and Fourier expanding gives
%
\begin{equation}
    \ddot{u} + \nu^2 u = -\nu^2 \sum_{n=0}^{\infty}\sum_{k=-\infty}^{\infty} C_{n,k} \, u^n \, e^{ik\phi}.
\end{equation} 
%
We then perturb around $u_0$, the solution to the homogeneous equation, writing $u = u_0 + \delta u$, and keep only linear powers of $\delta u$. 
%
\begin{equation}
    \ddot{\delta u} + \nu^2 \delta u \approx -\nu^2 \sum_{n=0}^{\infty}\sum_{k=-\infty}^{\infty} C_{n,k} \, u_0^n \, e^{ik\phi}.
\end{equation}
%
Noting that
%
\begin{equation}
    u_0^n \propto \cos^n(\nu\phi) = \frac{1}{2^n}\sum_{m=0}^{n} \binom{n}{m} e^{i(n-2m)\nu\phi},
\end{equation} 
%
leads to
%
\begin{equation}\label{eq:pert2}
    \ddot{\delta u} + \nu^2 \delta u \approx -\nu^2 \sum_{n=0}^{\infty}\sum_{k=-\infty}^{\infty} \sum_{m=0}^{n} {n \choose m} \frac{C_{n,k}}{2^n} e^{i\left[(n - 2m)\nu + k\right]\phi}.
\end{equation}
%
A resonance condition may occur when any of the frequency components of the driving terms are close to the tune $\nu$; i.e., when
%
\begin{equation}
    (n - 2m)\nu + k = \pm \nu.
\end{equation}
%
Dipole terms correspond to integer tunes, quadrupole terms to 1/2 integer tunes, sextupole terms to 1/3 integer tunes, and so on. The same is true in the vertical dimension. The inclusion of coupling between $x$ and $y$ leads to the following resonance conditions:
%
\begin{equation}\label{eq:resonance_lines1}
    M_x \nu_x + M_y \nu_y = N,
\end{equation}
%
where $M_x$, $M_y$, and $N$ are integers and $|M_x| + |M_y|$ is the order of the resonance. These resonance lines are plotted in Fig.~\ref{fig:resonance_lines}.
%
\begin{figure}[!p]
    \centering
    \includegraphics[width=\textwidth]{Images/chapter1/resonance_lines.png}
    \caption{Resonance lines in tune space defined by Eq.~\eqref{eq:resonance_lines1}.}
    \label{fig:resonance_lines}
\end{figure}
%
\begin{figure}[!p]
    \begin{subfigure}[b]{1.0\textwidth}
        \includegraphics[width=\textwidth]{Images/chapter1/sextupole.png}
        \label{fig:sextupole_a}
    \end{subfigure}
    \vfill
    \vspace*{1.0cm}
    \vfill
    \begin{subfigure}[b]{\textwidth}
        \centering
        \includegraphics[width=\textwidth]{Images/chapter1/sextupole_second_order.png}
        \label{fig:sextupole_b}
    \end{subfigure}
    \caption{Third-order (top) and fourth/fifth-order (bottom) resonances excited by a sextupole perturbation to a linear lattice. (Adapted from \cite{Lee2011}.)}
    \label{fig:sextupole}
\end{figure}
%

It is helpful to visualize the particle trajectory when a resonance line is encountered; therefore, a numerical experiment from \cite{Lee2011} is reproduced here. We consider a sextupole perturbation in an otherwise linear lattice, modeling the sextupole as a thin-lens kick. The turn-by-turn trajectories of particles with several different initial amplitudes are plotted in the top row of \ref{fig:sextupole} for different tunes $\nu_x$. The third-order resonance leads to a well-known triangular region of stability as the tune approaches 2/3. The bottom plot reveals fourth and fifth-order resonances only obtained from second-order perturbation analysis. 
% \chapter{Space charge resonances and instabilites} \label{app-B}

In Chapter \ref{chap-1}, it was mentioned that space charge can be divided into two categories: incoherent effects involving the motion of single particles, and coherent effects involving the self-consistent motion of the entire beam. Here, these effects are explored by carrying out several of the numerical experiments in \cite{Hofmann2017Book} using the PyORBIT code. 


\section*{Incoherent space charge resonances}

We first assume that the beam is matched — i.e., oscillates with the same periodicity as the external focusing — and track a particle in the field of the matched beam. In Chapter \ref{chap-1}, we stated that the primary concern in circular accelerators is that the shifted single-particle tunes cross low-order machine resonance lines. But it is also possible for the beam's electric field to drive single-particle resonances \cite{Holmes1999, Jeon1999, Li2014, Kojima2019, Asvesta2020}. For suppose the transverse electric field is expanded in powers of $x$ and $y$: these so-called ``psuedo-multipoles" can then be treated in a similar way to the magnetic multipoles in Appendix \ref{app-A}. For illustration, we reproduce a numerical study from \cite{Hofmann2017Book} using PyORBIT. Fig.~\ref{fig:incoherent_instability} shows a simulation of a truncated Gaussian distribution in a FODO lattice as the zero-current tune is decreased from 100\degree to 90\degree over 500 cells. The initial distribution has equal emittances in both planes and is matched to the lattice with a depressed tune of 92\degree.
%
\begin{figure}[!p]
    \centering
    \vspace*{2cm}
    \begin{subfigure}{\textwidth}
        \includegraphics[width=\textwidth]{Images/chapter1/incoherent_resonance_fourth_order.png}
        \label{fig:incoherent_instability_a}
        \caption{}
    \end{subfigure}
    \begin{subfigure}{0.5\textwidth}
        \includegraphics[width=\textwidth]{Images/chapter1/incoherent_resonance_fourth_order_emittance.png}
        \label{fig:incoherent_instability_b}
        \caption{}
    \end{subfigure}
    \caption{Simulation of a truncated Gaussian distribution in a FODO lattice. The zero-current tune is decreased from 100\degree to 90\degree over 500 cells. (a) $x$-$x'$ distribution. (b) RMS horizontal emittance. (Reproduced from \cite{Hofmann2017Book}.)}
    \label{fig:incoherent_instability}
    \vspace*{2cm}
\end{figure}
%
A fourth-order resonance is excited as the depressed tune approaches 90 degrees. The smooth emittance growth during most of the simulation shows that the core of the beam remains matched, justifying the use of ``incoherent" describe the resonance. Higher-order resonances can also occur for different combinations of beam intensity and focusing strength.





\section*{Coherent instabilities}

In \cite{Hofmann1983}, Hofmann et al. analytically studied perturbations of a round ($\varepsilon_x = \varepsilon_y$) KV distribution using the Vlasov equation in one of the simplest time-dependent cases: a FODO lattice with equal horizontal and vertical tunes. The result is shown in Fig.~\ref{fig:stopbands}, which plots the depressed tune as a function of beam intensity. Each thin line represents a different zero-current tune, and the thick lines represent regions of instability.
%
\begin{figure}[!p]
    \centering
    \includegraphics[width=\textwidth]{Images/chapter1/stopbands_hor.png}
    \caption{Instability stopbands obtained from perturbations of a KV distribution with equal emittances in a FODO lattice. (From \cite{Hofmann1983}).}
    \label{fig:stopbands}
\end{figure}
%

The second-order instabilities involve linear forces only, so they should appear in the KV envelope equations. We use a FODO cell with a zero-current tune of $100\degree$ corresponding to the second-to-bottom line on the left-most plot in Fig.~\ref{fig:stopbands}. The initial distribution is first matched to the lattice, then tracked for 500 cells by integrating the KV envelope equations. Fig.~\ref{fig:envelope_instability} shows the horizontal and vertical envelopes as the depressed KV tune is decreased from $90\degree$ to $71\degree$, crossing the stopband. This is known as the envelope instability. 
%
\begin{figure}[!p]
    \centering
    \includegraphics[width=\textwidth]{Images/chapter1/envelope_instability.png}
    \caption{Integrated KV envelope equations in a FODO lattice as the depressed KV tune $\nu_x$ is decreased. The zero-current tune is 100\degree.} \label{fig:envelope_instability}
\end{figure}
%

Observation of the higher-order stopbands requires PIC simulation. We choose a zero-current tune of 90\degree; according to Fig.~\ref{fig:stopbands}, a third-order and fourth-order instability should occur at a depressed tune of 45\degree and 30\degree, respectively. Fig.~\ref{fig:coherent_instabilities} shows the simulated evolution in PyORBIT for three different distributions: KV, Waterbag, and Gaussian. (Note that while the simulations in \cite{Hofmann2017Book} used a bunched beam, coasting beams were used in this simulation.)
%
\begin{figure}[!p]
    \begin{subfigure}[b]{0.45\textwidth}
        \includegraphics[width=\textwidth]{Images/chapter1/coherent_instability_fourth_order.png}
        \label{fig:coherent_instabilities_a}
    \end{subfigure}
    \hfill
    \begin{subfigure}[b]{0.45\textwidth}
        \includegraphics[width=\textwidth]{Images/chapter1/coherent_instability_third_order.png}
        \label{fig:f2}
    \end{subfigure}
    \vfill
    \begin{subfigure}[b]{\textwidth}
        \centering
        \includegraphics[width=0.8\textwidth]{Images/chapter1/coherent_instability_emittances.png}
        \label{fig:coherent_instabilities_b}
    \end{subfigure}
    \caption{Simulated Gaussian, Waterbag, and KV distributions in a FODO lattice with a zero-current tune of 90\degree and depressed KV tunes of $\nu_x$ = 45\degree (top left) and 30\degree (top right).}
    \label{fig:coherent_instabilities}
\end{figure}
%
The instabilities violently affect the KV distribution, but their effect is less pronounced in the other distributions. Thus, it is assumed that high-order coherent instabilities, while interesting, are not important in typical beams with large tune spreads.

The Danilov distribution and/or other self-consistent distributions may exhibit similar coherent instabilities. Lund, Kikuchi, and Davidson noted this in 2009 \cite{Lund2009}:
\begin{quote}
    \textit{Although the low-order properties of the KV distribution are appealing physically, the full four-dimensional structure corresponds to a singular, hyperellipsoidal shell in phase space. For strong space charge, this singular structure drives unphysical, higher-order instabilities which limit practical use of the KV distribution for initializing simulations. The KV distribution is the only exact Vlasov equilibrium known that is a function of linear-field Courant-Snyder invariants. Danilov et al. \cite{Danilov2003} investigate alternative classes of exact kinetic equilibrium distributions for linear forces. These distributions are highly singular, and based on elementary plasma physics considerations, can be expected to be unstable (similar to the KV distribution) in regimes of high space-charge intensity.}
\end{quote}
% \chapter{Fringe field compensation using solenoids} \label{app-C}

It is worth illuminating the finding from \cite{Holmes2018} that in an otherwise linear lattice, fringe fields tend to eliminate any cross-plane correlations in the beam when the tunes are near the difference resonance $\nu_x \approx \nu_y$. To demonstrate this, a Danilov distribution matched to a linearized version of the SNS ring was generated. Fringe fields were then turned on, and the particles were tracked without space charge. Fig.~\ref{fig:fringe_a} shows the turn-by-turn evolution of the distribution.

\begin{figure}[!p]
    \centering
    \includegraphics[width=0.7\textwidth]{Images/chapter3/fringe.png}
    \caption{Danilov distribution tracked in the SNS ring. Fringe fields are the only nonlinear external effect.}
    \label{fig:fringe_a}
    \vspace*{3cm}
\end{figure}

\begin{figure}[!p]
    \centering
    \includegraphics[width=0.7\textwidth]{Images/chapter3/fringe_solenoid.png}
    \caption{Danilov distribution tracked in the SNS ring with a solenoid added to the ring. Fringe fields are the only nonlinear external effect.}
    \label{fig:fringe_b}
    \vspace*{3cm}
\end{figure}

\begin{figure}[!p]
    \centering
    \includegraphics[width=0.7\textwidth]{Images/chapter3/fringe_spacecharge.png}
    \caption{Danilov distribution tracked in the SNS ring with space charge. Fringe fields are the only nonlinear external effect.}
    \label{fig:fringe_c}
    \vspace*{3cm}
\end{figure}

There is nonlinear coupling between the horizontal and vertical motion, and the final distribution is a superposition of rotating and counter-rotating modes. In Fig.~\ref{fig:fringe_b}, a solenoid magnet is added to the ring. The cross-plane correlations are now mostly maintained. The tunes $\nu_{1, 2}$ are no longer equal due to the linear coupling from the solenoid, so the resonance condition is avoided. In Fig.~\ref{fig:fringe_c}, the simulation is repeated with the inclusion of space charge instead of the solenoid magnet. An intensity of $10^{14}$ is used and the bunch length is equal to the ring length. It appears that a Danilov distribution will self-stabilize against the difference resonance when space charge is included.

It is recommended, however, that solenoid magnets be added to the ring to carry out this painting scheme. The difficulty is that the fringe fields dominate at the beginning of injection, when the transverse displacement is maximum, before space charge has a chance to stabilize the beam. Additionally, without a solenoid in the ring, the quality of the final distribution is very sensitive to the difference in horizontal and vertical tunes. With a solenoid in the ring, elliptical trajectories at the injection point are produced no matter the original horizontal and vertical tunes. 

\end{document}
