\unnumberedchapter{Abstract} 
\chapter*{Abstract} 

A self-consistent phase space distribution is a charged particle beam in which the electric field has a linear dependence on the particle coordinates, and furthermore, in which the linearity of the electric field is conserved as the beam is transported through arbitrary linear focusing fields. These features would minimize/eliminate the space charge tune shift/spread in a circular accelerator, increasing the possible beam intensity. Additionally, their uniform density would be ideal for fixed-target applications. Finally, in some cases, the special relationships between their phase space coordinates could be exploited to produce flat beams. 

Although self-consistent distributions are often used in theoretical studies, they are not assumed to be realistic. Yet simulations predict that at least one — the Danilov distribution — could be approximately produced in a real machine using a method called elliptical painting. This dissertation contributes to efforts to test this prediction in the Spallation Neutron Source (SNS). First, the beam envelope model was used to calculate the matched solutions of the Danilov distribution in periodic focusing channels, placing constraints on the elliptical painting method. Second, several methods to indirectly measure the four-dimensional (4D) phase space distribution of an accumulated beam in the SNS were identified, implemented using existing diagnostics, and optimized, allowing the comparison of real beams to the Danilov model in minimal time. Finally, three initial experiments to produce a Danilov distribution in the SNS were carried out. Although the experiments were performed under suboptimal conditions due to current hardware constraints, the measured reduction in 4D emittance was not insignificant in the final experiment, indicating that the beam was closer to the desired self-consistent case than a typical beam in the SNS. Simulations were included to benchmark the measurements, resulting in qualitative agreement and recommendations for future experiments. Small modifications to the SNS ring lattice are expected to bring the beam closer to a self-consistent state.
