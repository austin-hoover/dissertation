\unnumberedchapter{Abstract} 
\chapter*{Abstract} 

Maintaining low levels of beam loss will be increasingly difficult in future high-intensity accelerators. One limitation in circular accelerators is the space charge tune shift — the damping of single-particle oscillations due to the beam's electric field — which, when combined with the periodic electromagnetic fields in the accelerator, can lead to resonant oscillations and eventual particle loss. This problem is minimized if the beam's electric field has a linear dependence on the particle coordinates, but such a beam is difficult to realize and/or maintain in realistic conditions.

Self-consistent phase space distributions are beams in which the electric field has a linear dependence on the particle coordinates, and furthermore, in which the linearity of the electric field is conserved as the beam is transported through arbitrary linear focusing fields. Although such models are often used in theoretical analysis, simulations predict that at least one self-consistent distribution — the Danilov distribution — could be approximately produced in a real machine — the Spallation Neutron Source (SNS) — using a method called elliptical painting. This dissertation contributes to efforts to test this prediction in the SNS. 

First, the beam envelope model was employed to improve understanding of the dynamics of the Danilov distribution with space charge and place constraints on the elliptical painting method. Second, several existing methods to indirectly measure the four-dimensional (4D) phase space distribution of a fully-accumulated beam in the SNS were identified, implemented using existing diagnostics, and optimized, allowing comparison of real beams with the ideal Danilov model. Finally, three initial experiments to produce a Danilov distribution in the SNS were carried out. The experiments were performed under suboptimal conditions due to current hardware constraints; nonetheless, the reduction of the measured 4D beam emittance in the final experiment was not insignificant, indicating that the beam was closer to the desired case than a typical beam in the SNS. Particle-in-cell simulations were included to benchmark the measurements, resulting in qualitative agreement. Small modifications to the SNS are expected to bring the beam closer to a self-consistent state.