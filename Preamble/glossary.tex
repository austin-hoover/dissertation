\unnumberedchapter{Glossary} 
\chapter*{Glossary} 

Because accelerator physics is a small branch of applied physics, and because this dissertation introduces several terms that are not frequently used in the accelerator physics community, the following glossary has been included for the reader.

\textbf{Beta function} — A function that scales the amplitude of single-particle oscillations in the linear approximation.

\textbf{Beam envelope} — The root-mean-square ellipsoid defined by the covariance matrix.

\textbf{Beam perveance} — A dimensionless measure of space charge strength.

\textbf{Circular mode} — A beam with small four-dimensional emittance. It could also refer to the circular motion of the eigenvectors of a coupled transfer matrix.

\textbf{Courant-Snyder ellipse} — Particles move along this ellipse in the linear approximation. Its area is conserved.

\textbf{Danilov distribution} — A self-consistent distribution in two spatial dimensions. It is characterized by an elliptical shape, uniform charge density, and zero four-dimensional emittance.

\textbf{Effective lattice} —

\textbf{Emittance} — The root-mean-square volume or area of a phase space distribution.

\textbf{Kapchinskij-Vladimirskij (KV) distribution} — A self-consistent distribution in two spatial dimensions. Its particles are uniformly distributed on an ellipsoid in four-dimensional phase space.

\textbf{Matched beam} — A beam whose envelope oscillates with the same periodicity as the external focusing.

\textbf{Ring} — A circular accelerator.

\textbf{Painting} — An beam injection method in which the relative transverse distance and angle between the circulating and injected beam is varied.  

\textbf{Phase advance} — The integral of the inverse of the beta function. 

\textbf{Self-consistent distribution} — A phase space distribution which produces linear space charge forces under any linear transformation of the coordinates.

\textbf{Space charge} — The charge density of a beam in free space.

\textbf{Tune} — The number of phase space oscillations per performed by a single particle in one turn around a ring.

\textbf{Space charge tune shift} — The reduction in tune caused by the beam space charge. The linear space charge component results in the same reduction for every particle (tune shift), while the nonlinear component results in an amplitude-dependent reduction (tune spread).