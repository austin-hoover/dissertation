\chapter{Conclusion} \label{chap-6}

A brief summary of this work:
%
\begin{itemize}
    %
    \item The continued study of self-consistent phase space distributions, particularly the Danilov distribution, was motivated by making a connection to the recent interest in circular modes.
    %
    \item The dynamics of the Danilov distribution in linear focusing systems with space charge were investigated using envelope equations. An iterative algorithm was developed to find the matched solution in coupled or uncoupled focusing systems. The matched solution in the SNS was computed.
    %
    \item Particle-in-cell simulations of injection in the SNS were extended to include new experimental constraints. 
    %
    \item Methods to measure the four-dimensional phase space distribution were implemented in the SNS. The first method was to used 1D projections from wire-scanners. The machine optics were adjusted to reduce the sensitivity to errors and minimize the measurement time. The second method was to use 2D projections on the target to reconstruct the phase space distribution. 
    %
    \item An injection method called elliptical phase space painting was carried out for the first time in the SNS. The resulting distribution was measured and compared to simulation. In the final experiment, the measured four-dimensional emittance was significantly reduced relative to a distribution produced by traditional injection methods.
\end{itemize}
%

Related research problems may be pursued in the future. First, experiments will continue at the SNS. It may be possible to optimize the elliptical painting method by moving the foil position, using orbit corrector dipoles, or using skew quadrupoles in the ring. Additionally, a solenoid will be added to the SNS ring in the year 2022. The measurements described in this work will then be repeated. If the measured beam is close to a Danilov distribution, it would be interesting to measure the beam as it is stored in the ring after injection. There are several questions listed in \cite{Burov2013} about the use of circular modes in a collider that may be able to be addressed in SNS experiments.

Second, it may be possible to design a lattice with so-called ``circular mode optics" that keep the beam as round as possible throughout the ring, minimizing the space charge tune shift. This problem is being investigated by researchers at Argonne National Laboratory. 

Third, diagnostics can be improved. The electron-scanner, which has the ability to measure turn-by-turn 1D projections of the distribution in real time, will be recommissioned soon; although the phase space distribution cannot be reconstructed from these projections, they can be compared to the projections of a uniform density ellipse. For the tomographic reconstruction at the target, it would be ideal to use a camera instead of a fiber bundle and to eliminate the fiducial markers to produce a cleaner image. Other algorithms could be applied to the 4D reconstruction, such as MENT [Ref: Wong].

Lastly, there are theoretical problems related to the Danilov distribution and/or self-consistent distributions in general. The stability properties of the envelope equations, studied in detail for the KV distribution in \cite{Lund2004}, have not yet been studied specifically for the Danilov distribution. There will be so-called odd modes, or tilting modes, due to the cross-plane correlation in the beam. A framework to study the stability of all second-order modes was presented in \cite{Yuan2017}. Additionally, 3D envelope equations have also been studied using the KV model [Ref], but they are not closed since the KV distribution does not exist in three spatial dimensions. Several self-consistent distributions were derived in three spatial dimensions in \cite{Danilov2003}, and the resulting closed set of 3D envelope equations may be of interest. Finally, the high-order coherent instabilities present in the KV distribution can be investigated for the Danilov distribution.