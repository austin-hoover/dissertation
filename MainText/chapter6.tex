\chapter{Conclusion} \label{chap-6}

The following is a summary of this work.
%
\begin{itemize}
    %
    \item The attractive properties of the self-consistent Danilov distribution were described, as well as elliptical painting — a method to produce an approximate Danilov distribution in a circular accelerator.
    %
    \item The dynamics of the Danilov distribution in linear focusing systems with space charge were investigated using envelope equations. An iterative algorithm was developed to find the matched beam in coupled or uncoupled focusing systems. The matched beam in the SNS was calculated, placing constraints on the elliptical painting method in the SNS. 
    %
    \item A previous simulation of elliptical painting in the SNS was revisited, more clearly defining the best expected case in the SNS in terms of the measurable beam parameters such as the intrinsic emittances.
    %
    \item Methods to measure the 4D transverse phase space distribution were identified and implemented in the SNS using existing diagnostics. The first method was to use 1D projections from wire-scanners; the machine optics were adjusted to reduce the sensitivity to errors and minimize the measurement time. The second method was to use 2D projections on the target to reconstruct the phase space distribution. 
    %
    \item Elliptical painting was carried out for the first time in the SNS. The painted distribution was measured throughout accumulation and compared to simulations. In the final experiment, the measured four-dimensional emittance was reduced relative to a distribution produced by normal injection methods. Simulations indicated that the current setup is sensitive to the tune split in the ring, but that future modifications to the ring should bring the beam closer to a self-consistent state.
\end{itemize}
%

Work will continue at the SNS. It may be possible to increase the maximum injection angle by modifying the foil position, orbit corrector dipoles, Chicane dipoles, or HEBT trajectory, which would allow a larger, rounder beam to be painted and potentially allow elliptical painting at the nominal beam energy of 1 GeV. There may also be some use for the skew quadrupole correctors in the ring to reduce the required angular kicks. Additionally, solenoids, which will be installed in the SNS ring in late 2022, should stabilize the distribution against nonlinearities and reduce the sensitivity to the tune split in the ring. If the beam is measured and found to be close to a Danilov distribution, it will be measured as it is stored in the ring after accumulation to determine its stability. Such experiments could address some of the questions in \cite{Burov2013}.

Diagnostics can be improved. The electron-scanner, which can measure turn-by-turn 1D projections of the distribution in real-time, will be recommissioned soon; although the phase space distribution cannot be reconstructed from these projections, they can be compared to the projections of a uniform density ellipse. For the beam images on the target, it would be ideal to use a camera instead of a fiber bundle and to eliminate the effect of the fiducial markers to produce a cleaner image. Additionally, the methods described in Chapter \ref{chap-4} can be used to reconstruct the 4D phase space distribution from the target images; unfortunately, a quality set of beam images was not collected in this work. 

Related research problems may be pursued in the future. For example, it may be possible to design a lattice with so-called ``circular mode optics" that keep the beam as round as possible throughout the ring, with a very small 4D emittance, using locally rotational-invariant optics \cite{Burov2002}. This problem is being studied in a collaboration between the SNS, Argonne National Laboratory, and Fermi National Accelerator Laboratory \cite{Morozov-forthcoming}. Parallel design studies could be performed to optimize the injection region for elliptical painting. Realistic simulations of elliptical painting should be carried out with such an injection region, and losses should be compared to simulations of correlated and/or anti-correlated painting, especially as the beam intensity is scaled, to see if elliptical painting could outperform these methods.

There are also theoretical problems related to the Danilov distribution and/or self-consistent distributions in general. The stability properties of the envelope equations, studied in detail for the KV distribution in \cite{Lund2004}, have not yet been studied specifically for the Danilov distribution. There will be so-called odd modes, or tilting modes, due to the cross-plane correlation in the beam. A framework to study the stability of all second-order modes was presented in \cite{Yuan2017}. 3D envelope equations have also been studied using the KV model \cite{Qiang2018}, but they are not closed since the KV distribution does not exist in three spatial dimensions; several self-consistent distributions were derived in three spatial dimensions in \cite{Danilov2003}, and the resulting closed set of 3D envelope equations may be of interest. Finally, the high-order coherent instabilities present in the KV distribution can be investigated numerically for the Danilov distribution and/or other self-consistent distributions.
